% HaM15 Numerical solution of DDAES
% Phi 22.01.14
% Volker 09.03.14
% Volker 3.6.14
% Phi 28.10.14
% Volker 29.10.14
% Phi 06.11.14
% Phi 09.12.14
% Volker 17.12.14
% Phi 13.01.15
\documentclass[final]{siamltex}

\usepackage{amsfonts,epsfig}
\usepackage{amsmath,amssymb} %$#
%\usepackage{graphicx}

\usepackage[square,sort&compress,comma,numbers]{natbib}  % Use the "Natbib" style for the references in the Bibliography
\def\bibfont{\footnotesize}

\usepackage{algorithm}
\usepackage{algpseudocode}
\usepackage{algcompatible}

% \usepackage{paralist}
% \renewenvironment{itemize}[1]{\begin{compactitem}#1}{\end{compactitem}}
% \renewenvironment{enumerate}[1]{\begin{compactenum}#1}{\end{compactenum}}
% \renewenvironment{description}[0]{\begin{compactdesc}}{\end{compactdesc}}

\usepackage{showlabels}
\renewcommand{\showlabelfont}{\small\slshape\color{red}}
\usepackage{lineno}
\linenumbers
\usepackage[colorlinks,pdfpagelabels]{hyperref} %$#
\hypersetup{citecolor=blue}


% Common extra environments
\newtheorem{thm}[theorem]{Theorem}

\newtheorem{rem}[theorem]{Remark}
\newtheorem{lem}[theorem]{Lemma}
\newtheorem{defi}[theorem]{Definition}
\newtheorem{exa}[theorem]{Example}

\newtheorem{ass}[theorem]{Assumption}
\newtheorem{hyp}[theorem]{Hypothesis}
\newtheorem{pro}[theorem]{Procedure}

%=====================================================================================================================================
%
%opening
\begin{document}


%=====================================================================================================================================
%
% new def-s and commands
\input{new_command}
%

%=====================================================================================================================================

\title{Perturbation and stability analysis of Linear Delay Differential-Algebraic Equations\footnotemark[1]}

%\title{Reformulation and Numerical solution of general nonlinear Delay-Differential-Algebraic Equations \footnotemark[1]}

\author{Phi Ha\footnotemark[2] \and Volker Mehrmann\footnotemark[3] \and Caren Tischendorf\footnotemark[2]}

\renewcommand{\thefootnote}{\fnsymbol{footnote}}

\footnotetext[1]{This work was supported by DFG Collaborative Research Centre 910,
{\it Control of self-organizing nonlinear systems: Theoretical methods and concepts of application}}
\footnotetext[3]{ Institut f\"{u}r Mathematik, MA 4-5, TU Berlin, Stra\ss e des 17. Juni 136, D-10623 Berlin,
Germany; mehrmann@math.tu-berlin.de}
\footnotetext[2]{ Humboldt University of Berlin, Dept. of Mathematics, Rudower Chaussee 25, D-12489 Berlin,
Germany; \{ha,tischendorf\}@math.hu-berlin.de}

\maketitle

\newcommand{\thedate}{\today}

\begin{center}
\thedate
\end{center}

\begin{abstract}
In this article we study the perturbation analysis of initial value problems for linear delay differential-algebraic equations (DDAEs) with time variable coefficients.
First the perturbation index concept for DAEs \cite{HaiLR89} is extended to DDAEs, which followed by the index upper bound theorem for a general linear DDAEs. 
Then we consider the contractivity properties of the solutions and determne sufficient conditions for the asymptitic stability of the zero solution by considering a suitable reformulation of the given system.
In the last part of the article a cass of nuerical methods preserbing the above mentioned stability properties is studied.
\end{abstract}

\begin{keywords} Delay differential-algebraic equation, differential-algebraic equation, delay differential equations, method of steps, derivative array, classification of DDAEs.
\end{keywords}

\begin{AMS}
34A09, 34A12, 65L05, 65H10.
\end{AMS}

\pagestyle{myheadings}
\thispagestyle{plain}
\markboth{P. Ha and V. Mehrmann and C. Tischendorf}{Perturbation and stability analysis of DDAEs}


%
{
\newcommand{\mc}[3]{\multicolumn{#1}{#2}{#3}}

\begin{center}
\begin{tabular}{c|l} \hline
Notation & Meaning \\ \hline           
$\| \cdot \|$ & The usual Euclidean norm in $\C^n$ \\
$\bbI$ & The time interval, i.e. $\bbI= [t_0,t_f)$ \\
$C^m(\bbI)$ & The space of m times continuously differentiable functions on $\bbI$ \\
$\| \cdot \|_{\infty}$ & The sup-norm in $C^0$ defined as $\| f \|_m := \sup \{ \| f(t) \| | t \in \bbI\}$ \\
$\| \cdot \|^t_{\infty}$ & The sup-norm of the restricted function $f_{|[t_0,t]}$, i.e. \\
                         & \quad $\| f \|^t_{\infty} := \sup \{ \| f(s) \|, \ <t_0  \leq s \leq t \}$ \\ 
$\| \cdot \|_m$ & The norm in $C^m(\bbI)$ defined as $\| f \|_m := \sum_{i=0}^m \| f^{(i)} \|_{\infty} $ \\
$\| \cdot \|^t_m$ & The norm in $C^m(\bbI)$ of the restricted function $f_{|[t_0,t]}$, i.e. \\ 
                  & \quad $\| f \|_m := \sum_{i=0}^m \| f^{(i)} \|^t_{\infty} $ \\       
$g^i$  & The restricted function $g^i := g_{|\bbI_i}$, where $\bbI_j=[\eta_{j-1},\eta_{j}]$, for $j \geq 1$.\\
$\De$  & The shift backward operator, i.e. $\De x(t) := x(t-\tau(t))$\\ \hline 
\end{tabular}
\end{center}
}
%

\vskip .5cm

%===========================================================================================================================
%%%%%%%%%%%%%%%%%%%%%%%%%%%%%%%%%%%%%%%%%%%%%%%%%%%%%%%%%%%%%%%%%%%%%%%%%%%%%%%%%%%%%%%%%%%%%%%%%%%%%%%%%%%%%%%%%%%%%%%%%%%%
\section{Preliminaries and notations}\label{pre}
%%%%%%%%%%%%%%%%%%%%%%%%%%%%%%%%%%%%%%%%%%%%%%%%%%%%%%%%%%%%%%%%%%%%%%%%%%%%%%%%%%%%%%%%%%%%%%%%%%%%%%%%%%%%%%%%%%%%%%%%%%%%
%===========================================================================================================================

% In this paper we consider the following systems 

In this paper we study the perturbation analysis of initial value problems for general \emph{linear delay dif\-fer\-en\-ti\-al-al\-ge\-bra\-ic equations (DDAEs) with variable coefficients} 
and a delay function $\tau>0$ of the form
%
\begin{equation}\label{eq1.1}
  E(t) \dot{x}(t) = A(t) x(t) + B(t) x(t-\tau(t)) + f(t), 
\end{equation}
%
in a time interval $\bbI=[t_0,t_f)$, where $\dot{x}$ denotes the time derivative of the vector valued function $x$. As in many applications, usutally the delay function $\tau$ are required to satisfy the following properties, 
see \cite{BelZ03}:
%
\begin{enumerate}
 \item[H1)] $\tau(t)$ is a continuous function.
 \item[H2)] $\tau(t) \geq \tau_0 > 0$ for any $t\geq t_0$.
 \item[H3)] for every $s \geq t_0$ the equation $t-\tau(t)=s$ has a unique solution on $(s,t_f]$.
\end{enumerate}
%
The desired function $x$ maps from $\bbI_{\tau}:=[t_0-\tau_0,t_f)$ to $\complex^n$ and the coefficients are matrix functions $E, \ A, \ B: \bbI \rar \complex^{m,n}$, and $f: \bbI \rar \complex^{m}$. 
%
To achieve uniqueness of solutions of  \eqref{eq1.1} one typically has to prescribe initial functions of the form
%
\begin{equation}\label{eq1.1ic}
\phi: [-\tau_0,0] \rar \complex^{n},\ \mbox{\rm such that}\  x|_{[-\tau_0,0]}=\phi.
\end{equation}
%
% \textbf{
% For simplicity, we assume that  $t_f = \ell \tau$, so that the time interval is $\bbI=[0,\ell \tau)$, for an integer $\ell\in\N$. We also allow for $\ell=\infty$ and then set $\bbI=[0,\infty)$.
% }
%

% Most of the results in this paper can also be extended to multiple delays, but here we only discuss the single delay case. 
% DDAEs of the form (\ref{eq1.1}) arise as linearization of nonlinear DDAEs $F(t,\dot x(t), x(t), x(t-\tau(t))) = 0$
% around a non-stationary nominal solution \cite{Cam95b}, and they describe the local behavior in the neighborhood of the nominal solution. Here, however, we restrict ourselves to the linear variable coefficient case.

\medskip

Two important subclasses of \eqref{eq1.1} that occur in various applications are dif\-fe\-ren\-ti\-al-al\-ge\-bra\-ic equations (DAEs)
with $B\equiv 0$, and delay differential equations (DDEs), where $m=n$ and $E$ is the identity matrix. A typical viewpoint that is often taken in the analysis and numerical solution of DDEs and DDAEs is to introduce an artificial inhomogeneity
$g(t)=B(t)x(t-\tau)+f(t)$ and to consider instead of \eqref{eq1.1} the \emph{associated DAE}
%
\begin{equation}\label{eq1.2}
 E(t)\dot{x}(t) = A(t) x(t) + g(t) \ \mbox{ for all } \ t \in \bbI.
\end{equation}
%
If the associated DAE \eqref{eq1.2} is uniquely solvable for all sufficiently smooth inhomogeneities $g$ and appropriate consistent initial vectors, then the
solution of (\ref{eq1.1}) with initial function (\ref{eq1.1ic}) can be uniquely determined step-by-step by solving a sequence of DAEs on consecutive intervals
$[i\tau_0,(i+1)\tau_0]$. This is the most common approach for systems with delays, often called the \emph{(Bellman) method of steps},  see e.g., \cite{AscP95,BakPT02,BelC63,BelZ03,Cam80,Cam91,GugH07,ShaG06,ZhuP97}.
However, even for DDAE system with constant matrix coefficients, this approach may fail for general, since the dynamic of DDAEs is much richer than the one for DAEs. 

Even in the case of constant delay, i.e. $\tau(t) \equiv \tau$, most of the investigation sofar reformulate the system in a DAE form by introducing a new inhomogeniety function. For example the linear DDAE \eqref{eq1.1} will be reinterpreted as the associated DAE \eqref{eq1.2} 
with the inhomogeniety $g:=B(t)x(t-\tau)+f(t)$. Therein the index concepts for DDAEs are defined to be the corresponding index concepts for DAEs, for example, see e.g. \cite{AscP95,GugH01,ShaG06,CamL09}. 
In a more general situation, the dynamic of the DDAE \eqref{eq1.1} is much richer than the one for the associated DAE \eqref{eq1.2}, for example \eqref{eq1.1} has a unique solution, eventhough \eqref{eq1.2} has infinitely many 
solution. One of these important situations, namely \emph{noncausal}, i.e., the solution at the present time $t$ depends not only on the systems coefficients at past and current time points ($s\leq t$), has been considered in 
\cite{HaMS14,HaM15}. Therein, the index concept for DDAE systems is studied for general noncausal, linear time variable coefficient DDAEs. 
We recall the following result from \cite{HaM15}, in comparison with Theorem 3.2 of \cite{HaM15}.
%
\begin{thm}
Consider the DDAE \eqref{eq1.1} and assume that the following hold
 %
 \begin{enumerate}
  \item[i)] The pair of shift index functions $\ka(t)$ and strangeness index $\mu(t)$ is well-defined for every $t\in \bbI$.
  \item[ii)] The shift index function $\ka$ is a constant on the whole interval $\bbI$.
  \item[iii)] The system \eqref{eq1.1} is not of advanced type.
  \item[iv)] The corresponding initial value problem for the DDAE \eqref{eq1.1} has a unique solution.
 \end{enumerate}
%
Then solution of the DDAE \eqref{eq1.1} is exactly the solution of the so-called \emph{regular, strangeness-free} DDAE
%
\be\label{eq1.3}
  \underbrace{\m{\hE_{1}(t) \\ 0}}_{\hE} \dot{x}(t) \!=\! 
  \underbrace{\m{\hA_{1}(t) \\ \hA_{2}(t) }}_{\hA} x(t) \!+\!
  \underbrace{\m{\hB_{1}(t) \\ \hB_{2}(t) }}_{\hB} x(t-\tau) \!+\! 
  \underbrace{\m{\hf_{1}(t) \\ \hf_{2}(t) }}_{\hf}, \quad \pm{ d \\ a}
\ee
%
where $d$, $a$ are the size of the corresponding block equations and the matrix-valued function $\m{\hE_{1} \\ \hA_{2} }$ is pointwise invertible. Moreover, herein \eqref{eq1.3}, the functions 
$\hf_1$, $\hf_2$ depends on $f^{(i)}(t+j\tau)$, $i=0,\dots,\mu$, $j=0,\dots,\ka$.\\
\end{thm}
%

We note that, under the smoothness assumption $\hE \in C^0(\bbI,\C^{d,n})$, $\hA \in C^0(\bbI,\C^{a,n})$, there exist pointwise orthogonal matrix functions $P\in C^0(\bbI,\C^{n,n})$ and 
$Q\in C^1(\bbI,\C^{n,n})$, see e.g. \cite{DieE99,KunM91}, such that 
%
\be\label{eq1.4}
 P\hE Q = \m{I_d & 0 \\ 0 & 0} , \quad  P \hA Q - P \hE \dot{Q} = \m{A_{11} & 0 \\ 0 & -I_a}.
\ee
%
In fact, the columns of $P$ and $Q$ could be constructed from the ranges and null spaces of $\hE$ and $\hA$ as follows
%
\bens
 P = \m{\range(\hE^H) & \ker(\hE)^H}^T, \quad Q = \m{\range(\hE) & \ker(\hE)}
\eens
%
where the superscripts $H$ (reps. $T$) indicates the conjugate transpose (resp. the transpose) of the corresponding matrix.

Changing the variable $x=Qy := Q \m{y_1 \\ y_2}$ and scaling the whole system \eqref{eq1.3} with $P$ we obtain the following system 
%
\be\label{eq1.5}
  \m{I_d & 0 \\ 0 & 0 } \m{\dot{y}_1(t) \\ \dot{y}_2(t)} \!=\! \m{A_{11} & 0 \\ 0 & -I_a} \m{y_1(t) \\ y_2(t)} \!+\!
  \m{B_{11} & B_{12} \\ B_{21} & B_{22}} \m{y_1(t-\tau) \\ y_2(t-\tau)} \!+\! \m{f_{1} \\ f_{2} }. \quad \pm{ d \\ a}
\ee
%
The computation of these matrix-valued functions is not numerically stable and hence, is still an open problem. We, therefore, will directly consider the regular, strangeness-free DDAE \eqref{eq1.3}. 
%
Since $\rank(\hE)=a$ for all $t\in\hro{I}$, there exists a smooth orthogonal projector $Q$ onto the kernel of $\hE$, see e.g. \cite{DieE99,KunM91}. Let $P=I_n-Q$ which is the orthogonal projection on the cokernel of $\hE^T$. 
Making use of the tractability index concept \cite{LamMT13}, we decouple the system \eqref{eq1.3} as follows.\\ 
%
\begin{theorem}
Consider the DDAE \eqref{eq1.3} with the smooth orthogonal projections $P$ (resp. $Q$) onto the kernel of $\hE$ (resp. $\hE^T$). Then $G:=\hE-\hA Q$ is pointwise invertible. Moreover, the solution $x$ of the corresponding IVP 
for the DDAE \eqref{eq1.3} can be represented in the following form
%
\bens\label{eq1.6}
 x(t) &=& z(t) + v(t), \\
 v(t) &=& Q(t) G^{-1}(t)  \Big( \hA(t)z(t) + \hB(t)x(t-\tau) + \hf(t) \Big)
\eens
%
where $z(t)=P(t)x(t)$ solves the following linear system
%
\bens
   \dot{z}(t) &=& \Big( \dot{P}(t) \!+\! P(t) [I \!+\! \dot{P}(t)] G^{-1}(t) \hA(t) \Big) z(t) \!+\! P(t) [I \!+\! \dot{P}(t)]  G^{-1}(t)  \Big( \hB(t)x(t-\tau) \!+\! \hf(t) \Big), \\
   z(t) &=& P(t) \phi(t) \mbox{  for all  } t \in [t_0-\tau_0,t_0].
\eens
%
\end{theorem}
%

%===========================================================================================================================
%%%%%%%%%%%%%%%%%%%%%%%%%%%%%%%%%%%%%%%%%%%%%%%%%%%%%%%%%%%%%%%%%%%%%%%%%%%%%%%%%%%%%%%%%%%%%%%%%%%%%%%%%%%%%%%%%%%%%%%%%%%%
\section{Perturbation analysis of linear DDAEs}\label{perturbation}
%%%%%%%%%%%%%%%%%%%%%%%%%%%%%%%%%%%%%%%%%%%%%%%%%%%%%%%%%%%%%%%%%%%%%%%%%%%%%%%%%%%%%%%%%%%%%%%%%%%%%%%%%%%%%%%%%%%%%%%%%%%%
%===========================================================================================================================

To our best knowledge, the perturbation theory of DDAEs is almost open, and only several results are already known \cite{AscP95,GugH07}. In order to partially fill in this gap, in this section we firstly study 
the sensitivity of the solution $x(t)$ to the IVP \eqref{eq1.1},\eqref{eq1.1ic} with respect to systems perturbation, which is followed by the discussion of contractivity and robust stability. 
%
Inherited from the perturbation analysis of DDEs and of DAEs, one can perturb not only the system coefficients $E$, $A$, $B$, $f$ (as for DAEs) but also the delay function $\tau(t)$ and the initial function $\phi(t)$ (as for DDEs) 
as well. However, the structual properties of the systems, for example the index concept, will be strongly affected by arbitrary perturbation on the system coefficients. 
The similar situation will occur for the perturbation in the delay function $\tau$, which could lead to stabilization or destablization effect, even for scalar equations. 
These topics go beyond the scope of this article, and therefore, will be left for future researchs. We refer the 
interested readers to \cite{DuLMT13} (resp. \cite{BelZ03}, Chapter 1, \cite{Log98}) for further details in the \emph{''structural perturbation''} analysis of DAEs (reps. perturbation in the delay function).

\begin{rem}
 The robustness of regular, sfree DDAEs with respect to the perturbation only in $\phi$, but not in $\fr{d \phi}{dt}$. This feature distinguishes the two classes of sfree DDAEs and neutral DDEs?
\end{rem}

Now we recall the following result, see \cite{Sor84,Sor06,Zen97}.
\begin{lem}\label{Lem1}
Consider the following ODE
%
\bens
 \dot{x}(t) &=& L(t) y(t) +  \Phi(t), \quad t \in \bbI = [t_0,t_f], \\
 x(t_0) &=& x_0, 
\eens
%
where the forcing term $\Phi \in C^0$. Given an inner product $\langle \cdot, \cdot  \rangle$ and the corresponding norm $\| \cdot \|$. Let $\mu[L](t)$ be the logarithmic norm induced by $\langle \cdot, \cdot  \rangle$. 
Then the following inequalities hold for all $t \geq t_0$
%
\be\label{eq2.-1}
 \|x(t)\| \leq E(t,t_0) \|x_0\| +  \int_{t_0}^{t} E(t,s) \n{\Phi(s)}ds.
\ee
%
where $E(t_2,t_1) := exp \Big(\int_{s=t_1}^{t_2} \mu[L](s) \Big) ds)$.\\ 
%
Moreover, in the case that $\mu[L](t) \not=0$ for all $t\geq t_0$, then 
%
\be\label{eq2.0}
 \| x(t) \|  \leq E(t,t_0) \| x_0 \| + \Big| 1 - E(t,t_0) \Big|  \sup_{t_0 \leq s \leq t} \Big\| \fr{\Phi(s)}{|\mu[L](s)|} \Big\|_{\infty}. 
\ee
%
\end{lem}
\begin{proof}
The idea is combined from the articles \cite{Sor06} and \cite{Zen97}, which is briefly described in the followings.
\begin{enumerate}
 \item[A)] Using the upperright Dini derivative, we obtain the following estimation
 %
 \be\label{eq2.-2}
   \mathrm{D}^+_t \|x(t)\| \leq \mu[L](t) \ \|x(t)\| + \n{\Phi(t)}
 \ee
 %
 \item[B)] Noticing that the function $E(t,t_0) = exp \Big(\int_{s=t_0}^{t} \mu[L](s) \Big) $ has the property
 %
 \be\label{eq2.-5}
     \frac{d}{dt} E(t,t_0) =  \mu[L](t) E(t,t_0).
 \ee
 %
 \item[C)] Consider the scalar function $y(t) := \fr{\|x(t)\|}{E(t,t_0)}$, from \eqref{eq2.-2} we obtain
 %
 \be\label{eq2.-3}
   \mathrm{D}^+_t y(t) \leq \fr{ \n{\Phi(t)} }{E(t,t_0)}.
 \ee
 %
 \item[D)] Integrate the inequality \eqref{eq2.-3} from $t_0$ to $t$ we obtain 
 %
 \bens
  y(t) &\leq& y(t_0) + \int_{t_0}^{t} \fr{ \n{\Phi(s)} }{ E(s,t_0) }ds, \\
  \iff  \quad \|x(t)\| &\leq& E(t,t_0) \|x_0\| +  \int_{t_0}^{t} E(t,s) \n{\Phi(s)}ds,
 \eens
 %
which is nothing else than \eqref{eq2.-1}.\\

\noindent Similar to \eqref{eq2.-5} we have the identity
%
\[
 \frac{d}{ds} E(t,s) =  -\mu[L](s) E(t,s).
\]
%
and therefore \eqref{eq2.-1} gives us
%
\bens
   \|x(t)\| &\leq& E(t,t_0) \|x_0\| +  \int_{t_0}^{t} \Big( \frac{d}{ds}E(t,s) \Big)  \frac{\n{\Phi(s)}}{-\ml(s)}ds, \\
            &\leq& E(t,t_0) \|x_0\| +  \Big| \int_{t_0}^{t} \Big( \frac{d}{ds}E(t,s) \Big) ds \Big| \  \sup_{t_0 \leq s \leq t} \Big\| \fr{\Phi(s)}{|-\mu[L](s)|} \Big\|_{\infty}, 
\eens
%
which yields \eqref{eq2.0} after direct calculation.
\end{enumerate}

\end{proof}

In the following two theorems we study the sensitivity and robust stability of the corresponding IVP for system \eqref{eq1.3}. 

%
\begin{thm}\label{thm1}
Consider the regular, strangeness-free DDAE \eqref{eq1.3}. Moreover, assume that the matrix coefficients satisfy the following properties:
%
\begin{enumerate}
 \item[i)] The matrix-valed functions $E$, $A$, $B$, $f$ are sufficiently smooth so that the matrix functions $P$ and $Q$ in \eqref{eq1.4} exist, and the system \eqref{eq1.5} is well defined.
 \item[ii)] The inverse of the transformation matrix $Q$ is uniformly bounded on $\hro{I}$, i.e. $\|Q^{-1}\|_{\infty} < \infty$. 
\end{enumerate}
%
If $\bbI$ is bounded, then there exists a positive constant $C$ which depends on the systems coefficients of \eqref{eq1.3}, and of length of $\bbI$, so that 
%
\be\label{eq2.1} 
   \| x(t) \|  \leq C \ \Big(  \| \phi \|_{\infty} + \| f \|^t_{\infty} \Big) .
\ee
%
\end{thm}
\begin{proof} Within this proof, for convenience, we skip the argument $(t)$ in all system coefficients and also in the delay function $\tau(t)$. 
By the assumption on $\tau$, we can split the interval $\bbI$ into subintervals by the following points
%
\be
 \eta_0 = t_0 < \eta_1  < \dots < \eta_j < \eta_{j+1} < \dots 
\ee
%
where $\eta_{j+1}$ is the unique solution of the equation $t-\tau(t) = \eta_{j}$. We set $\bbI_0 = [-\tau_0,t_0]$, $\bbI_j=[\eta_{j-1},\eta_{j}]$, for $j \geq 1$. For an arbitrary function $g$, the super script $i$ indicates 
the restricted function on the interval $\bbI_{i}$, i.e., $g^i = g_{|\bbI_i}$. We rewrite the system \eqref{eq1.5} as the coupled system
%
\be
 \bc 
    \dot{y}_1(t) &= A_{11} y_1(t) + \m{B_{11} & B_{12}} \De y(t-\tau) + f_{1}, \\
     y_2(t)      &= \m{B_{21} & B_{22}} \De y(t-\tau) \!+\! f_{2}.
 \ec
\ee
%
Without loss of generality, we assume that $t\in \bbI_j$. Thus we have
%
\[
 \bc 
    \dot{y}^j_1(t) &= A_{11} y^j_1(t) + \m{B_{11} & B_{12}} y^{j-1}(t-\tau) + f^j_{1}, \\
     y^j_2(t)      &= \m{B_{21} & B_{22}} y^{j-1}(t-\tau) \!+\! f^j_{2}.
 \ec
\]
%
Set $\Phi^j = \m{B_{11} & B_{12}} y^{j-1}(t-\tau) + f_{1}$, $t\in \bbI_j$. By applying Lemma \ref{Lem1} we see that there exist constant $\a_1$, $\a_2 \in \R_+$ so that the following estimation holds
%
\bsq
%
\be\label{eq2.10a}
  \n{y^j_1(t)} \leq \a_1 \n{y^j_1(\eta_{j-1})} + \a_2 \n{\Phi^j}^t_{\infty}.
\ee
%
On the other hand we see that 
%
\be\label{eq2.10b}
  \n{y^j_2(t)} \leq \n{ \m{B_{21} & B_{22}}  }_{\infty} \ \n{y^{j-1}}_{\infty} + \n{f^j_{2}}^t_{\infty}.
\ee
%
\esq
%
Combining \eqref{eq2.10a} and \eqref{eq2.10b} and notice that 
%
\bens
  \n{y^j_1(\eta_{j-1})} \leq \n{y^{j-1}_1}_{\infty}, \quad \n{\Phi^j}^t_{\infty} \leq \n{ \m{B_{11} & B_{12}}}_{\infty} \ \n{y^{j-1}}_{\infty} + \n{f^j_1}^t_{\infty},
\eens
%
we see that there exist $\b \in \R_+$ so that 
%
\be\label{eq2.11} 
   \n{y^j(t)} \leq \b \n{y^{j-1}}_{\infty} + \b \n{f^j}^t_{\infty}.
\ee
%
Due to the arbitrarity of $t \in \bbI_j$ this also leads to
%
\bes
 \n{y^j}_{\infty} \leq \b \n{y^{j-1}}_{\infty} + \b \n{f^j}_{\infty}.
\ees
%
It is clear that the constant $\b$ depends on $j$. However, if the interval $\bbI$ is bounded, one may assume that this constant is uniform for every $j$. Thus, simple induction gives 
%
\bes
    \n{y^{j-1}}_{\infty} \leq \b^{j-1} \n{y^{0}}_{\infty} + \sum_{i=1}^{j-2} \b^i \n{f^{j+1-i}}_{\infty},
\ees
%
and finally \eqref{eq2.11} leads to $\n{y^j(t)} \leq \b^{j-1} \n{y^{0}}_{\infty} + \sum_{i=1}^{j-2} \b^i \n{f^{j+1-i}}_{\infty} + \b \n{f^j}^t_{\infty}.$\\
%
% \bes
%   \n{y^j(t)} \leq \b \n{y^{j-1}}_{\infty} + \b \n{f^j}^t_{\infty} \leq \b^{j-1} \n{y^{0}}_{\infty} + \sum_{i=1}^{j-2} \b^i \n{f^{j+1-i}}_{\infty} + \b \n{f^j}^t_{\infty}.
% \ees
%
Let $C:= \underset{j}{\max} \{ \sum_{i=1}^{j-2}\b^i + \b,\ \b^{j-1} \}$ we then have \eqref{eq2.1}.
\end{proof}




\begin{rem} 
We notice that, the estimation \eqref{eq2.1} requires the infinity-norm of the function $f$ on the whole interval $[0,t]$, instead of only at the point $t$ as for DAEs. This feature is typical for time delay systems, since the 
inhomogeneity in the past can also have strong impact on the present solution.
\end{rem}

From Theorem \ref{thm1} we can easily see that the solution $x(t)$ to the corresponding IVP of the DDAE \eqref{eq1.4} is robust under perturbation of the initial function $\phi$ and of the inhomogeneity $f$. 
This means that the corresponding IVP of the DDAE \eqref{eq1.4} has perturbation index index at most $1$ along an arbitrary solution, in the sense of the following definition, which is 
directly extended from the concept of perturbation index for DAEs \cite{HaiLR89}.
%Now we want extend this result for arbitrarily high index DDAE systems. The following definition directly extend the concept of perturbation index for DAEs to the one for DDAEs.

\begin{defi} The IVP 
%
\bens
  F(t,x(t),\dot{x}(t),x(t-\tau(t))) &=& 0,  \qquad t \in \bbI, \\
                        x|_{[-\tau_0,0]} &=& \phi, 
\eens
%
has \emph{perturbation index $\nu \geq 1$} along the solution $\bar{x}$ if $\nu$ is the smallest positive integer such that for the perturbed problem 
%
\bens
  F(t,x(t),\dot{x}(t),x(t-\tau(t))) &=& \de (t), \qquad t \in \bbI, \\
                        x|_{[-\tau_0,0]} &=& \phi + \de\phi, 
\eens
%
the defect $\de x(t) := x(t)-\bar{x}$ satisfies the following inequality
%
\be
   \| \de x(t) \|  \leq C \ \Big(  \| \de \phi \|_{\nu-1} + \| \de \|^t_{\nu-1} \Big) .
\ee
%
for sufficiently small $\de(t)$ in the $\| \cdot \|_{\nu-1}$ norm. Here $C$ is a positive constant which depends on $F$, $\phi$, $\bar{x}$, and length of the time interval $\bbI$.\\
\noindent In the case that there exist the estimation 
%
\be
   \| \de x(t) \|  \leq C \ \Big(  \int_{t_0-\tau_0}^{t_0} \| \de \phi(s)\|ds  +  \int_0^t \| \de(s)\|ds \Big) .
\ee
%
\end{defi}




%===========================================================================================================================
%%%%%%%%%%%%%%%%%%%%%%%%%%%%%%%%%%%%%%%%%%%%%%%%%%%%%%%%%%%%%%%%%%%%%%%%%%%%%%%%%%%%%%%%%%%%%%%%%%%%%%%%%%%%%%%%%%%%%%%%%%%%
\section{Contractivity and stability properties of linear DDAEs}\label{contractivity}
%%%%%%%%%%%%%%%%%%%%%%%%%%%%%%%%%%%%%%%%%%%%%%%%%%%%%%%%%%%%%%%%%%%%%%%%%%%%%%%%%%%%%%%%%%%%%%%%%%%%%%%%%%%%%%%%%%%%%%%%%%%%
%===========================================================================================================================




%===========================================================================================================================
%%%%%%%%%%%%%%%%%%%%%%%%%%%%%%%%%%%%%%%%%%%%%%%%%%%%%%%%%%%%%%%%%%%%%%%%%%%%%%%%%%%%%%%%%%%%%%%%%%%%%%%%%%%%%%%%%%%%%%%%%%%%
\section{Conclusion and outlooks}\label{conclusion}
%%%%%%%%%%%%%%%%%%%%%%%%%%%%%%%%%%%%%%%%%%%%%%%%%%%%%%%%%%%%%%%%%%%%%%%%%%%%%%%%%%%%%%%%%%%%%%%%%%%%%%%%%%%%%%%%%%%%%%%%%%%%
%===========================================================================================================================



\bibliographystyle{siam}
\bibliography{Phi_Jan_20_15}

\end{document}

