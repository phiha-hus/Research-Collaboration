%\documentclass{amsart}
\documentclass[12pt,a4paper,reqno]{amsart}


\textwidth=16.5cm
\textheight=20cm%21.5cm
\oddsidemargin=0cm
\evensidemargin=0cm


\newtheorem{theorem}{Theorem}[section]
\newtheorem{lemma}[theorem]{Lemma}
\theoremstyle{definition}
\newtheorem{definition}[theorem]{Definition}
\newtheorem{example}[theorem]{Example}
\newtheorem{xca}[theorem]{Exercise}
\newtheorem{corollary}[theorem]{Corollary}
\newtheorem{proposition}[theorem]{Proposition}
\theoremstyle{remark}
\newtheorem{remark}[theorem]{Remark}
\renewcommand{\labelenumi}{\roman{enumi})}
\numberwithin{equation}{section}

\font\name=cmr8
\begin{document}
\def\cal{\mathcal}

%\thispagestyle{empty}

\medskip
\title[\rm Exponential Dichotomy]{Exponential Dichotomy of Difference Equations \\
and Application to Evolution Equations on the Half-Line}
\author[\sc N.T. Huy]{Nguyen Thieu Huy*}\thanks{* The work of the first author
was supported by the National Basic Research Program KT137.}
\address{Nguyen Thieu Huy, Department of Applied Mathematics,
Ha Noi University of Technology,
Khoa Toan Ung dung - Dai hoc Bach khoa Ha Noi,
Dai Co Viet St., Hanoi, Vietnam}
\email{thieuhuy@yahoo.com}
\author[\sc N.V. Minh]{Nguyen Van Minh}
\address{Nguyen Van Minh, Department of Mathematics,
University of Hanoi,
Khoa Toan, Dai Hoc Khoa Hoc Tu Nhien,
334 Nguyen Trai St., Hanoi, Vietnam}
\email{nvminh@mathhnu.netnam.vn}

\subjclass{39A11, 34G10}
%\date{October 30, 1999}
%\dedicatory{}




\begin{abstract} For a sequence of bounded linear operator 
$\{A_n\}_{n=0}^\infty$ on a Banach space X we investigate the 
characterization of exponential dichotomy of the difference equations 
$v_{n+1}=A_nv_n$. We  characterize the exponential dichotomy of difference equations 
in terms of the existence of  
solutions to the equations $v_{n+1}=A_nv_n+f_n$ in $l_\infty$ space. 
Then we apply the results to study the exponential dichotomy of evolution families
generated by evolution equations. 
\end{abstract}
\keywords{Difference equations, discrete evolution family, evolution family, exponential stability, 
exponential dichotomy}

\maketitle


%\baselineskip= .6cm

%%%%%%%%%%%%%%%%%%%%%%%%%%%%%%%  Section 1
%%%%%%%%%%%%%%%%%%%%%%%%%%%%%%%%%%


 \section{INTRODUCTION AND PRELIMINARIES} \label{section 1}

In this paper we are concerned with  differrence equations of the form
\begin{equation}\label{1.1}
x_{n+1}=A_nx_n, \ n\in {\bf N}
\end{equation}
and
\begin{equation}\label{1.2}
x_{n+1}=A_nx_n+f_n, \ n\in {\bf N},
\end{equation}
where $A_n, n=0,1,2, ...,$ is a sequence of bounded linear operators on a given Banach space $X$,
$x_n, f_n \in X$.

One of the central interests in the asymptotic behavior of solutions to Eq.
({\ref{1.1}}) is to find conditions
for solutions of  Eq. ({\ref{1.1}}) to be stable, unstable, and especially to have an exponential dichotomy
(see e.g. \cite{Li}, \cite{cofsch1}, \cite{Hen}, \cite{Slj}, \cite{AulMin2} 
and the references therein for more details on the history of
this problem). In the infinite dimensional case, a sufficient condition for
Eq. ({\ref{1.1}})
to have an exponential dichotomy is a {\it a priori} condition that 
the stable space is complemented (see e.g. \cite{cofsch1}). 
In our recent paper (see \cite{MRS}) in the case evolution equations
we have replaced this condition 
by a rather Perron-styled one. As a result, we have obtained a necessary
and sufficient condition for an evolution equation to have an exponential dichotomy.
As is known, there is an analogy between difference equations and differential equations.
The central purpose of this paper is to provide for linear difference equations
the analogues of the most central
results for linear evolution equations. Moreover, we will show that using the obtained results
one can find sufficient conditions for linear evolution equations to have an exponential
dichotomy. 

To describe more detailedly our construction we will use the following notations:
In this paper $X$ denotes a given complex Banach space. As usual, we denote by {\bf N},
{\bf R}, ${\bf R}_+$ and {\bf C} the set of natural, real, nonnegative real and complex numbers,
respectively.
Throughout of this paper we shall consider the following sequence spaces:
\begin{eqnarray*}
l_\infty({\bf N},X)&:=&\{v=\{v_n\}_{n\in {\bf N}}:v_n\in X:\sup_{n\in {\bf N}}v_n<\infty\}:=l_\infty \\
l^0_\infty({\bf N},X)&:=&\{v=\{v_n\}:v\in l_\infty;\; v_0=0\}:=l_\infty^0\\
l_\infty([n_0,\infty),X)&:=&\{v=\{v_n\}:v\in l_\infty;\; 0<n_0\le n\in {\bf N}\}.
\end{eqnarray*}
Let $\{A_n\}_{n\in {\bf N}}$ be a sequence of bounded linear operators from $X$ to $X$ 
which is uniformly bounded. That means that there exists $M >0$ such that $\|A_nx\|\le M\|x\|$ 
for all $n\in {\bf N}$ and $x\in X$.
Next we define a discrete evolution family ${\cal U} = (U_{n,m})_{n\ge m\ge 0}$ associated
with the sequence $\{A_n\}_{n\in {\bf N}}$ as follows:
\begin{eqnarray*}
U_{m,m}&=&Id \ \ (\hbox{ the identity operator in }X)\\
U_{n,m}&=&A_{n-1}A_{n-2}...A_m\hbox{ for }n>m.
\end{eqnarray*}
The uniform boundedness of $\{A_n\}$ yields the 
exponential boundedness of the evolution family $(U_{n,m})_{n\ge m\ge 0}$. That is, there exist positive 
constants $K, \alpha $ such that $\|U_{n,m}x\|\le Ke^{\alpha(n-m)}\|x\|;\; x\in X; n\ge m\ge 0$.

\begin{definition}\label{def 1.1}
The equation ({\ref{1.1}}) is said to have an {\it exponential dichotomy} if there exist a sequence of projection $(P_n)_{n\in {\bf N}}$ on $X$ and positive constants $N, \nu $ such that:
\begin{enumerate}
\item[(1)] $A_nP_n=P_{n+1}A_n.$
\item[(2)] $A_n:\ker P_n\rightarrow \ker P_{n+1}$ is an isomorphism and we denote its inverse by $A^{-1}_{\mid n}$.
\item[(3)] $\|U_{n,m}x\|\le Ne^{-\nu (n-m)}\|x\|; n\ge m\ge 0 ; x\in P_mX$
\item[(4)] Denote by $U_{\mid m,n}=A^{-1}_{\mid m}.A^{-1}_{\mid m+1}..A^{-1}_{\mid n-1};\; n> m$ and $U_{\mid m,m}=Id$ then 
$$\|U_{\mid m,n}x\|\le Ne^{-\nu(n-m)}\|x\|;\; n\ge m\ge 0;\; x\in \ker P_n.$$
\end{enumerate}
\end{definition}


\medskip

We define an operator $T: l_\infty\rightarrow l_\infty$ as follows:
If $u=\{u_n\},\, f=\{f_n\}\in l_\infty$ satisfy the equation ({\ref{1.2}}) set:
\begin{eqnarray*}
Tu:=f.
\end{eqnarray*}
For $u=\{u_n\}\in l_\infty$, take $f=\{f_n\}$ where 
$f_n= u_{n+1}-A_nu_n$ we have $\|f_n\|\le (1+ M)\|u\|_{l_\infty}$, hence $f\in l_\infty$ and $Tu=f$. That means $D(T)=l_\infty$. 
It is easy to derive  that operator $T$ is a well-defined, bounded linear operator. 
We denote the restriction of $T$ on $l_\infty^0$ by $T_0$. 
From the definition of $T$ the following are obvious:
\begin{remark}\label{rem 1.1}
\begin{enumerate}
\item $\ker T=\{u=\{u_n\}\in l_\infty: u_n=U_{n,0}u_0\},\; n\in {\bf N}$
\item It is easy to verify that $T_0$ is injective. 
Indeed, let $u=\{u_n\},\ v=\{v_n\} \in l_\infty^0$ and $T_0u=T_0v$. Then 
we have $u_0=v_0=0, u_1=(T_0v)_0=v_1, u_2=A_1u_1+(T_0u)_1=A_1v_1+(T_0v)_1=v_2,.., u_{n+1}
=A_nu_n+(T_0u)_n=A_nv_n+(T_0v)_n=v_{n+1}$, for all $n\in {\bf N}$. Hence, $u=v$.
\item $D(T_0)=l^0_\infty.$ Indeed, For $u=\{u_n\}\in l^0_\infty$, 
take $f=\{f_n\}$ where 
$f_n= u_{n+1}-A_nu_n$ we have $\|f_n\|\le (1+ M)\|u\|_{l_\infty}$, 
hence $f\in l_\infty$ and $T_0u=f$. That means $D(T_0)=l^0_\infty$. 
\end{enumerate}
\end{remark}

Recall that for an operator $B$ on a Banach space $Y$ the approximate point
spectrum $A\sigma (B)$ of $B$ is the set of all complex numbers $\lambda$ such that
 for every $\epsilon>0$ there exists $y \in D(B)$ with $\|y\|=1$ and
 $\|(\lambda-B)y\|\le \epsilon$. The following lemmas will be needed in the sequel:

\smallskip
\begin{lemma}\label{lem 1.1}
Let $\{\chi_n\}_{ n_1>n\ge n_0}$ be positive real numbers
and let $c>1$ and $K,\alpha > 0$ be constants such that
$\chi_n\le Ke^{\alpha (n-n_0)}$ and  $\sum_{k=n_0}^n\chi_n\chi_k^{-1}\le c$ with  
$ n_0\le n< n_1 $. Then there exist $N$, $\nu$ dependent only on 
$K, c, \alpha$ such that $\chi_n\le Ne^{-\nu(n-n_0)}$ for $n_0\le n<n_1$.
\end{lemma}
\begin{proof} 
Put $S_n=\sum_{k=n_0}^n\frac{1}{\chi_k}$. From $\chi_n.S_n\le c$ we have
$$\frac{-1}{\chi_nS_n}\le -c^{-1}.$$
Hence,
$$
S_{n-1}=S_n-\chi_n^{-1}=S_n(1-\frac{1}{\chi_nS_n})\le S_n(1-c^{-1}).
$$
Therefore ${1}/{S_n}\le {(1-c^{-1})}/{S_{n-1}}$. Thus, 
$$\chi_n\le \frac{c}{S_n}\le c\frac{(1-c^{-1})}{S_{n-1}}\le ...\le
 c\frac{(1-c^{-1})^{n-n_0}}{S_{n_0}}=c{(1-c^{-1})^{n-n_0}}{\chi_{n_0}}$$
$$\le Kc\left(\frac{c-1}{c}\right)^{n-n_0}$$
By choosing $N=Kc;\; \nu = \ln \frac{c}{c-1}$ we complete the proof.\end{proof}

\smallskip
\begin{lemma}\label{lem 1.2}
Let $\{\chi_n\}_{n\in{\bf N}}$ be a sequence of positive real numbers.
Assume that there are constants $c>1$ and $K,\alpha \ge 0$  such that
$\chi_n\le Ke^{\alpha (n-m)}\chi_m$  and  
$$\sum_{k=m}^n\chi_m\chi_k^{-1}\le c, \ 
\forall \ \ n\ge m\ge 0.$$  
Then there exist $N$, $\nu$ dependent only on 
$K, c, \alpha$ such that $\chi_n\ge Ne^{\nu(n-m)}\chi_m$ for $n\ge m\ge 0$.
\end{lemma}
\begin{proof} 
Put $S_m=\sum_{k=m}^n\frac{1}{\chi_k}$. From $\chi_m.S_m\le c$ we have
$$\frac{-1}{\chi_nS_n}\le -c^{-1}
.$$
Hence,
$$
S_{n}=S_{n-1}-\chi_n^{-1}=S_{n-1}(1-\frac{1}{\chi_{n-1}S_{n-1}})\le S_{n-1}(1-c^{-1}).
$$
Therefore ${1}/{S_{n-1}}\le {1-c^{-1}}/{S_{n}}$. Thus, 
$$\chi_m\le \frac{c}{S_m}\le c\frac{(1-c^{-1})}{S_{m+1}}\le ...\le
     c\frac{(1-c^{-1})^{n-m}}{S_{n}}=c{(1-c^{-1})^{n-m}}{\chi_{n}}.$$

To finish the proof we can choose $N=\frac{1}{c};\; \nu = \ln \frac{c}{c-1}$. 
\end{proof}

\medskip

%%%%%%%%%%%%%%%% Section2
%%%%%%%%%%%%%%%%%%%%%%%%%%%%%%%%%%%%%%%

\section{EXPONENTIAL STABILITY OF DISCRETE BOUNDED ORBITS}\label{section 2}
In this section we will give a sufficient condition for stability of  bounded
orbits of a discrete evolution family $\cal U$. The obtained results will be 
used in the next section to characterize the exponential dichotomy of the
equation ({\ref{1.1}}).

\smallskip
\begin{theorem}\label{the 2.1}
Let the operator $T_0$ defined as above satisfy the condition
$0\notin A\sigma (T_0) $. Then
every discrete bounded orbit of ${\cal U}$ is exponentially
stable. Precisely, if
$$\sup_{n_0\le n\in {\bf N}}\| U_{n,n_0}x \|  < \infty,$$
 with x $\in $ X and $n_0 > 0$, then there exist  positive constants 
$N, \nu $  independent of x and $n_0$ such that:
$$\|U_{n,n_0}x\| \le Ne^{-\nu (n-s)}\|U_{s,n_0}x\|, n\ge s\ge n_0.$$
\end{theorem}
\begin{proof}
Let us start by proving that:
$$\|U_{n,n_0}x\| \le Ne^{-\nu (n-n_0)}\|x\|, \ \forall \ n\ge n_0.$$
Without loss of generality we may assume that $\|x\|$ = 1. Since
 $0\notin A\sigma (T_0)$ there exists a constant $\delta >0$ such that
$\|T_0v\| \ge\delta \|v\| $,   for  $v\in l^0_\infty$. Replacing $\delta$ by a smaller one if necessary we can assume that $\delta<1$.
Let $u_n = U_{n,n_0}x$ for $n\ge n_0$; $n_1:=\sup\{n\ge n_0: U_{n,n_0}x\not= 0\}$.  
The exponential boundedness of ${\cal U}$ yields
$$ \|u_n\| \le Ke^{\alpha (n-n_0)}; n\ge n_0,$$
where $K, \alpha$ are positive constants.
For any natural number $n_2< \infty $ such that $n_0\le n_2\le n_1$ take
\begin{eqnarray*}
v&=&\{v_n\} \hbox{ with } v_n=\begin{cases}
0&\hbox{ for }\ 0\le n<n_0\\
u_n\sum_{k=n_0}^{n}\frac{1}{\|u_k\|} &\hbox{
for }\ n_0\le n\le n_2,\\
u_n\sum_{k=n_0}^{n_2}\frac{1}{\|u_k\|} &\hbox{ for } \ n>n_2\end{cases}\\
f&=&\{f_n\}\hbox{ with } f_n=\begin{cases}0, &\hbox{ for }\ 0\le n<n_0-1\\
\frac{u_{n+1}}{\|u_{n+1}\|}&\hbox{ for }\ n_0-1\le n< n_2\\
0 &\hbox{ for }\ n\ge n_2. \end{cases}
\end{eqnarray*}
Then $v_{n+1}=A_nv_n+f_n$ and $v\in l_\infty^0, f\in l_\infty$.
It follows that $T_0v=f$ and $\|f\|\ge \delta\|v\|$. That means
$$\delta\sup_n\|u_n\|\sum_{k=n_0}^{n}\frac{1}{\|u_k\|}\le \|f\|_{l_\infty}=1,\hbox{ or }
\|u_n\|\sum_{k=n_0}^{n}\frac{1}{\|u_k\|}\le \frac{1}{\delta}.$$
Lemma {\ref{lem 1.1}} yields the exitences of $N,\nu >0$ such that $\|u_n\|\le Ne^{-\nu(n-n_0)}$

Now we fix $s\ge n_0$, set $y:=U_{s,n_0}x$. Then $\sup_{n\ge s} \|U_{n,s}y\| < \infty $,
and
$$
\|U_{n,n_0}x\|=\|U_{n,s}y\|\le Ne^{-\nu (n-s)}\|y\|=Ne^{-\nu(n-s)}\|U_{s,n_0}x\|, n\ge s.
$$
\end{proof}
From this theorem we obtain the following corollary:
\begin{corollary}\label{cor 2.1}
Under the conditions of  Theorem {\ref{the 2.1}} we have 
\begin{eqnarray*}
X_0(n_0)&:=&\{x\in X: \sup_{n\ge n_0}\|U_{n,n_0}x\|<\infty\}\cr
        &= &\{x\in X: \|U_{n,n_0}x\|\le Ne^{-\nu (n-n_0)}\|x\|;\;n\ge n_0\ge 0\}.
\end{eqnarray*}
for certain positive constants
$N,\;\nu $, is a closed linear subspace of X.
\end{corollary}

\medskip
%%%%%%%%%%%%%%%%%%%%%%%%           Section3
%%%%%%%%%%%%%%%%%%%%%%%%%%%%%%%%%

\section{EXPONENTIAL DICHOTOMY}\label{section 3}

We will characterize the exponential dichotomy of the equation ({\ref{1.1}})  by
using the operators $T_0, \ T$. In particular, applying Corollary {\ref{cor
2.1}} we will get necessary
and sufficient conditions for exponential dichotomy in Hilbert spaces and finite dimensional spaces.
\begin{lemma}\label{lem 3.1}
Assume that the equation ({\ref{1.1}}) has an exponential dichotomy 
with corresponding family of projections $P_n, n\ge 0$ and constants $N>0, \nu>0$,
then 
$M:=\sup_{n\ge 0}\|P_n\|<\infty $.
\end{lemma}
\begin{proof} Fix $n_0>0$, and set $P^0:=P_{n_0}; P^1:=Id-P_{n_0}$,
$X_k=P^kX, k=0,1$. Set $\gamma_0:=\inf\{\|x^0+x^1\|: x^k\in X_k,\ \|x^0\|=\|x^1\|=1\}$.
 If $x\in X $ and $P^kx\not= 0$, then
$$\gamma_{n_0}\le \|{\frac{P^0x}{\|P^0x\|}}+{\frac{P^1x}{\|P^1x\|}}\|\le
{\frac{1}{\|P^0x\|}}\|P^0x+{\frac{\|P^0x\|}{\|P^1x\|}} P^1x\|$$
$$
\le  {\frac{1}{\|P^0x\|}} \|x+{\frac{\|P^0x\|-\|P^1x\|}{\|P^1x\|}} P^1x\|\le 
{\frac{2\|x\|}{\|P^0(x)\|}}.
$$
Hence, $\|P^0\|< {2/\gamma_{n_0}}$. It remains to show that there is constant $c>0$ (independent of $n_0$) 
such that $\gamma_{n_0}\ge c$. For this fix $x^k\in X_k,\ k=0.1.$ with $\|x^k\|=1$. By the exponential boundedness 
of $\cal U$ we have $\|U_{n,n_0}(x^0+x^1)\|\le Ke^{\alpha(n-n_0)}\|x^0+x^1\|$ for $ n\ge n_0$ and constants $K,\ \alpha\ge 0$. Thus,
$$\|x^0+x^1\|\ge K^{-1}e^{-\alpha(n-n_0)}\|U_{n,n_0}x^0+U_{n,n_0}x^1\|$$
$$\ge K^{-1}e^{-\alpha(n-n_0)}(N^{-1}e^{\nu(n-n_0)}-Ne^{-\nu(n-n_0)})=:c_{n-n_0},$$
and hence $\gamma_{n_0}\ge c_{n-n_0}$. Obviously $c_m>0$ for $m$ sufficiently large. Thus
$0< c_m\le \gamma_{n_0}.$
\end{proof}

\bigskip
Now we come to our first main result. It characterizes the exponential dichotomy of the 
equation ({\ref{1.1}}) by properties of the operator $T$.

\begin{theorem}\label{the 3.1}
Let $\{A_n\}_{n\in {\bf N}}$ be a family of bounded linear and uniformly bounded operators on the Banach space X. Then the following assertions are equivalent:
\begin{enumerate}
\item The equation ({\ref{1.1}}) has an exponential dichotomy
\item $T$ is surjective and  $X_0(0)$ is complemented in X.
\end{enumerate}
\end{theorem}
\begin{proof}
 (i)$\Rightarrow$(ii): Let $(P_n)_{n\ge 0}$ be the family of projections
determined by the exponential dichotomy. Then
$X_0(0)=P_0X$, and hence $X_0(0)$ is complemented. If $f\in l_\infty$ define
$v=\{v_n\}_{ n\in{\bf N}}$  by
\begin{equation}\label{3.1}
v_n=\begin{cases}\sum_{k=1}^n U_{n,k}P_{k}f_{k-1}-\sum_{k=n+1}^\infty 
U_{\mid n,k}(Id-P_{k})f_{k-1}&\hbox{ for }n\ge 1\cr 
-\sum_{k=1}^\infty U_{\mid 0,k}(Id-P_{k})f_{k-1}&\hbox{ for }n=0\cr
\end{cases}
%Yours Dear
%v_n=\sum_{k=0}^n U_{n,k}P_kf_k-\sum_{k=n+1}^\infty U_{\mid n,k}(Id-P_k)f_k .
\end{equation}
then $v_{n+1}=A_nv_n+f_n$ and $v\in l_\infty$.
By the definition of $T$ we have
$Tv=f$. 
Therefore $T : l_\infty\rightarrow l_\infty$ is surjective.

(ii)$\Rightarrow$(i):

\medskip
{\bf A)} Let  $Z\subseteq X$ be a complement of $X_0(0)$ in $X$
i.e.:$X=X_0(0)\oplus Z$. Set $X_1(n)=U_{n,0}Z$. Then
\begin{equation}\label{3.2}
U_{n,s}X_0(s)\subseteq X_0(n),\;\; 
U_{n,s}X_1(s) = X_1(n),\;n\ge s\ge 0.
\end{equation}

\medskip
{\bf B)} There are constants $N,\ \nu >0$ such that
\begin{equation}\label{3.3}
\|U_{n,0}x\| \ge Ne^{\nu (n-s)}\|U_{s,0}x\|\;\hbox{for}\; 
x \in X_1(0) \;, n\ge s\ge 0. 
\end{equation}
In fact, let $Y:=\{(v_n)_{n\in {\bf N}} \in l_\infty: v_0\in X_1(0)\}$ 
endowed with $l_\infty$-norm.
Then $Y$ is a closed subspace of the Banach space $l_\infty$
and hence $Y$ is complete. By Remark {\ref{rem 1.1}} we have ker$T:=\{v\in l_\infty:v_n=U_{n,0}x$
for some $x\in X_0(0)\}$. Since $X=X_0(0)\oplus X_1(0)$ and 
$T$ is surjective  we obtain
$$T:Y\rightarrow l_\infty$$
 is bijective and hence an isomorphism. Thus there is a constant
$\delta >0$ such that  
\begin{equation}\label{3.4}
\|Tv\|_{l_\infty} \ge  \delta\|v\|_{l_\infty},\hbox{ for }
 v \in Y.
\end{equation}

\smallskip
Let $0 \not= x\in X_1(0)$, set $u_n:=U_{n,0}x, n\ge 0$. By Remark {\ref{rem
1.1}} we
have $u_n \not = 0$ for all $n\ge 0$.
For a natural large number $\tau >0 $   take $v=\{v_n\},  f=\{f_n\}$, where
\begin{eqnarray*}
v_n&=&\begin{cases}
u_n\sum_{k=n+1}^{\tau}\frac{1}{\|u_k\|}&\hbox{
for }\ 0\le n< \tau\\
0 &\hbox{ for }\ n\ge \tau\end{cases}\\
f_n&=&\begin{cases}
-\frac{u_{n+1}}{\|u_{n+1}\|}&\hbox{
for }\ 0\le n<\tau\\
0,&\hbox{ for }\ n\ge\tau \end{cases}
\end{eqnarray*}
Then $v \in Y$, and $f \in l_\infty$ which
satisfy the equation $ v_{n+1}=A_nv_n+f_n$. It follows that
$$Tv=f\Rightarrow \|f\|_{l_\infty} \ge \delta\|v\|_{l_\infty}.$$
Hence, $$1\ge \delta \|u_n\|\sum_{k=n+1}^{\tau}\frac{1}{\|u_k\|}\Rightarrow
\|u_n\|\sum_{k=n}^{\tau}\frac{1}{\|u_k\|}\le \frac{1}{\delta}+1.$$
Therefore the exponential boundedness of ${\cal U}$ and
Lemma {\ref{lem 1.2}} imply that there are  constants $ N , \nu > 0$ independent of $x$ such that
$$\|u_n\|\ge Ne^{\nu (n-s)}\|u_s\|;\; n\ge s\ge 0.$$

\medskip
{\bf C)} $X=X_0(n)\oplus X_1(n)$,\    $n\in {\bf N}$.

Let $Y\subset l_\infty$ be as in {\bf B)}. Then by Remark {\ref{rem 1.1}} $l^0_\infty\subset Y$. 
From this and
({\ref{3.4}})  we have $\|T_0v\|_{l_\infty}\ge \nu\|v\|_{l_\infty},$ for $v\in l^0_\infty$. Thus, 
$0\notin A\sigma(T_0)$ and 
Corollary {\ref{cor 2.1}} imply that $X_0(n)$ is closed.
From ({\ref{3.2}}) , ({\ref{3.3}}) and the closedness of $X_1(0)$ we can easily derive that $X_1(n)$ is closed and $X_1(n)\cap X_0(n)=\{0\}$
for $n\ge 0$.

\smallskip
Finally, fix $n_0>0$, and $x\in X$. For large natural number $n_1$ set
$$
v=\{v_n\}\hbox{ with } 
v_n=\begin{cases} 
(n-n_0+1)U_{n,n_0}x, &\hbox{ for }\ n_0\le n\le n_1\\
0, &\hbox{  for }\ n>n_1. \end{cases}
$$
$$
f=\{f_n\}\hbox{ with }
 f_n=\begin{cases} U_{(n+1),n_0}x,&\hbox{ for }\ n_0\le n<n_1\\
-(n_1-n_0+1)U_{(n+1),n_0}x &\hbox{ for }  \ n=n_1\\
 0 , &\hbox{ for }\ n>n_1.\end{cases}
$$
Then $v_n, f_n$ solve the equation ({\ref{1.2}}) with $n \ge n_0> 0$ and $v\in l_\infty([n_0,\infty),X)$.
 Set $f_n=0$ for $0\le n<n_0$.
Then $f\in l_\infty({\bf N},X)$ by assumption there exists $w \in l_\infty$ such that
$Tw=f$. By the definition of $T$,  $w_n$ is a solution of the equation
({\ref{1.2}}). 
In particular,
$\{w_n\}_{n_0\le n< \infty}$ satisfies ({\ref{1.2}}) as well. Thus, 
$$v_n-w_n=U_{n,n_0}(v_{n_0}-w_{n_0})=U_{n,n_0}(x-w_{n_0}),\;    n\ge n_0.$$
Since for $n_0\le n<\infty $ we have $\{v_n-w_n\}_{n\ge n_0}\in l_\infty([n_0,\infty),X)$.
This implies $x-w_{n_0} \in X_0(n_0)$.
On the other hand, since $w_0=w^0+w^1$ with $w^k\in X_k(0)$, 
$w_{n_0}=U_{n_0,0}w^0+U_{n_0,0}w^1$ and by ({\ref{3.2}}) we have  $U_{n,n_0}w^k\in X_k(n_0)$,
 $k=0,1$. Hence $x=x-w_{n_0}+w_{n_0} \in X_0(n_0)+X_1(n_0)$. This proves {\bf C)}.

\medskip
{\bf D)} Let $P_n$ be the projections from X onto $X_0(n)$ with kernel $X_1(n)\,$, $ n\ge 0$.
Then ({\ref{3.2}}) implies that $P_{n+1}U_{(n+1)n}=U_{(n+1)n}P_n$, or $A_nP_n=P_{n+1}A_n$ for 
 $n\ge  0$. From ({\ref{3.2}}), ({\ref{3.3}}) and $A_n=U_{(n+1)n}$ we obtain that
$A_n: \ker P_n\rightarrow \ker P_{n+1},\; n\ge  0$ is an isomorphism.
Finally, by ({\ref{3.3}}), Theorem {\ref{the 2.1}} and the assumption $0\notin A\sigma(T_0)$  there exist
constants $N,\nu >0$ such that
$$\|U_{n,m}x\|\le Ne^{-\nu (n-m)}\|x\|\; \hbox{ for }\; x \in P_mX\; ,n\ge m\ge 0$$
$$\|U_{\mid m,n}x\|\le Ne^{-\nu (n-m)}\|x\|\; \hbox{for}\; x \in \ker P_n\; , n\ge m\ge 0.$$
Thus the equation ({\ref{1.1}}) has an exponential dichotomy.
\end{proof}

\bigskip
If $X$ is a Hilbert space we need only the closedness of $X_0(0)$. Therefore,
we have 
\begin{corollary}
If $X$ is a Hilbert space then the conditions that
$0\notin A\sigma(T_0)$ and $T$ is surjective are necessary and sufficient for
the equation ({\ref{1.1}})
to have an exponential dichotomy.

\smallskip
{This can be restated as follows:}

\smallskip
If $X$ is a Hinbert space then the condition that  for all $f\in l_\infty$ there exists 
a solution $x\in l_\infty $ of the equation ({\ref{1.2}}) and there exists constant $c>0$ such that 
all of bounded solution $x=\{x_n\}$ (with $x_0=0$ and $f\in l_\infty$) of the equation 
({\ref{1.2}}) satisfies $\sup_{n\in {\bf N}}\|x_n\|\le c\sup_{n\in 
{\bf N}}\|f_n\|$ are necessary and sufficient for the equation ({\ref{1.1}})
to have an exponential dichotomy.

\end{corollary}

\begin{proof} \ The corollary is obvious in view of Corollary {\ref{cor 2.1}} and Theorem 
\ref{the 3.1}.
\end{proof}

\medskip
If $X$ is a finite dimensional space then every subspace of $X$ is closed and complemented. 
Hence, by Theorem {\ref{the 3.1}} we have

\begin{corollary}\label{cor 3.2}
If $X$ is a finite dimensional space, then the condition that $T$ is surjective  is necesary and sufficient for existence of exponential dichotomy
of the equation ({\ref{1.1}}).

\end{corollary}

%%%%%%%%%%%%%%%%%%%%%%%%%%%%%%%%%%%%%%%%%SECTION 4	
%%%%%%%%%%%%%%%%%%%%%%%%%%%%%%%%%%%%%%%%%%%%%%%%%%%

\section{APPLICATION TO EVOLUTION FAMILIES}\label{section 4}
In this section we shall consider evolution families ${\cal U}=U(t,s)_{t\ge s\ge 0}$ 
defined as below.
We shall characterize the exponential dichotomy of ${\cal U}$ by discretizing the evolution family and using the results obtained in previous sections.  

\begin{definition}\rm
A family of operators ${\cal U}=(U(t,s))_{t\ge s\ge 0}$ on a Banach space
$X$ is said to be a {\it (strongly continuous, exponential bounded) evolution family} on the half line
if
\begin{enumerate}
\item $U(t,t)=Id$ and $U(t,r)U(r,s)=U(t,s)$ for $ t\ge r\ge s\ge 0$,
\item The map $(t,s)\mapsto U(t,s)x$ is continuous for every $x\in X$,
\item There are constants $K\ge 0$ and $\alpha \in {\bf R}$ such that
$\|U(t,s)\|\le Ke^{\alpha (t-s)}$ for $t\ge s \ge 0.$
\end{enumerate}
\end{definition}
Then $\omega({\cal U}):=\inf\{\alpha \in {\bf R}:$ there is
$K\ge 0$ such that $\|U(t,s)\|\le Ke^{\alpha (t-s)},\quad t\ge s\ge 0\}$
is called the {\it growth bound} of ${\cal U}$. The notion of evolution families arises
naturally when we are concerned with "well-posed" evolution equations of the form
$$
\frac{du(t)}{dt}=A(t)u(t), \ t \ge 0,
$$
where $A(t)$, for fixed $t$, is in general unbounded linear operator. 
For more details on this notion, conditions for the existence of such families
and applications to partial differential equations
we refer the reader to \cite{Hen}, \cite{Paz}.

\bigskip
For an evolution family ${\cal U}$ and each $t_0\in {\bf R}_+$ we consider the sequence of uniformly bounded operators $\{A_n(t_0)\}_{n\in {\bf N}}$ with $A_n(t_0)=U(t_0+n+1,t_0+n)$ and the following difference equations:
\begin{equation}\label{4.1}
x_{n+1}=A_n(t_0)x_n, \ n\in {\bf N}
\end{equation}
and
\begin{equation}\label{4.2}
x_{n+1}=A_n(t_0)x_n+f_n, \ n\in {\bf N}
\end{equation}
We shall define the two concepts of exponential dichotomy
{\it exponential dichotomy} and  
{\it discrete exponential dichotomy.} 
\begin{definition}
\rm An evolution family ${\cal U}$ =$(U(t,s))_{t\ge s\ge 0}$
on the Banach space $X$
is said to have an {\it exponential dichotomy} if there exist bounded
linear projections $P(t) ,\; t\ge 0$
on $X$ and positive constants  $N,\ \nu$ such that
\begin{enumerate}
\item[a)]$U(t,s)P(s)=P(t)U(t,s),\; t\ge s\ge 0$,
\item[b)]the restriction $U(t,s)_\mid:\ker P(s)\to \ker P(t),\; t\ge s\ge 0$
is an isomorphism (and we denote its inverse by
 $U_\mid(s,t) :\ker P(t)\to \ker P(s)$),
\item[c)]$\|U(t,s)x\|\le Ne^{-\nu(t-s)}\|x\|$ for  $x\in P(s)X,\; t\ge s\ge 0,$
\item[d)]$\|U_\mid (s,t)x\|\le Ne^{-\nu(t-s)}\|x\|$ for $x\in \ker P(t),\; t\ge s\ge 0.$
\end{enumerate}
\end{definition}

\begin{definition}\label{def 4.3}
\rm An evolution family ${\cal U}$ =$(U(t,s))_{t\ge s\ge 0}$
on the Banach space $X$
is said to have a {\it discrete exponential dichotomy}  if  for each $t_0\in {\bf R}_+$ 
the equation (4.1) has exponential dichotomy with family of projection 
$(P_n(t_0))_{n\in {\bf N}}$ and positive constants $N(t_0),\ \nu(t_0)$.
\end{definition}
\begin{definition}\label{def 4.4}
An evolution family ${\cal U}=(U(t,s))_{t\ge s\ge 0}$ is said to have an 
exponential dichotomy (a discrete exponential dichotomy, respectively) in the sense of 
Sacker and Sell if
and only if it has an exponential dichotomy (a discrete exponential dichotomy, respectively) 
and 
$dim\ker P(t)=k<\infty$ for all $t\ge 0$ 
($dim\ker P_n(t_0)=k<\infty$ for all $t_0\ge 0$ and $n\in {\bf N}$, respectively).
\end{definition}
From Theorem {\ref{the 3.1}} we obtain
\begin{theorem}
Let ${\cal U} =(U(t,s))_{t\ge s\ge 0}$ be an evolution family
on the Banach space $X$. Then the following assertions are equivalent:
\begin{enumerate}
\item ${\cal U}$ has a discrete exponential dichotomy for each $t_0\in{\bf R}_+$
\item For each $t_0\in {\bf R}_+,\ f\in l_\infty $ the equation ({\ref{4.2}}) has at least a solution $u\in l_\infty$ and the spaces
$$X_0(t_0)(0):=\{x\in X: \sup_{n\in {\bf N}}\|U(t_0+n,t_0)x\|<\infty\}$$ is complemented in X.
\end{enumerate}
\end{theorem}

\medskip
In what follows we will need the fact that the constants $N,\ \nu$ in Definition
{\ref{def 4.3}} 
are independent of $t_0$. The following lemma supplies a criterion for this.
\begin{lemma}
Let  the evolution family ${\cal U}$ =$(U(t,s))_{t\ge s\ge 0}$ 
on the Banach space $X$ have a discrete exponential dichotomy. We define a bounded linear operator 
$S(t_0): l_\infty\rightarrow l_\infty$ as follows: for 
$x=\{x_n\}\in l_\infty $ put $ (S(t_0)x)_n= A_n(t_0)x_n$. We denote a complement of $X_0(t_0)(0)$ by $X_1(t_0)(0)$ and $Y(t_0):=\{(v_n)_{n\in {\bf N}} \in l_\infty: v_0\in X_1(t_0)(0)\}$.
If there exists a constant $\gamma>1$ such that 
$\|S(t_0)x\|\ge \gamma \|x\|$ for all $ t_0\in {\bf R}_+,\ x\in Y(t_0),$ then the constants $N,\ \nu$ determined by the discrete exponential dichotomy  are independent of $t_0$.
\end{lemma}
\begin{proof} For each $t_0\in {\bf R}$
we define the  operator $L :Y(t_0)\rightarrow l_\infty$ as follows: For  $x=\{x_n\}\in Y(t_0) $ take 
$$(Lx)_n=x_{n+1}\hbox{ (the shift operator) }.$$
Then $\sup\{\|x_0\|,\|x_1\|,...,\|x_n\|,..\}
\ge   \sup\{\|x_1\|,...,\|x_n\|,..\}$, so $\|x\|\ge \|Lx\|$. Therefore, for $x\in Y(t_0)$:
$$\|Tx\|=\|(L-S(t_0))x\|\ge \|S(t_0))x\|-\|Lx\|\ge (\gamma-1)\|x\|.$$
Thus, the constant $\delta$ in the equality ({\ref{3.4}}) 
can be replaced by $\gamma-1$ which is independent of $t_0$. That means the constants
$N,\ \nu$ are independent of $t_0.$
\end{proof}

\begin{theorem}
Let ${\cal U} =(U(t,s))_{t\ge s\ge 0}$ be an evolution family
on the Banach space $X$. If ${\cal U}$ has an exponential dichotomy then ${\cal U}$ has a discrete exponential dichotomy for each $t_0\in {\bf R}_+$ with projections $P_n(t_0)=P(t_0+n)$ 
and positive constants $N,\ \nu $ independent of $t_0$.
\end{theorem}
\begin{proof}
{\bf A)} $A_n(t_0)P_n(t_0)=P_{n+1}(t_0)A_n(t_0).$

In fact, 
$$A_n(t_0)P_n(t_0)=U(t_0+n+1,t_0+n)P(t_0+n) =$$ 
$$P(t_0+n+1)U(t_0+n+1,t_0+n)= P_{n+1}(t_0)A_n(t_0).$$

{\bf B)} $A_n(t_0): \ker P_n(t_0)\rightarrow \ker P_{n+1}(t_0)$ is an isomorphism. We denote its inverse by $A^{-1}_{\mid n}(t_0)$.

This can be derived from the fact that 
$U(t_0+n+1,t_0+n):\ker P(t_0+n)\rightarrow \ker P(t_0+n+1)$ is an isomorphism. 

{\bf C)} If we put $U_{n,m}=A_{n-1}(t_0)A_{n-2}(t_0)..A_m(t_0)$ for $n>m$ and
$U_{m,m}=Id$, then $U_{n,m}=U(t_0+n,t_0+m)$ for $n\ge m\ge 0$. Hence,
$$\|U_{n,m}x\|=\|U(t_0+n,t_0+m)x\|\le Ne^{-\nu(n-m)}\|x\|$$
for $x\in P_m(t_0)X.$

{\bf D)} Denote by $U_{\mid m,n}=A^{-1}_{\mid m}(t_0)A^{-1}_{\mid m+1}(t_0)..A^{-1}_{\mid n-1}(t_0)$ for $n>m$ and $U_{\mid m,m}=Id$ we have $U_{\mid m,n}=U_\mid(t_0+m,t_0+n)$. Therefore, 
$$\|U_{\mid m,n}x\|=\|U_\mid(t_0+m,t_0+n)x\|\le Ne^{-\nu(n-m)}\|x\|$$
for $x\in \ker P_n(t_0).$
\end{proof}

\begin{theorem}
Let ${\cal U} =(U(t,s))_{t\ge s\ge 0}$ be an evolution family
on the Banach space $X$. If for each $t_0\in {\bf R}_+$, ${\cal U}$ has a discrete 
exponential dichotomy in the sense of Sacker and Sell with positive constants $N,\ \nu $ 
independent of $t_0$ then ${\cal U}$ has  an exponential dichotomy in the sense of Sacker and Sell.
\end{theorem}
\begin{proof} We define the family of projections on $X$ as follows: $P(t_0)=P_0(t_0)$ for all $t_0\in {\bf R}_+$. 

\medskip
{\bf A)} There exist $N_1,\ \nu_1 >0$ such that $\|U(t,s)x\|\le N_1e^{-\nu_1(t-s)}\|x\|$ for 
$x\in P(s)X$. 

In fact, let $n\in {\bf N}$ be such that $n\le t-s<n+1$. Then, for $x\in P(s)X$ 
$$\|U(t,s)x\|=\|U(t,s+n)U(s+n,s)x\|\le Ke^\alpha Ne^{-\nu n}\|x\|\le  
KNe^{\alpha+\nu}e^{-\nu (t-s)}\|x\|.$$   
Hence, put $N_1:=KNe^{\alpha+\nu};\ \nu_1:=\nu$  the claim is proved.

\medskip
{\bf B)} Let $X_0(t_0):=\{x\in X:\sup_{t\ge t_0}\|U(t,t_0)x\|<\infty\}$,then $X_0(t_0)=P(t_0)X$.

In fact, from {\bf A)} we have $P(t_0)X\subseteq X_0(t_0)$. Set $x\notin P(t_0)X$. Then
$x=P(t_0)x+(Id-P(t_0))x$ with $(Id-P(t_0))x\not= 0$. Therefore, $$\|U(t_0+n,t_0)x\|=
\|U(t_0+n,t_0)P(t_0)x+U(t_0+n,t_0)(Id-P(t_0))x\|$$
$$\ge \|U(t_0+n,t_0)(Id-P(t_0))x\|-\|U(t_0+n,t_0)P(t_0)x\|.$$
Since ${\cal U}$ has a discrete exponential dichotomy we have  $\|U(t_0+n,t_0)P(t_0)x\|\rightarrow 0$ and $\|U(t_0+n,t_0)(Id-P(t_0))x\|\rightarrow \infty$ when $n\rightarrow \infty$. Hence, $ \|U(t_0+n,t_0)x\|\rightarrow \infty$ when $n\rightarrow \infty$.
So $x\notin X_0(t_0)$. Thus, $X_0(t_0)\subseteq  P(t_0)X$. Therefore, $P(t_0)X = X_0(t_0)$

\medskip
{\bf C)} $U(t,t_0)P(t_0)X\subseteq P(t)X$. 

In fact, let $x\in P(t_0)X\Leftrightarrow P(t_0)x=x$ then $\|U(t,t_0)x\|$ bounded for $t\ge t_0$. We have, 
$$\sup_{s\ge t} \|U(s,t)U(t,t_0)x\|\le \sup_{s\ge t_0} \|U(s,t_0)x\|<\infty.$$ Hence, from {\bf B)} we get  $U(t,t_0)x\in P(t)X$.
 
\medskip
{\bf D)} $U(t,t_0)_\mid \ker P(t_0) $ is one to one.

In fact, for the purpose of contradiction let $0\not= x\in \ker P(t_0): U(t,t_0)x=0$. Taking
$n\in {\bf N}$ such that $t_0+n>t$ we have $U(t_0+n,t)U(t,t_0)x=0$ or $U(t_0+n,t_0)x=0$. This contradicts to the fact that $U(t_0+n,t_0):\ker P(t_0)\rightarrow \ker P_n(t_0)$ is isomorphism.

\medskip
{\bf E)}
Because a complement of a complemented subpace of Banach space $X$ is not unique, the family $(P_n(t_0))_{n\in {\bf N}}$ 
(precisely, the family of spaces $(\ker P_n(t_0))_{n\in {\bf N}}$) for each $t_0\ge 0$ is not unique. However, we shall point out that for each $t_0\ge 0$ there exists a family $(P_n(t_0))_{n\in {\bf N}}$ such that:
$$U(t_1,t_0)\ker P_0(t_0)= \ker P_0(t_1)\hbox{ for } t_1\ge t_0\ge 0.$$

\medskip
Firstly we prove that $ U(t_1,t_0)\ker P_0(t_0)$ is a closed subspace of $X$. Indeed, take $n\in N$ such that $n+1\ge t_1-t_0\ge n$, for all $y\in \ker P_0(t_0)$ we have 
\begin{eqnarray*}
Ke^\alpha\|U(t_1,t_0)y\| &\ge& \|U(t_0+n+1,t_1)U(t_1,t_0)y\|\\
&=&
\|U(t_0+n+1,t_0)y\|\ge Ne^{(n+1)\nu}\|y\|.
\end{eqnarray*}
Hence,
\begin{equation}\label{4.3}
\|U(t_1,t_0)y\|\ge \frac{N}{K}e^{-\alpha}e^{(n+1)\nu}\|y\|\ge 
\frac{N}{K}e^{-\alpha}e^{\nu (t_1-t_0)}\|y\|.
\end{equation}
From this inequality and the closedness of $\ker P_0(t_0)$ we easily derive that 
the space \ $ U(t_1,t_0)\ker P_0(t_0)$ is a closed subspace of $X$. 

\medskip
Now we prove that $ U(t_1,t_0)\ker P_0(t_0)\cap X_0(t_1) = {0}$. 

Indeed, suppose that $x\in U(t_1,t_0)\ker P_0(t_0)\cap X_0(t_1)$. 
Then from definition of $X_0(t_1)$ we have that \ $\sup_{t\ge t_1}\|U(t,t_1)x\|=M<\infty$. Since $x\in U(t_1,t_0)\ker P_0(t_0)$, there exists
$y\in\ker P_0(t_0)$ such that $x= U(t_1,t_0)y$. By the inequality ({\ref{4.3}}) we have 
$$M\ge \|U(t,t_1)x\|=\|U(t,t_1)U(t_1,t_0)y\|=\|U(t,t_0)y\|\ge \frac{N}{K}e^{-\alpha}e^{\nu (t-t_0)}\|y\|$$
for all $t\ge t_1\ge t_0$. Therefore, $y=0$, thus $x=0$. 

Since $U(t_1,t_0)_\mid\ker P_0(t_0)$ is one to one we have 
$$dim U(t_1,t_0)\ker P_0(t_0)= dim \ker P_0(t_0)= k = dim\ker  P_0(t_1).$$ 
That means we have
$$ X=X_0(t_1)\oplus U(t_1,t_0)\ker P_0(t_0).$$
Hence, we can take $P_0(t_1)$ as the projection on to $X_0(t_1)$ with 
$$
ker P_0(t_1)=U(t_1,t_0)\ker P_0(t_0).$$ 
Therefore, 
$$U(t_1,t_0)\ker P_0(t_0)= \ker P_0(t_1)\hbox{ for } t_1\ge t_0\ge 0.$$

\medskip


From {\bf D)} and {\bf E)} we have $U(t,t_0):\ker P(t_0) \rightarrow \ker P(t)$ is an isomorphism and we denote its inverse by $U_\mid(t_0,t): \ker P(t) \rightarrow \ker P(t_0)$, for $t\ge t_0\ge 0$.

\medskip
{\bf F)} $\|U_\mid(t_0,t)x\|\le N_1e^{-\nu_1(t-t_0)}\|x\|$ for $x\in \ker P(t)$ and $t\ge t_0\ge 0$.

In fact, firstly we prove that for $t\ge s\ge 0$ with $0\le t-s\le 1$ there exists $0<M<\infty$ such that
$\|U_\mid(s,t)x\|\le M\|x\|$ for $x\in \ker P(t)$. Indeed, 
since $U_\mid(s,t)x\in \ker P(s)$ for $x\in \ker P(t)$ we have
$$Ke^{\alpha }\|x\|\ge \|U(s+1,t)x\|=\|U(s+1,s)U_\mid(s,t)x\|\ge Ne^{\nu}\|U_\mid(s,t)x\|.$$
Hence, $\|U_\mid(s,t)x\|\le \frac{K}{N}e^{\alpha-\nu}\|x\|$ for $x\in \ker P(t)$, so we may take $M:=\frac{K}{N}e^{\alpha-\nu}$.
Now, let $n\in {\bf N}$ such that $n\le t-t_0\le n+1$ then,
$$\|U_\mid(t_0,t)x\|=\|U_\mid(t_0,t_0+n)U_\mid(t_0+n,t)x\|\le Ne^{-\nu n}M\|x\| \le  NMe^\nu e^{-\nu(t-t_0)}\|x\|.$$
Take $N_1:=NMe^\nu;\ \nu_1=\nu$ then the claim is proved .

\medskip
{\bf G)} $U(t,s)P(s)=P(t)U(t,s)$.

In fact, for $x\in \ker P(s): U(t,s)P(s)x=P(t)U(t,s)x=0$ and 
$$x\in  P(s)X: U(t,s)(Id-P(s))x=(Id-P(t))U(t,s)x=0.$$ 
Thus,  $$U(t,s)P(s)x=P(t)U(t,s)x.$$
Therefore, for $x\in X$ we have $x=x_1+x_2$ with 
$x_1\in \ker P(s)$ and $x_2\in P(s)X$. So, $$U(t,s)P(s)x=U(t,s)P(s)(x_1+x_2)= 
P(t)U(t,s)(x_1+x_2)=P(t)U(t,s)x.$$
\end{proof}

\def \Sm{Semigroups}
\def \sm{semigroups}
\def \esta {exponential stability}
\def \exp{exponential}
\def \Exp{Exponential}
\def \Dic{Dichotomies}
\def \dic{dichotomies}
\def \Sta {Stability}

\vspace{.5 cm}

\bibliographystyle{amsplain}
\begin{thebibliography}{10}

\bibitem{AulMin} B. Aulbach, N. V. Minh, {Semigroups and exponential
stability of nonautonomous linear differential on the half-line.} {\it in} (R.P.
Agrawal Ed.), "Dynamical systems and Application", World Scientific, Singapore
{1995, pp. 45-61.}
\bibitem{AulMin2} B. Aulbach, N. V. Minh, 
The concept of spectral dichotomy for difference equations. II., 
Journal of  Difference Equations and Applications {\bf 2}(1996), N.3, 251-262.
\bibitem{cofsch1}
C.V. Coffman, J.J. Sch\"affer, 
Dichotomies for linear difference equations, {\it Math. Anal. } {\bf 172}
(1967), 139-166.

\bibitem{cofsch2}
C.V. Coffman, J.J. Sch\"affer, {Linear differential equations with delays: 
Admissibility and conditional exponential stability}, {\it J. Diff. Eq.} {\bf 9}(1971), 
521-535. 


\bibitem{cholei} 
S.N. Chow, H. Leiva, { Existence and roughness of the exponential 
dichotomy for skew-produt semiflows in Banach spaces}, {\it J. Diff. Eq.} 
{\bf 120} (1995), 429-477.


\bibitem{DalKre} L. Ju. Daleckii, M.G. Krein, { ''\Sta\ of solution of differential
equation in Banach spaces''}. Trans. Amer. Math. Soc. . Provindence RI, {1974.}
\bibitem{Dat} R. Datko, Uniform asymptotic stability of evolutionary processes in a
Banach space, {\it SIAM J. Math. Anal.} {\bf 3}(1972), 428-445.

\bibitem{Hen} D.~Henry, {"Geometric Theory of Semilinear Parabolic 
Equations",} Lecture Notes in Mathematics, No.~840, Springer, 
Berlin-Heidelberg-New York 1981.



\bibitem{Zhi} B.M. Levitan, V.V. Zhikov, {"Almost Periodic functions
and Differential Equations"}. Moscow Univ. Publ. House 1978. English tranl.
by Cambrige University Press 1982.

\bibitem{Li}
T. Li, Die Stabilitatsfrage bei Differenzegleichungen, {\it Acta Math.}, {\bf 63}(1934),
99-141.

\bibitem{MasSch} J. J. Massera, J. J. Sch\" affer, { ''Linear differential Equations and
function spaces''}. Academic Press, New York 1966.
\bibitem{MRS} N.V. Minh, F. Rabiger, R. Schnaubelt {Exponential dichotomy
Exponential expansiveness and exponential Dichotomy of evolution equation on the
half line,}, {\it Integral Eq. and Oper. Theory.} {\bf 32}(1998), 332-353.
\bibitem{MurNaiMin}
S. Murakami, T. Naito, N.V. Minh, Evolution semigroups and sums of commuting operators:
a new approach to the admissibility of function spaces, {\it J. Diff. Eq.}. To appear.
\bibitem{NaiMin}
T. Naito, N.V. Minh, Evolution semigroups and spectral
criteria for almost periodic solutions of periodic evolution equations, {\it J. Diff. Eq.}
{\bf 152}(1999), 358-376.
\bibitem{Nee} 
J. van Neerven,  "The asymptotic Behaviour of Semigroups of Linear Operator",
Operator Theory, Advances 
and Applications Vol.88 .Birkha$\ddot{u}$ser, Basel-Boston-Berlin 1996.

\bibitem{Paz} A. Pazy, {" Semigroup of Linear operators and application to partial
differential equations"}. Springer-Verlag, Berlin 1983.
\bibitem{Per} O. Perron, Die Stabilit\"atsfrage bei Differentialgleichungen, 
{\it Math. Z.} {\bf 32}(1930), 703-728.
\bibitem{sacsel2} 
R. Sacker, G. Sell, {Dichotomies for linear evolutionary equations 
in Banach spaces}, {\it J. Diff. Eq.} {\bf 113} (1994), 17-67.

\bibitem{Slj}V.E. Sljusarchuk, {On exponential dichotomy of solutions 
of discrete systems,} {\it Ukrain.~Mat.~Zh.} {\bf 35} (1983), 109-115. 



\end{thebibliography}
\end{document}
