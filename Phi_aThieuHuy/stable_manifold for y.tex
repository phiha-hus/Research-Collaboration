We define the so-called \emph{constrained manifold}, which all the solution to \eqref{eq1} must lie on
%
\[
\bbL(t,x):= \{(t,x) \in \r_+ \times \r^n\ | \  0 = A_2(t)x(t) + f_2(t,x(t)) \}  \ . 
\]
%
We further notice that this manifold is of dimension $d$, which is the degree of freedom to the DAE \eqref{eq1}. 
Now we are able to introduce the concept of a stable manifold for the solutions of the integral-algebraic coupled system \eqref{mild equation}.

\begin{definition}\label{def4.2}
	A subset $\bbM$ of the constrained manifold $\bbL(t,x)$ is said to be an \emph{invariant stable manifold} for the solutions to $\eqref{mild equation}$  if for every $t\in\hro{R_+}$  the phase space $\r^{d}$ splits into a direct sum  $\r^{d}=X_0(t)\oplus  X_1(t)$  such that
	\begin{equation*}
	\inf_{t\in\hro{R_+}}Sn(X_0(t), X_1(t)):=\inf_{t\in\hro{R_+}} \ \inf\{\|x_0+x_1\|, \  x_i\in X_i(t),\  \|x_i\| = 1, \ i=0,1 \} >0,
	\end{equation*}
	and if there exists a family of Lipschitz continuous mappings
	\begin{equation*}
	g_t:X_0(t)\rar X_1(t), \quad  t\in\hro{R_+},
	\end{equation*}
	with the Lipschitz constants independent of  $t$ such that
	\begin{enumerate}
		\item[(i)]$\bbM=\{(t,x + g_t(x))\in\hro{R_+} \times (X_0(t)\oplus X_1(t))\mid x\in X_0(t)\}$, and we denote by $\bbM_t:=\{ x+g_t(x)\mid (t,x+g_t(x))\in \bbM \}$,
		\item[(ii)]$\bbM_t$  is homeomorphic to $X_0(t)$  for all $t\geq 0$,
		\item[(iii)] to each $x_0\in\bbM_{t_0}$  there corresponds one and only one solution  $x$ to $\eqref{mild equation}$  satisfying $x(t_0)=x_0$  and  $ess\sup_{t\geq t_0}\|x(t)\|<\infty$,
		\item[(iv)]$\bbM$  is invariant under system $\eqref{mild equation}$, i.e., if $x$ is a solution to $\eqref{mild equation}$, and $ess\sup_{t\geq t_0}\|x(t)\|<\infty$, then $x(s)\in \bbM_s$
		for all  $s\geq t_0.$
	\end{enumerate}
\end{definition}

For notational convennience, we can identify $X_0(t)\oplus X_1(t)$  with  $X_0(t)\times X_1(t)$ and write $\bbM_t=graph(g_t)$.

\vskip 1cm

We define the so-called \emph{constrained manifold}, which all the solution to \eqref{eq4.1a}-\eqref{eq4.1b} must lie on
%
\[
\bbL(t,y):= \{(t,y) \in \r_+ \times \r^n\ | \  y_2(t) = \hA_3(t)y_1(t) + \hf_2(t,y(t)) \}  \ . 
\]
%
We further notice that this manifold is of dimension $d$, which is the degree of freedom to the DAE \eqref{mild equation}. Now we are able to introduce the concept of a stable manifold for the solutions of the integral-algebraic coupled system \eqref{mild equation}.

\begin{definition}\label{def4.2}
	A subset $\bbM$ of the constrained manifold $\bbL(t,y)$ is said to be an \emph{invariant stable manifold} for the solutions to $\eqref{mild equation}$  if for every $t\in\hro{R_+}$  the phase space $\r^{d}$ splits into a direct sum  $\r^{d}=Y_0(t)\oplus  Y_1(t)$  such that
	\begin{equation*}
	\inf_{t\in\hro{R_+}}Sn(Y_0(t), Y_1(t)):=\inf_{t\in\hro{R_+}} \ \inf\{\|Y_0+Y_1\|, \  Y_i\in Y_i(t),\  \|Y_i\| = 1, \ i=0,1 \} >0,
	\end{equation*}
	and if there exists a family of Lipschitz continuous mappings
	\begin{equation*}
	g_t:Y_0(t)\rar Y_1(t), \quad  t\in\hro{R_+},
	\end{equation*}
	with the Lipschitz constants independent of  $t$ such that
	\begin{enumerate}
		\item[(i)]$\bbM=\{(t,y + g_t(y))\in\hro{R_+} \times (Y_0(t)\oplus Y_1(t))\mid y\in Y_0(t)\}$, and we denote by $\bbM_t:=\{ y+g_t(y)\mid (t,y+g_t(y))\in \bbM \}$,
		\item[(ii)]$\bbM_t$  is homeomorphic to $Y_0(t)$  for all $t\geq 0$,
		\item[(iii)] to each $Y_0\in\bbM_{t_0}$  there corresponds one and only one solution  $y$ to $\eqref{mild equation}$  satisfying $y(t_0)=Y_0$  and  $ess\sup_{t\geq t_0}\|y(t)\|<\infty$,
		\item[(iv)]$\bbM$  is invariant under system $\eqref{mild equation}$, i.e., if $y$ is a solution to $\eqref{mild equation}$, and $ess\sup_{t\geq t_0}\|y(t)\|<\infty$, then $y(s)\in \bbM_s$
		for all  $s\geq t_0.$
	\end{enumerate}
\end{definition}