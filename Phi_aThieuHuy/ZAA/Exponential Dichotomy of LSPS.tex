\documentclass[12pt,reqno]{amsart}
\usepackage{amsmath,amssymb,amsfonts,amsthm}

\textwidth=16.5cm
\textheight=23cm
\oddsidemargin=0cm
\evensidemargin=0cm

\newtheorem{theorem}{Theorem}[section]
\newtheorem{lemma}[theorem]{Lemma}
\theoremstyle{definition}
\newtheorem{definition}[theorem]{Definition}
\newtheorem{example}[theorem]{Example}
\newtheorem{xca}[theorem]{Exercise}
\newtheorem{corollary}[theorem]{Corollary}
\newtheorem{proposition}[theorem]{Proposition}
\theoremstyle{remark}
\newtheorem{remark}[theorem]{Remark}
\renewcommand{\labelenumi}{\roman{enumi})}
\numberwithin{equation}{section}

%font\name=cmr8
\begin{document}
\def\cal{\mathcal}
%\pagestyle{plain}
\def \ud{\underline }
\def\id{{\indent }}
\def\f{\frac}
%\def\frac{\fracrac}
\def\non{{\noindent}}
 \def\leq{\leqslant} 
\def\rar{\rightarrow}
\def\Rar{\Rightarrow}
\def\ti{\times}
\def\si{\sigma}
\def\Ga{\Gamma}
\def\ga{\gamma}
\def\ld{{\lambda}}
\def\Si{\Psi}
%\def\oplus{\tau}
\def\a{\alpha}
\def\p{\varphi}
\def\ro{\rho}
\def\P{\varPhi}
\def\de{\delta}
\def\ep{\varepsilon}
\def\Ep{\epsilon}
\def\De{\Delta}
\def\om{\omega}
\def\Om{\Omega}
\def\ift{\infty}
\def\tet{\theta}
\def\Tet{\Theta}
\def\hro{\mathbb}
\def\ho{\mathcal}
\def\E{\mathcal{E}}
\def\n{\mathbb{N}}
\def\A{A\si({B})}
\def\AT {A\si(T_{0\tet})}
\def\lsps{linear skew- product semiflow}
\def\ed{exponential dichotomy}
\def\udd{ uniform discrete-dichotomy}
\def\pdd{ pointwise-discrete-dichotomy}
\def\sf{semi-flows}
\def\dsp{discrete-skew-product}
\def\Pnt{\P_{n}(\tet)}
\def\pnt{P_{n}(\tet)}
\def\mt{M_{\tet}}
\def\at{\a_{\tet}}
\def\l{l_{\ift}(\hro{N};X)}
\def\lo{l_{\ift}^{0}}
\def\lz{l_{\ift}^{Z(\tet)}}
\def\con{\subset}
\def\Con{\subseteq}
\def\td{\Leftrightarrow}

%\thispagestyle{empty}


\title[Characterizations of Exponential Dichotomy of LSPS]{Characterizations of Exponential Dichotomy 
of Linear Skew-Product Semiflows over semiflows}
\author[N.T. Huy]{Nguyen Thieu Huy}
\address{Nguyen Thieu Huy, Faculty of Applied Mathematics and Informatics,
Ha Noi University of Technology,
Khoa Toan-Tin ung dung, Dai hoc Bach khoa Ha Noi,
1 Dai Co Viet, Hanoi, Vietnam}
\email{huynguyen@mail.hut.edu.vn}
\author[H. Phi]{Ha Phi}
\address{Ha Phi,  Department of Mathematics, Hanoi University of Education,
Khoa Toan-Tin, Dai Hoc Su Pham Ha Noi,
136 Xuan Thuy St., Hanoi, Vietnam}
\email{hpdhsp@yahoo.com}

%\date{January,20,2006}
%\dedicatory{}

\begin{abstract} Using the method of discretization, we investigate the necessary and 
%\break \hskip 3cm
sufficient conditions  for the existence of
exponential dichotomy of linear skew-product semiflows over semiflows through the existence of discrete exponential dichotomy of the discretized linear-skew product semiflows. We then apply the obtained results to consider the roughness of exponential dichotomy of linear-skew product semiflows.
 \end{abstract}
\keywords{Linear Skew-product Semiflows, Discrete skew-product, uniform discrete dichotomy, pointwise discrete dichotomy, 
     exponential dichotomy, perturbations}
\maketitle


%\baselineskip= .6cm

%%%%%%%%%%%%%%%%%%%%%%%%%%%%%%%  Section 1
%%%%%%%%%%%%%%%%%%%%%%%%%%%%%%%%%%
 \section{INTRODUCTION AND PRELIMINARIES} \label{section 1}
 The notion of linear skew-product semiflows (LSPS) arises naturally when one
considers the linearization along an invariant manifold
of a dynamical system (generated by an autonomous differential 
equation),
especially in the infinite dimensional case \cite[Chapt. 4]{SY}.
 On the other hand, one can associate to a linear nonautonomous
differential equation a linear skew-product semiflow
 whose asymptotic behavior such as exponential stability or dichotomy is
closely related to that of the equation under consideration (see 
 Sacker and Sell \cite{sacsel,sacsel2,sacsel3,sacsel4,sacsel5,sacsel6} and Hale \cite{hal}). 
Recently (see, e.g., Chow and Leiva \cite{cholei}, Latushkin and Schnaubelt \cite{LS}, 
Sell and You \cite{SY}),
there has been an increasing interest in the asymptotic behavior of
linear skew-product semiflows
 over flows. To the best of our knowledge, the main results
 in this direction are focused on the characterization of exponential 
dichotomy
 of linear skew-product semiflows over flows in terms of
 Sacker-Sell spectral properties \cite{sacsel} or
 the hyperbolicity of the associated evolution semigroups and their
generators \cite{LS}, \cite{cholei}. In particular, 
 a characterization of exponential dichotomy for linear skew-product
semiflows over flows was given in \cite{sacsel} assuming the dimension of the unstable manifold 
to be finite. Meanwhile, in
\cite{LS} a characterization is given through the 
hyperbolicity
 of the associated evolution semigroup and its generator. 
 Another characterization in \cite{cholei} uses 
discrete
linear skew-product semiflows over discretized flows. 
%(which seems to be).

 \medskip
 Once the exponential dichotomy is characterized, it is
natural to study
 its robustness (linear perturbation) and the 
existence of invariant manifolds (nonlinear perturbation). For more
information in this direction
 we refer the reader to \cite{aulmin}, \cite{cholei},  \cite{sacsel}
 and the references therein.

 \medskip
 In this paper we will make an attempt to characterize  exponential
dichotomy in a more general setting and  consider  {\it 
linear
skew-product semiflows
 over semiflows}, i.e., there is only a semiflow on the base space. This 
setting is particularly appropriate in
the infinite dimensional case since in this case the dynamical 
systems
restricted to 
 invariant manifolds are only semiflows in general. 
%Moreover, this setting
%allows one to apply the results to stochastic dynamical systems as shown in
%Example \ref{exa} below. 

Our method is to discretize 
the linear skew-product semiflow to obtain the corresponding discrete linear skew-product semiflow. This is a natural extension 
of the discretizing technique which can be traced back to Henry \cite{Hen}. 
As a result, we obtain a characterization 
of the exponential dichotomy of a linear skew-product semiflow  over a semiflow. We
then use this characterization to prove the robustness of exponential dichtomy.
Our results are contained in theorems \ref{the 3.5}, \ref{the 3.7}, \ref{the 3.12}, \ref{the 4.2} and
\ref{the 4.3}. We now start by some preliminaries.
 
 
 Consider the (trivial) Banach bundles $\ho{E} := X\times \Theta$ where X is a fixed Banach space(the state space) and $\Theta$  is Hausdorff metric space(the base space).
Throughout this paper we shall consider the following sequence spaces endowed with the sup-norm.
\begin{eqnarray*}
l_\infty({\mathbb N},X)&:=&\{v=\{v_n\}_{n\in {\mathbb N}}:v_n\in X:\sup_{n\in {\mathbb N}}\|v_n\|<\infty\}:=l_\infty \\
l^0_\infty({\mathbb N},X)&:=&\{v=\{v_n\}\in l_\infty;\; v_0=0\}:=l_\infty^0\\
l_\infty([n_0,\infty),X)&:=&\{v=\{v_n\}\in l_\infty;\; 0<n_0\leq n\in {\hro N}\}.
\end{eqnarray*}
On the base space $\Theta$, the semiflow is now defined as follows. 
\begin{definition}\label{def 1.1}
A family  $(\varphi^t)_{t\geq0} $\quad  of maps $\p^t:\Tet\rightarrow\Tet $ is
  called  a  continuous  semiflow  on \quad $\Tet  $ if   the  following  properties  hold:
\begin{enumerate}
\item[(i)] The map $(\tet,t)\rightarrow\p^t\tet$ is continuous,
\item[(ii)] $\p^0\tet=\tet ; \p^{t+s}=\p^t\p^s$ for all $t,s \in{\hro R}_+$ and $\tet\in\Tet.$
\end{enumerate}
\end{definition}
Given a semiflow, the linear skew-product semiflow (LSPS) can then be defined as follows.
\begin{definition}\label{def 1.2}
A linear skew-product semiflow  $\Pi=(\P;\p)$ on $\ho{E}=X\times\Tet$ is a mapping:
\begin{eqnarray*}\Pi:\ho{E}\ti\hro{R}_+&\to&\ho{E}\cr
\Pi(x,\tet,t)&=&(\P(\tet;t)x;\p^t\tet)
\end{eqnarray*}
where $ (\p^t)_{t \geq 0}$ is the semiflow on 
$\Tet$ and $(\P(\tet;t))_{\tet\in\Tet;t\in\hro{R}_+}$ is the so-called
(strongly continuous, exponential bounded) cocycle satisfying the following properties.
\begin{enumerate}
\item[(1)] $\P(\tet;t)\in\ho{L}(X);\qquad \P(\tet;0)=Id$ (the indentity operator on X)
 for all $\tet\in\Tet;t\in\hro{R}_+.  $
\item[(2)] The map $(\tet;t)\rar\P(\tet;t)x$ are continuous for each $x\in\ X.$
\item[(3)] The cocycle identity satisfied ,i.e.,
     $$\P(\tet;t+s)=\P(\p^t\tet;s)\P(\tet;t)\hbox{ for all }t,s \geq 0\hbox{ and  }
\tet\in\Tet.$$
\item[(4)]  There exist  constants $N,\a$ such that  $$ \| \P(\tet;t) \| \leqslant 
{ Ne^{\a t}}\qquad \hbox{for}\quad t \geq 0 \quad \hbox{and}\quad  \tet\in\Tet.$$
\end{enumerate}
\end{definition}

To define and study the exponential dichotomy of  LSPS we need the following concept of projector on $\ho{E}$.

\begin{definition}\label{def 1.3}
\begin{enumerate}
\item[(i)]A mapping  $ \bf {P}:\ho{E}\rar\E $  is said to be a projector if $ {\bf { P}}$ has the form
        $${\bf {P}}(x;\tet)=(P(\tet)x;\tet)\hbox{ where,  } P(\tet) \in\ho{L}(X)\hbox{ is a projection on } X \hbox{ for each }\theta\in \Theta.$$
   \item[(ii)]  The projector  $\bf {P}$  is said to be invariant with respect to the cocycle $(\P(\tet;t))_{\tet\in\Tet;t \geq 0}$    if it satisfies
           $$P(\p^t\tet)\P(\tet;t)=\P(\tet;t)P(\tet) ; \qquad  t \geq 0;\tet\in\Tet.$$
\end{enumerate}
\end{definition}

We now define the exponential dichotomy of a LSPS over semiflows.
\begin{definition}\label{def 1.4}
A linear skew-product semiflow (LSPS)  $\Pi$  on a Banach  bundles $ X\ti\Tet$ is said  to have an exponential dichotomy 
if there exists an invariant  projector ${\bf{P}}$ such that : 
\begin{enumerate}
\item[(a)]$ \P(\tet;t)\mid_{\ker P(\tet)} : \ker P(\tet)\rar \ker P(\p^t\tet)$ is an 
isormorphirsm  whose inverse is denoted by
       $$ \P(\p^t\tet;-t): \ker P(\p^t\tet)\rar  \ker P(\tet),$$
\item[(b)] There exist constants $N;\nu > 0$ such that\\
     $ \|\P(\tet;t)x\|\leq Ne^{-\nu t}\|x\|$ for $x\in P(\tet)X;t \geq 0$,\\
$\|\P(\tet;t)x\|      \leq Ne^{\nu t}\|x\|$  \hskip 0.25cm   for $x\in \ker P (\tet);t <  0.$
\end{enumerate}
\end{definition}


%%%%%%%%%%%%%%%% Section2
%%%%%%%%%%%%%%%%%%%%%%%%%%%%%%%%%%%%%%%
 \section{DISCRETE LINEAR SKEW-PRODUCT SEMIFLOWS  OVER DISCRETIZED SEMIFLOWS} \label{section 2}
As said above, to investigate the exponential dichotomy of a LSPS we use the method of discretization. That is, we discretize the (continuous) LSPS to obtain some kind of discrete 
 skew-product.  We next use the so-called  input-output technique (see \cite{Hen, cholei}) to obtain characterizations of the exponential dichotomy of discrete skew-product. We then convert the results to the (continuous) LSPS by proving the equivalence between the existence of exponential dichotomy of LSPS and that of its dicretized skew-product.  We now start with the basic definition and properties of discrete LSPS.

\begin{definition}\label{def 2.1}
Let $(\p^t\tet)_{t \geq 0}$ is a  semiflow on $\Tet.$
 A discrete skew-product is a mapping $\Pi^{*}=(\Phi;\varphi) :\E\ti \hro{N}\rar\E$   which is given by
$$\Pi^{*}(x,\tet,n)=(\P_n(\tet)x;\p^n\tet)$$ where, $\P_n(\tet)\in\ho{L}(X)$ for each $(n,\theta)\in  \hro{N}\times\Theta$, satisfying the following properties:
\begin{enumerate}
\item[(i)]there exists $ \ro>0$ such that $ \|\P_n(\tet)\|\leq\ro$ for all
 $n\in {\hro {N}}$ and $\tet\in\Tet,$
\item[(ii)]the mapping $(x;\tet)\rar\P_n(\tet)x$ is continuous for each $ n\in\hro{N}$.
\end{enumerate}
\end{definition}
\begin{remark}
For a (continuous) LSPS $\Pi=(\Phi,\varphi)$ we will discretize it by simply setting
$\Phi_n(\theta):=\Phi(\varphi^n\theta,1)$. Then we obtain the corresponding discrete skew-product $\Pi^{*}=(\P;\p)$.
\end{remark}
We now define the exponential dichotomy of a discrete skew-product.
\begin{definition}\label{def 2.2}
We say that a discrete skew product $ \Pi^{*}$ has an pointwise discrete dichotomy over $ \Tet$ if for each $\tet\in\Tet$ there exist  the positive constants
$M_{\tet}$, $\a_{\tet}<1$; and the family  of projections ${\{}P_{n}(\tet)\}_{n\in\hro{N}}$  in X such that:
\begin{enumerate}
\item[(1)]$\P_{n}(\tet)P_{n}(\tet)=P_{n+1}(\tet)\P_{n}(\tet)$
\item[(2)]$\Pnt\mid_{Ker \pnt}:Ker \pnt\rar KerP_{n+1}(\tet)$ is an isomorphirsm,
and we denote its inverse  by  $ \P_{n}^{-1}:Ker P_{n+1}(\tet)\rar Ker\pnt\qquad  n \geq 0$
\item[(3)] $$ \| \P_{n,m}(\tet)x\| \leq M_{\tet}  \a_{\tet}^{n-m}\|x\|
   \mbox{ for   $x\in P_{m}(\tet)X$    and   $n\geq m \geq 0$} $$
 $$\mbox{where} \quad \P_{n,m}(\tet)=\P_{n-1}(\tet)\P_{n-2}(\tet)...\P_{m}(\tet) (n>m)\quad   \mbox{and}\quad 
    \P_{m,m}=I$$
\item[(4)] $$ \|  \P_{n,m}(\tet)x\|  \leq M_{\tet}  \a_{\tet}^{n-m}\|x\|  
      \mbox {  for  $x\in  KerP_{m}(\tet)$  and $n<m$ }$$ 
$$\mbox{where}  \quad \P_{n,m}(\tet)=\P^{-1}_{m+1}(\tet)\P^{-1}_{m+2}(\tet)...\P^{-1}_{n}
(\tet)(n<m)$$\\ 
\end{enumerate}
The projections $P_{n}(\tet)$ and the constants $M_{\tet}$, $\a_{\tet}$ are called dichotomy projections and  dichotomy constants, respectively.
\end{definition}
\begin{remark}\label{Remark}
If  $\Pi_{*}$ has a pointwise discrete dichotomy over  $\Tet$   then by the same argument as in [3] we can see that  $\sup_{n\geq 0}\|P_{n}(\tet)\|<\infty $.   Therefore  in definition 2.2, the condition (3) is equivalent to the condition
$ \| \P_{n,m}(\tet)P_{m}(\tet)\| \leq M_{\tet}  \a_{\tet}^{n-m}(n\geq m\geq 0)$     and,the condition(4) is equivalent to  the condition $ \|  \P_{n,m}(\tet)(Id-P_{m}(\tet))\|  \leq M_{\tet}  \a_{\tet}^{n-m}(n<m)$.\\
\end{remark}  
\begin{definition}\label{def 2.4}
We say that a discrete skew product $ \Pi^{*}$ has a uniform discrete dichotomy over $\Tet$ if in definition \ref{def 2.2}  the dichotomy contants $M_{\tet},\a_{\tet}$ are independent of $\tet$  (i.e,  $M_{\tet}=M;\a_{\tet}=\alpha\quad  \mbox {for all}\quad \tet\in\Tet$).
\end{definition}
From definition \ref{def 2.2}  the following properties are obivious.
\begin{remark}\label{Remark 2.5}
 For the discrete skew-product having an exponential dichotomy, the following assertions hold.
\begin{enumerate} 
\item $\P_{n,m}(\tet)\P_{m,k}(\tet)=\P_{n,k}(\tet)$,
\item $\P_n(\tet)\P_{n,m}(\tet)=\P_{n+1,m}(\tet) \quad$ for all $n>m\geq k$, 
\item $\P_{n,m}(\tet)P_m(\tet)=P_n(\tet)\P_{n,m}(\tet)\quad$ for all $n\geq m$,
\item $\P_{n,n-1}(\tet)=\P_{n-1}(\tet);\qquad  \P_{n,r}(\tet)\P_{r-1}(\tet)=\P_{n,r-1}(\tet)$
 for $1\leq r<n$.
\end{enumerate}
\end{remark}
For later use, we state the following lemma whose proof can be done easily by induction.
\begin{lemma}\label{lem 2.5}
Let   $ \Pi^{*} :\E\ti\hro{N}\rar\E$ be a discrete skew product.\\If for a sequence $y=\{y_n\}$  in X and   $\tet\in\Tet$   we have\quad $x_{n+1}=\P_n(\tet)x_n+y_n\quad\forall n\in\hro{N}$\quad then $$x_n=\P_{n,m}(\tet)x_m+\sum _{k=m}^{n-1}\P_{n,k+1}(\tet)y_k;\quad n>m\geq 0.$$
\end{lemma}       
 We also need the following notion of the discrete Green's function.
\begin{definition}\label{lem 2.7}
Assume that   a discrete skew product   $\Pi^{*}$   has pointwise discrete dichotomy with corresponding dichotomy constants
 $\mt;\at$  and projections $\{\pnt\}_{n\in \hro{N}}.$ We define  the discrete Green's function as follows:
    $$G_{n,m}(\tet)=\begin{cases}
\P_{n,m}(\tet)P_{m}(\tet)& for \quad n\geq m \geq 0 ,\\
-\P_{n,m}(\tet)(Id-P_{m}(\tet))& for \quad 0\leq n < m.
\end{cases}$$
\end{definition}
Then, it follows directly from the definition \ref{def 2.2} that 
$\|G_{n,m}(\tet)\|\leq \mt\at^{|n-m|}\quad \forall n;m\in\hro{N}$.

The following lemma plays an important role in our strategy.
\begin{lemma}\label{lem 2.8}
Let \quad $ \Pi^{*} :\E\ti\hro{N}\rar\E$ be a discrete skew product  which has pointwise discrete dichotomy over $ \Tet$.  Consider $y=\{y_n\} \in\l$ and let $ x=\{x_n\}$ be a sequence with each element $x_n$ belonging to $X$. Then, for each $\tet\in\Tet$ the following assertions are equivalent.
\begin{enumerate} 
\item $x$ and $y$ satisfy  $x_{n+1}=\P_n(\tet)x_n+y_n\quad  \forall n\in\hro{N}$, and $x\in \l$.
\item $x$ and $y$ satisfy  ${x_n=\P_{n,0}(\tet)P_0(\tet)x_0+\sum\limits^{\ift}_{k=0}G_{n,k+1}(\tet)y_k}.$
\end{enumerate}
\begin{proof} 
$(i)\Rar(ii)$:   By lemma 2.7 we have:
$$x_r=\P_{r,n}(\tet)x_n+\sum _{k=n}^{r-1}\P_{r,k+1}(\tet)y_k\quad r>n\geq 0$$
Then     $\P_{n,r}(\tet)[I-P_r(\tet)]x_r=\P_{n,r}(\tet)x_r-\P_{n,r}(\tet)P_r(\tet)x_r$\\
$=\P_{n,r}(\tet)[\P_{r,n}(\tet)x_n+\sum _{k=n}^{r-1}\P_{r,k+1}(\tet)y_k]-\P_{n,r}(\tet)P_r(\tet)[\P_{r,n}(\tet)x_n+\sum _{k=n}^{r-1}\P_{r,k+1}(\tet)y_k]$\\
$ =x_n+\sum^{r-1}_{k=n}\P_{n,k+1}(\tet)y_k-\P_{n,r}(\tet)P_{r}(\tet)\P_{r,n}(\tet)x_n-\sum _{k=n}^{r-1}\P_{n,r}(\tet)P_{r}(\tet)\P_{r,k+1}(\tet)y_k$.\\
By (iii) in  remark 2.6 we have:\\
\begin{align*}
\P_{n,r}(\tet)[I-P_r(\tet)]x_r =&x_n+\sum^{r-1}_{k=n}\P_{n,k+1}(\tet)y_k-\P_{n,r}(\tet)\P_{r,n}(\tet)P_{n}(\tet)x_n\\
                                                   &\quad-\sum _{k=n}^{r-1}\P_{n,r}(\tet)\P_{r,k+1}(\tet)P_{k+1}(\tet)y_k\\
                                                   =&x_n+\sum^{r-1}_{k=n}\P_{n,k+1}(\tet)y_k-P_{n}(\tet)x_n-\sum _{k=n}^{r-1}\P_{n,k+1}(\tet)P_{k+1}(\tet)y_k .
\end{align*}
Hence\quad  $-G_{n,r}(\tet)x_r=[I-P_n(\tet)]x_n-\sum^{r-1}_{k=n}G_{n,k+1}(\tet)y_k$.\\
Since $\| G_{n,r}(\tet)x_r \|\leq \mt\at^{|r-n|}\|x_r\|$ and $x=\{x_n\}\in l_\ift(\hro{N};X)$, 
 we obtain that
 $ \lim\limits_{r\rar \ift}\|  G_{n,r}(\tet)x_r\| \rar \ift$
     and  the  series\quad  $ \sum^{r-1}_{k=n}G_{n,k+1}(\tet)y_k$  converges.
Therefore,
\begin{equation}\label{2.9.1}
 [I-P_n(\tet)]x_n=\sum^{r-1}_{k=n}G_{n,k+1}(\tet)y_k.
\end{equation}

\non On other hand\quad $x_n=\P_{n,0}(\tet)x_0+\sum _{k=0}^{n-1}\P_{n,k+1}(\tet)y_k$.\\
Then \hskip 1.5cm $P_n(\tet)x_n=\P_{n,0}(\tet)P_0(\tet)x_0+\sum _{k=0}^{n-1}\P_{n,k+1}(\tet)P_{k+1}(\tet)y_k$. \\
Thus
\begin{equation}\label{2.9.2}
 P_n(\tet)x_n=\P_{n,0}(\tet)P_0(\tet)x_0+\sum _{k=0}^{n-1}G_{n,k+1}(\tet)y_k  .
\end{equation}
From \eqref{2.9.1} and \eqref{2.9.2} we get $\quad x_n=\P_{n,0}(\tet)P_0(\tet)x_0+\sum^{\ift}_{k=0}G_{n,k+1}(\tet)y_k$.\\

$(ii)\Rar(i)$: Using remark \ref{Remark 2.5}, one can easily see that if the sequences  $x=\{x_n\}$ and $y=\{y_n\}$  satisfy
$\quad x_n=\P_{n,0}(\tet)P_0(\tet)x_0+\sum^{\ift}_{k=0}G_{n,k+1}(\tet)y_k $  
 then   $\quad x_{n+1}=\P_n(\tet)x_n+y_n\quad \forall n\in\hro{N}\quad.$

To complete this assertion  we shall prove $x\in l_\ift (\hro{N};X)\quad.$
By definition 2.8 we have that  $ \|x_n\|\leq \mt\at^n\|x_0\|+\sum^{\ift}_{k=0}\mt\at^{|n-k-1|}\|y\|$. Hence,
$\|x_n\|\leq \mt\at^n\|x_0\|+\mt\|y\|\frac{1+\at}{1-\at}$ finishing the proof.
\begin{remark}\label{rm 293}
Under the hypothesis of lemma \ref{lem 2.8}, let $x, y\in \l$ satisfy one of the equivalent assertions of this lemma, and let $x_0\in \ker P_0(\theta)$. Then, by the above proof one can easily see that
\begin{equation}\label{2.9.3}
\|x\|\leq \frac{M_{\tet}(1+\at)}{1-\at}\|y\|
 \end{equation}
for $M_\theta$ and $\at$ being the dichotomy constants of discrete LSPS $\Pi^*$.
\end{remark}
\end{proof}
\end{lemma}

%%%%%%%%%%%%%%%% Section2
%%%%%%%%%%%%%%%%%%%%%%%%%%%%%%%%%%%%%%%
 \section{DISCRETE DICHOTOMY OF SKEW-PRODUCT ON THE HALF LINE} \label{section 3}

In this section we shall give the necessary and sufficient conditions for discrete skew product to have a pointwise discrete dichotomy and uniform discrete dichotomy. We begin with the definitions of some difference operators which are key tools in our strategy.


For each $\tet\in\Tet$  we define the operator $T_{\tet}:\l\rar \l$   by
  $$T_{\tet}(x)_n=x_{n+1}-\P_n(\tet)x_n
\hbox{ for }x=\{x_n\}  \in \l. $$
We can see that $\|x_{n+1}-\P_{n}(\tet)x_n\|\leq(1+\ro)\|x\|_{l_{\ift}}$. 
Therefore, we obtain that $T_{\tet}$ is a bounded linear operator from $l_{\ift}$ to $l_\infty$.

We denote the restriction of $ T_{\tet}$ on $\lo$ by $T_{0\tet}$, i.e., $T_{0\tet}:\lo\to\l$ and $T_{0\tet}x=T_{\tet}x$ \\for $x\in \lo$. We also have that $T_{0\tet}$ is a bounded linear operator from $\lo$ to $l_\infty$.\\
Moreover,\quad since $ \ker T_{\tet}={\{}\,u=\{u_n\}\in\l:u_n=\P_{n,0}(\tet)u_0\}$ it follows that $T_{0\tet}$ is injective.\\



To estimate the growth and decay of the solutions we need the following lemmas which are taken from \cite{huyminh}.
\begin{lemma}\label{lem 3.1}
Let $\{\chi_n\}_{n_1>n\geq n_0}$ be positive real numbers and let $c>1$ and\quad $K,\a>0$ be constants such that $\chi_n\leq Ke^{\a(n-n_0)} $\quad and\quad $\sum^{n}_{k=n_0}\chi_n\chi_k^{-1}\leq c$  with $n_0\leq n < n_1$.\\ Then exist $N,\nu$ dependent only on $K,c,\a$ such that $\chi_n\leq Ne^{-\nu(n-n_0)}$ for all $n_0\leq n<n_1$.\\
\end{lemma}
\begin{lemma}\label{lem 3.2}
Let $\{\chi_n\}_{n\in\hro{N}} $ be a sequence of positive real numbers. Assume that there is a constant $c>1$\quad such that $\sum^{n}_{k=m}\chi_m\chi_k^{-1}\leq c$\quad $ \forall n\geq m\geq 0$.\\
\non Then exist $N,\nu$ dependent only on $c$ such that\quad $\chi_n\geq Ne^{\nu(n-m)}\chi_m\quad \forall n\geq m\geq 0.$
\end{lemma}

Recall that for an operator $B$ on a Banach space Y the approximation point spectrum $\A$ of B  is the set of all complex number $\ld$ such that for every $\Ep>0$ there exists $y\in D(B)$ with $\|y\|=1$   and  $\|(\ld-B)y\|\leq\Ep$.

To characterize the stability and dichotomy of a discrete skew-product $\Pi^*=(\Phi_n,\varphi^n)$ we also need the following notion of stable subspaces $X_{0}(n_0,\tet)$ of $X$
which are defined by 
$$X_{0}(n_0,\tet):={\{}\ x\in X:\sup\|\P_{n,n_0}(\tet)x\|< \ift\}\hbox{ for each }
(n_0,\theta)\in {\mathbb N}\times\Theta.
$$

The following theorem connects the spectral properties of $T_{0\theta}$ to the exponential stability of discrete bounded orbits.
\begin{theorem}\label{the 3.3}
 Let the operator $T_{0\tet}$ define as above and $0\notin \AT$ then every discrete bounded orbits of the family  $\P(\tet)={\big\{}\P_{n,m}(\tet)\}_{n\geq m\geq 0}$  is exponential stable.
 
Precisely, if $\sup_{n_0\le n\in\n}\{\|\P_{n,n_0}(\tet)x\|\}<\infty $ with $x\in X$ and $n_0\ge 0$, then there exist positive constants\quad $N_{\tet};\nu_{\tet}<1$\quad independent of $n_0$ and $x$  such that:
    $$\|\P_{n,n_0}(\tet)x\|\leq N_{\tet}\nu_{\tet}^{n-s} \|\P_{s,n_0}(\tet)x\|;\quad n\geq s\geq n_0$$
\end{theorem}
\begin{proof} Firstly, we shall prove for the case $s=n_0$.\\
Since  $0\notin \AT$ there exists $\de_{\tet}>0$ such  that\quad $\|T_{0\tet}v\|\geq \de_{\tet} \|v\|\quad \forall v\in\l$.\\
 Replacing $\de_{\tet}$ by a smaller one, if necessary, we can suppose that   $\de_{\tet}<1$. 
Consider $0\not=x\in X$. Without loss of generality we can let $\|x\|=1$. Put  $u_n=\P_{n,n_0}x;\quad n\geq n_0$;\quad $n_1=\sup\{n\geq n_0:\P_{n,n_0}(\tet)x\neq 0\}$.\\
 For any  natural number $n_2$ satisfying  $n_0\leq n_2\leq n_1$ we take\\
$$v=\{v_n\}\quad\mbox{with}\quad v_n= 
\begin{cases}
0 &0\leqslant n<n_0,\\
u_n\sum_{k=n_0}^{n} \frac{1}{\|u_k\|}  &n_0\leqslant n\leqslant n_2,\\
u_n\sum_{k=n_0}^{n_2} \frac{1}{\|u_k\|} &n\geq n_2;
\end{cases}
$$
$$f=\{f_n\}\quad\mbox{with}\quad f_n= 
\begin{cases}
0 &0\leqslant n<n_0-1,\\
 \frac{u_{n+1}}{\|u_{n+1}\|}  &n_0-1\leqslant n< n_2,\\
0 &n\geq n_2.
\end{cases}
$$
Then we have $v_{n+1}=\P_n(\tet)v_n+f_n\quad \forall n\geq 0;v\in\lo\quad \mbox{and}\quad f \in \l $.\\
It follows that\quad $T_{0\tet}v=f \quad \mbox{and}\quad \|f\|\geq\de_{\tet} \|v\|$. That means that\\
 $1=\|f\|\geq \de_{\tet}\sup_{n\geq n_0}\|u_n\|\sum^{n}_{k=n_0} \frac{1}{\|u_k\|},\mbox{ and hence, }\sum^{n}_{k=n_0} \frac{1}{\|u_k\|} \leq \frac{1}{\de_{\tet}} $  for all  $n\geq n_0$. \\
Moreover, from definition 2.1 (i) we get $$\|u_n\|=\|\P_{n,n_0}(\tet)x\|\leq e^{(n-n_0)ln{(\ro)}}\|x\|= e^{(n-n_0)ln{\ro}}.$$
Lemma 3.1 yields the existences of the constants $N_{\tet}=\frac{1}{\de_{\tet}};\a_{\tet}={ln{\Big(} \frac{1}{1-\de_{\tet}}{ \Big)}}$  depending only on $\tet$\quad such that \qquad $\|u_n\|\leq N_{\tet}e^{-\a_{\tet}(n-n_0)}$.\\
Now, we fix $s\geq n_0$,\quad set\quad  $y:=\P_{s,n_0}x\quad \mbox{then}\quad \sup_{n\geq s}\|\P_{n,s}(\tet)y\|<\ift$\quad\mbox{and}
$$\|\P_{n,n_0}x\|=\|\P_{n,s}y\|\leq N_{\tet}e^{-\a_{\tet}(n-s)}\|\P_{s,n_0}x\|\quad\forall n\geq s\geq 0.$$
\end{proof}
From this theorem we obtain the following corollary properties of stable subspace $X_{0}(n_0,\tet)$.
\begin{corollary}\label{cor 3.4}
Under the conditions of  Theorem {\ref{the 3.3}} we have 
$$X_{0}(n_0,\tet)={\{}\ x\in X:\|\P_{n,n_0}(\tet)x\|\leq N_{\tet}e^{-\a_{\tet}(n-n_0)}\|x\|: n\geq n_0 \}$$
for certain positive constants $N_{\tet};\a_{\tet}$. 
Therefore,  $X_{0}(n_0,\tet)$ is a closed subspace of X.
\end{corollary}

We now come to our first main result. It characterizes the pointwise dichotomy of a discrete skew-product via properties of the operators $T_{\tet}$ and the subspaces $ X_{0}(0, \tet)$ of X.
\begin{theorem}\label{the 3.5}
 For the  discrete skew product  $ \Pi^{*}$     the following assertions are equivalent.
\begin{enumerate}
\item[(i)]$\Pi^{*}$   has   a pointwise discrete dichotomy    over     $\Tet$.
\item[(ii)]For each  $\tet\in\Tet$  the linear operator  $T_{\tet}$  is surjective and the stable subspace $X_{0}(0, \tet)$ is complemented in X.
\end{enumerate}
\end{theorem}
\begin{proof}
$\bf{(i)\Rar (ii)}$ Let   $\Pi^{*}$  has a pointwise discrete dichotomy over $\Tet$ 
with the  corresponding family of projections $(P_n(\tet))_{(n,\theta)\in \hro{N}\times\Theta}$.

For $f\in\l$ we define $v=(v_n)\in\l$as follow:
$$v_n= 
\begin{cases}
\sum^{n}_{k=1}\P_{n,k}(\tet)P_{k}(\tet)f_{k-1}-\sum^{\ift}_{k=n+1}\P_{k,n}^{-1}(\tet)(I-P_k(\tet))f_{k-1} & n\geq 1,\\
 -\sum^{\ift}_{k=1}\P_{0,k}(\tet)(I-P_k(\tet))f_{k-1}  &n=0.
\end{cases}
$$
Then we have $v_{n+1}=\P_n(\tet)v_n+f_n$ and $v\in\l.$\\ So\quad $T_{\tet}v=f$ ,\quad hence \quad $T_{\tet}$ is surjective.

We now prove that$X_0(0, \tet)=P_0(\tet)X$. It is clear that  $ P_0(\tet)X \subset X_0(0, \tet)$.\\
Conversely, take $x\notin P_0(\tet)X$ then $(I-P_0(\tet))x\neq 0$ and we get the inequality 
$$\|\P_{n,0}(\tet)x\|\geq\|\P_{n,0}(\tet)[I-P_0(\tet)]x\|-\|\P_{n,0}(\tet)P_0(\tet)x\|\geq [M_{\tet}^{-1}\a_{\tet}^{-n} -M_{\tet}\a_{\tet}^{n}]\|x\|.$$
Since $0<\a_{\tet}<1$ then $\|\P_{n,0}(\tet)x\|\rar +\ift$ as $ n\rar+\ift$.\\
It follows that $x\notin X_0(0, \tet)$. So, we have $X_0(0, \tet)\subset P_0(\tet)X$. Therefore, $X_0(0, \tet)=P_0(\tet)X$. This yields that $X_0(0, \tet)$   
is complemented in $X$.


$\bf{(ii)\Rar (i)}:$  We  prove this in several steps.

$\bf{A)}$ Let $Z(\tet)\subset X$ be a complement of $X_0(0, \tet)$,i.e.,\quad $Z(\tet) \oplus X_0(0, \tet)=X$\\
For $n\ge 0$ set\quad $X_1(n, \tet)=\P_{n,0}(\tet)Z(\tet).$ \qquad Then 
 $$\P_{n,s}(\tet)X_{0}(s, \tet)\subseteq X_0(n, \tet);\qquad\P_{n,s}(\tet)X_1(s, \tet)=X_1(n, \tet)\quad(n\geq s\geq 0)$$

$\bf{B)}$ We shall prove that there exist constants $N_{\tet};\a_{\tet}>0$\quad such that 
    $$\|\P_{n,m}(\tet)y\|\geq N_{\tet}e^{v_{\tet}(n-m)}\|y\| \quad \mbox{for}\quad n\geq s \geq 0;y\in X_1(m, \tet).$$
In fact ,let $Y(\tet):={\{}\ v=\{v_n\}_{n\in \hro{N}}\in \l :v_0\in Z(\tet)\}$ endowed with the  $l_\ift$-norm. Then $Y(\tet)$ is a closed subspace of the Banach space $l_{\ift}$ and hence $Y(\tet)$ is complete.\\
Since $X=X_1(0, \tet)\oplus X_0(0, \tet)$ and  $T_{\tet}$   is surjective, we obtain
$T_{\tet}\mid _{Y(\tet)}:Y(\tet)\rar \l$ is bijective so it is an isomorphism. Thus there is a constant $\de_{\tet}>0$ such that 
\begin{equation}\label {(3.5.1)} 
\|T_{\tet}v\|_{\l}\geq \de_{\tet}\|v\|_{\l}\quad \mbox{for all}\quad v\in Y(\tet)
\end{equation}
 For $x\ne 0;x\in X_1(0, \tet)$ set $u_n=\P_{n,0}(\tet)x;n\geq 0$. Then , we have 
$u_n \ne 0$    for every   $n\geq 0$\\
(because if there exists $u_k=0(k\geq 0)$ then $x\in X_0(0, \tet)$  which is a contradiction since $x\ne 0; x\in X_1(0, \tet)$)\\
For a natural large number $\xi$ we take $v=\{v_n\}$ and $f=\{f_n\}$ where:\\
$$v_n= 
\begin{cases}
u_n\sum^{\xi}_{k=n+1}\frac{1}{\|u_k\|} &0\leqslant n<\xi,\\
0  &n\geq \xi;
\end{cases}
$$
$$f_n= 
\begin{cases}
 \frac{-u_{n+1}}{\|u_{n+1}\|}&0\leqslant n<\xi, \\
 0 &n\geq \xi.
\end{cases}
$$
Then  $v\in Y(\tet);f\in\l$  which satisfy the equation $v_{n+1}=\P_{n} v_{n}+f_{n}\quad \forall n\in\hro{N}$.\\
It follows that $T_{\tet}v=f\Rar\|f\|\geq \de_{\tet}\|v\|.$ That means
$$ 1\geq \de_{\tet}\|u_n\|\sum^{\xi}_{k=n+1}\frac{1}{\|u_k\|}\hbox{ hence,   }\|u_n\|\sum^{\xi}_{k=n+1}\frac{1}{\|u_k\|}\geq \frac{1}{\de_{\tet}}+1  \quad .$$
From lemma 3.2 we have that $\|u_n\|\geq N_{\tet}e^{v_{\tet}(n-m)}\|u_m\|; \quad n\geq m \geq 0$, where
$$N_{\tet}=\frac{\de_{\tet}}{1+\de_{\tet}};v_{\tet}=ln(1+\de_{\tet}).$$
Moreover,\quad since \quad $\P_{m,0}(\tet)X_1(0, \tet)=X_1(m, \tet)$,\quad we have $$\|\P_{n,m}(\tet)y\|\geq N_{\tet}e^{\nu_{\tet}(n-m)}\|y\|\hskip 1 cm\forall y\in X_1(m, \tet),n\geq m\geq 0.$$

$\bf{C)}$  We prove     $X=X_0(n,\theta)\oplus X_1(n, \tet):n\in \hro{N}$\hfill.\\
Let $Y(\tet)\con\l$as in \quad$\bf{B)}$.\quad Then  $\lo\con Y(\tet)$ so we have $\|T_{0\tet}v\|\geq \de_{\tet}\|v\|$ for $v\in\lo$.\\
Thus $0 \notin \AT$  and corollary 3.4 implies that $ X_0(n,\theta) $ is  a closed subspace of X.\\
Moreover $ X_0(n,\theta)\cap  X_1(n, \tet)={\{}\ 0 \} \quad \forall n\in \hro{N}$\quad by parts $\bf{A)}$ and $\bf{B)}$ \quad .\\
Finally,  fix  $n_0>0$  and  $x\in X $; for large natural number $n_1$ set\\
 $$v=\{v_n\} \quad \mbox{with}\quad v_n= 
\begin{cases}
 (n-n_0+1)\P_{n,n_0}(\tet)x  &n_0\leqslant n\leq n_1, \\
 0 &n>n_1.
\end{cases}
$$
$$f=\{f_n\} \quad \mbox{with}\quad f_n= 
\begin{cases}
 \P_{n+1,n_0}(\tet)x&n_0\leqslant n<n_1 ,\\
 -(n_1+1-n_0)\P_{n+1,n_0}(\tet)x &n=n_1,\\
0&n>n_1.
\end{cases}
$$
Thus $ v_{n+1}=\P_{n}(\tet) v_{n}+f_{n}:n\geq n_0>0 $ and $v\in l_{\ift}([n_0;\ift);X). $\quad Set  $f_n=0$  for  $0\leq n_0<n$. Then\quad $f\in\l .$\quad By assumption there exists  $\om\in\l $\quad such that\quad $ T_{0\tet}\om=f$.\\
By the definition of $T_{\tet}$ then  $v_n-\om_{n}=\P_{n,n_0}(\tet)(v_{n_0}-\om_{n_0})=\P_{n,n_0}(\tet)(x-\om_{n_0})$  for  $n\geq n_0$ . \\
It follows that  $x-\om_{n_0}\in X_0(\tet)(n_0).$\quad Since     $f_n=0$   for $0\leq n<n_0$ then $\om_{n_0}=\P_{n_0;0}(\tet)\om_0$.\\ Writing $\om_0$ in the form $\om_0+\om_1;$ for \quad $\om_{k}\in X_k(0)\quad(k=0;1)$  we have 
$$\om_{n_0}=\P_{n_0;0}(\tet)\om_0^0+\P_{n_0;0}(\tet)\om_0^1\in  X_0(\tet) (n_0) +X_1(\tet)(\tet)(n_0).$$
Therefore  $x=[x-\om_{n_0}+\P_{n_0;0}(\tet)\om_0^0]+\P_{n_0;0}(\tet)\om_0^1\in X_0(\tet)(n_0) +X_1(\tet)(n_0)$ .\\
Since $ X_0(n_0, \tet)\cap  X_1(n_0, \tet)=\{ 0 \}$ we obtain that  $X=X_0(n_0, \tet)\oplus X_1(n_0,\tet)$.\\
$\bf{D)}$  Let $P_n(\tet)$ be the projection from X on to $X_0(n,\theta)$ with  kernel $X_1(n, \tet) $ then from  part $\bf{(A)}$ we have $$P_{n+1}(\tet)\P_n(\tet)=\P_n(\tet)P_{n}(\tet) .$$
From part $\bf{(B)}$  and corollary (3.4) we have:
\begin{eqnarray*}
\|\P_{n,m}(\tet)x\|  &\leq &\frac{1+\de_{\tet}}{\de_{\tet}}e^{[ln(1+\de_{\tet})](n-m)}\|x\|\hskip0.6cm \mbox{for}\quad x\in KerP_{n}(\tet);n<m ,\\
\|\P_{n,m}(\tet)x\|  &\leq& \frac{1}{\de_{\tet}}{e^{(n-m)ln(1-\de_{\tet})}}\|x\| \hskip1.5cm\mbox{for}\quad x\in P_n(\tet)X;n\geq m .
\end{eqnarray*}

Thus       $\Pi^{*}$   has an pointwise discrete dichotomy with the dichotomy constants 
 $\a_{\tet}=1-\de_{\tet}<1$ and $M_{\tet}=\frac{1+\de_{\tet}}{\de_{\tet}}$.  
\end{proof}

To characterize a discrete skew-product having  uniform discrete dichotomy we need the following notion of uniform correctness of a family of operators. 
\begin{definition}\label{def 3.6}
The family ${\{A_{\tet}\}}_{\tet\in\Tet}$ is called uniformly correct if there exists a positive constant
$\nu$ independent on $\tet$ such that $\|A_{\tet}x\|\geq \nu\|x\|$ for all $\tet\in\Tet$  and  all  $x\in D(A)$.
\end{definition}


\begin{theorem}\label{the 3.7}
For a  discrete skew product $\Pi^{*}$ the following assertions are equivalent:
\begin{enumerate}
\item[(i)]$\Pi^{*}$   has an uniform discrete dichotomy
\item[(ii)]For each $\tet\in\Tet$ we have that $T_{\tet}$  is surjective and $X_0(0, \tet)$ has a complement $Z(\tet)$  in $X$. Moreover, setting  $Y(\tet):=\{\{v_n\}_{n\in\hro{N}}\in \l:v_0\in  Z(\tet) \}$  and  $\{T_{Y\tet}\}$  being the restriction of  $\{T_{\theta}\}$  on $Y(\tet)$,  then  the family ${\{}\ T_{Y\tet}\}_{\tet\in\Tet}$ is uniformly correct.
\end{enumerate}
\end{theorem}
\begin{proof}$\bf{(ii)\Rar (i)}:$ From the part $\bf{D)}$  in theorem 3.5 we have that $N_{\tet};v_{\tet}$   depend only on $\tet$. By the same technique which is given in \cite[Lemma 3.1]{huyminh} we obtain that 
$M_{\tet}=\sup\{\, \|P_{n}(\tet)\|:n\geq 0 \}$ is finite and depends only on $\tet$.
Since the family ${\{}\ T_{0\tet}\}_{\tet\in\Tet}$ is uniformly correct, there exists a positive constant
$\de$ independent of $\tet$ such that $\|T_{0\tet}\|\geq \de$ for all $\tet\in\Tet$.
Therefore, the constant  $\de_{\tet}$  in inequality \eqref{(3.5.1)} can be replaced by $\de$  for  all  $\tet\in\Tet$.\\
That means there  exist $N;\nu >0$ independent of $\tet$ such that:
\begin{eqnarray*}
\| \P_{n,m}(\tet)P_{m}(\tet)\|        &\leq Ne ^{-\nu(n-m)}\qquad &n \geq m \geq 0 ,\\
\| \P_{n,m}(\tet)(I-P_{m}(\tet))\|   &\leq Ne^{\nu(n-m)}\qquad &0\leq n< m.
\end{eqnarray*} 
Then it follows that $\Pi^{*}$ has \udd.\\
$\bf{(i)\Rar (ii)}$: Let  $\Pi^{*}$ has uniform discrete dichotomy with the associated constants $N;\a=e^{-\nu}<1$.\\
 Due to theorem 3.5 we need only to prove  that the family $\{\ T_{Y\tet}\}_{\tet\in\Tet}$ is uniformly correct.
By the proof of theorem 3.5 (implication (ii)$\Rar$(i)) we obtain that \quad $T_{Y\tet}$\quad is an 
isomorphism.\\
For $y\in\l$  take  $x=T^{-1}_{Y\tet}y$  then, by the definition of $T_{Y\tet}$  and lemma 2.9 we have that $x_n=\P_{n,0}(\tet) P_0(\tet)x_0+\sum^{\ift}_{k=0}G_{n,k+1}(\tet)y_k$.\\
Since $x_0\in X_1(0, \tet)$ we obtain that $P_0(\tet)x_0=0$, and hence,  \\
$ \|T^{-1}_{Y \tet}y\| \leq \sum_{k=0}^{n-1}Ne^{-\nu (n-k-1)}\|y\|+\sum^{\ift}_{k=n}Ne^{\nu (n-k-1)}\|y\|$\\
$ =\|T^{-1}_{Y\tet}y\|\leq N\|y\|e^{-\nu n}\frac{e^{\nu n}-e^{\nu}}{e^{\nu}-1}
+N\|y\|\frac{e^{-\nu }}{1-e^{-\nu}}\leq 2N\frac{\|y\|}{\nu}.$ Therefore, $ \|T_{Y\tet}\|\geq \frac{\nu}{2N}>0.$\\
\end{proof}
From this theorem we obtain the following corollary.
\begin{corollary}\label{cor 3.8}
If the discrete dichotomy $\Pi^{*}$ satisfy $\|\P_n(\tet)\|\geq\de>1\quad(\mbox{or}\quad \|\P_n(\tet)\|\leq\de<1)$\quad   for  all    $(n,\tet)\in(\hro{N}\times\Tet)$\quad  then the following assertions are equivalent.
\begin{enumerate}
\item[(i)] $\Pi^{*}$   has an uniform discrete dichotomy.
\item[(ii)] For each $\tet\in\Tet$ the operator $ T_{\tet}$  is surjective and $X_{0\tet}$ is complemented in $X$.
\end{enumerate}
\end{corollary}


\begin{definition}\label{def 3.11} Let $(Z(\tet))_{\tet\in\Tet}$ be  a  family  of  closed subspaces of  $X$. We define $\lz:={\Big\{}\,x\in\l:x_0\in Z(\tet) \Big\}$ and $T_{Z(\tet)}:=T_{\tet}\mid_{\lz}$.
\end{definition}

\begin{theorem}\label{the 3.12} Let $\Pi^{*}$ be a  discrete skew product and $(Z(\tet))_{\tet\in\Tet}$ be the family of closed subspaces of $X$. Then, the following assertions are equivalent.  
\begin{enumerate}
\item[(i)]  $\Pi^{*}$   has an uniform discrete dichotomy with the corresponding projections  $\{P_n(\tet)\}$  satisfying  $\ker P_0(\tet)=Z(\tet).$
\item[(ii)]  For each $\tet\in\Tet$\quad  the operator \quad $T_{Z(\tet)}$  is invertible
 and the family $\{\ T_{Z(\tet)}\}_{\tet\in\Tet}$ is uniformly correct.
\end{enumerate}
\end{theorem}
\begin{proof}${(i)\Rar (ii)}$: 
We first prove that $T_{Z(\tet)}$ is invertible. Indeed, for $f\in\l$, since $T_{\tet}$ is surjective (by theorem 3.5),   there exists $x\in\l$ such that $T_{\tet}x=f$.
Let $u\in\l$ be defined as follow $u_n=\P_{n,0}(\tet)P_{0}(\tet)x_0$; $n\in \n$. Then $T_{\tet}(u)=0$
because $\ker T_{\tet}=\{\,u\in\l: u_n=\P_{n,0}u_0\Big\}$.
Note that $x-u\in\lz$  and $T_{Z(\tet)}(x-u)=T_{\tet}(x-u)=T_{\tet}x=f$ then\quad $T_{Z(\tet)}$  \quad is surjective.
Moreover, $T_{Z(\tet)}$ is injective  since\\
 $\ker T_{Z(\tet)}={\big\{}\,u\in\l:u_n=\P_{n,0}u_0\hbox{ for }u_0\in P_{0}(\tet)X \big\}\cap{\big\{}\,u\in\l:u_0\in Z(\theta)=\ker P_0(\tet)\big\}=\{0\}$.
Thus, $T_{Z(\tet)}$ is invertible.

To complete this part we shall prove that the family ${\{}\ T_{Z(\tet)}\}_{\tet\in\Tet}$ is uniformly correct. In fact,
since   $\Pi^{*}$   has an uniform discrete dichotomy, we have that the corresponding dichotomy constants $M, \a$ do not depend on $\theta$.
By  remark \ref{rm 293} we obtain that  $\|T_{Z(\tet)}\|\geq  \frac{1-\a}{M(1+\a)}\quad\forall \tet\in\Tet$. Therefore,
the family ${\{}\ T_{Z(\tet)}\}_{\tet\in\Tet}$ is uniformly correct.\\

${(ii)\Rar (i)}$:  Since $T_{Z(\tet)}$ is invertible  and  the family ${\{}\ T_{Z(\tet)}\}_{\tet\in\Tet}$ is uniformly correct, it follows that $T_{\tet}$ is surjective, and there exists $\de>0$   such that   $\|T_{Z(\tet)}(x)\|\geq \de\|x\| \quad\forall  x\in\lz$. \\
We note that $T_{0\tet}$ is the restriction of $T_{Z(\tet)}$ on $l^0_\infty$. Therefore, $0\notin\AT$.
By  corollary 3.4 $X_0(0, \tet)$ is a closed subspace of X.\\
 We now prove that  $X=X_0(0, \tet)\oplus Z(\tet)$.\\
For any $x\in X$ let $u=\{u_n\}$ and $f=\{f_n\}$ be defined by $u_0=x;\ u_n=0$ for $ n> 0$ and $f_0=-\P_0(\tet)x;\ f_n=0$ for $n> 0$, respectively. Then, $u, f\in\l$ and $T_{\tet}u=f$.
Since $T_{Z(\tet)}$ is invertible then there exists $v=\{v_n\}\in\l$ such that $T_{\tet}(v)=T_{Z(\tet)}(v)=f$. Hence,
$u-v\in \ker T_{\tet}.$ Therefore, $u_0-v_0=x-v_0\in X_0(0, \tet)$ and $v_0\in Z(\tet)$ then\quad $x\in X_0(0, \tet)+ Z(\tet)$.
If now  $y\in X_0(0, \tet)\cap Z(\tet)$\quad then the sequence \quad $\om=\{\om_n\}$\quad defined by
$\om_n=\P_{n,0}(\tet)y;\quad n\in\hro{N}$ belongs to \quad $l_{\ift}^{Z(\tet)}\cap \ker T(\tet).
$\quad Hence, \quad $T_{Z(\tet)}\om=0$.   This implies that \quad$\om=0$. That yields $X=X_{0}(0, \tet)\oplus Z(\tet)$ ; so $X_{0}(0, \tet)$ is complemented in X. The assertion is now followed from theorem 3.7.
\end{proof}
%%%%%%%%%%%%%%%% Section2
%%%%%%%%%%%%%%%%%%%%%%%%%%%%%%%%%%%%%%%
 \section{APPLICATIONS TO LINEAR SKEW PRODUCT SEMIFLOWS} \label{section 4}
\smallskip 
In this part we give the necessary and sufficient conditions for linear skew-product semiflows to have
exponential dichotomy.
 The equivalence between uniform  discrete dichotomy and exponential dichotomy will be given in the following theorem whose proof can be done in the same way as in \cite[Theorem 4.1]{cholei}.
\begin{theorem}\label{the 4.1} 
Assume that $\Pi=(\P,\p)$ is a linear skew-product semiflows on $\E=X\ti\Tet$. Then,
the following assertions are equivalent.
\begin{enumerate}
\item[(i)]    The discretized skew-product $\Pi^{*}$ given as follow $\Pi^{*}(x;\tet;n)=(\P(\p^n\tet;1)x;\p^{n}(\tet))$ has a uniform discrete dichotomy.
 \item[(ii)]  $\Pi$ has  an exponential dichotomy.
\end{enumerate}
\end{theorem} 
\non By theorem 3.10 and 4.1 we obtain the following characterization of a linear skew product semiflow having an exponential dichotomy.
\begin{theorem}\label{the 4.2}
Let $\Pi=(\Phi,\varphi)$ be a linear skew product semiflow  on  $\E$ and $(Z(\tet))_{\tet\in\Tet}$ be a family of closed subspaces of $X$. Then, the following assertions are equivalent.
\begin{enumerate}
\item[ a)]  $\Pi$      has  an    exponential    dichotomy      with      the       corresponding       projector ${\bf P}$ which, by definition, has the form ${\bf P}(x,\theta)= (P(\theta)x,\theta)$ for $(x,\theta)\in X\times\Theta $ and  $P(\theta)$ being projections on $X$  satisfying  $\ker P(\tet)=Z(\tet)$.
\item[b)] For each  $\tet\in\Tet$ the operator  $T_{Z(\tet)} : \lz\to \l$  defined by $$(T_{Z(\tet)}x)_n=x_{n+1}-\Phi(\varphi^n\theta,1)x_n \hbox{ for }x\in \lz$$ is invertible.
  Moreover, the family  ${\{}\,{T_{Z(\tet)}}\}_{\tet\in\Tet}$ is uniformly correct.
\end{enumerate}
\end{theorem}
\smallskip
   To finish this article we shall prove exponential dichotomy is not affected by  small pertubations.
\begin{theorem}\label{the 4.3}
    Assume that  $\Pi=(\P,\p)$ is a linear skew product semiflows on  $\E$  having an  exponential dichotomy with dichotomy  $M$ and $\a$.  Then, there exists
  $\de=\de(N;\a)>0$  small enough such that any linear skew product semiflows 
 $\Ga=(\Psi,\p)$ on $\E$ satisfying
 $\sup{\{}\,\|\P(\tet;t)-\Psi(\tet;t)\| : 0\leq t\leq 1;\tet\in\Tet\}\leq \de$
has an exponential  dichotomy.
\end{theorem}
\begin{proof} 
By assumption, $\Pi=(\P,\p)$   has  exponential dichotomy. Let ${\bf P}$ be the projector corresponding the   exponential dichotomy of $\Pi=(\P,\p)$. By definition,  ${\bf P}$ has the form ${\bf P}(x,\theta)= (P(\theta)x,\theta)$ for $(x,\theta)\in X\times\Theta $ and  $P(\theta)$ being projections on $X$. Putting $\ker P(\tet)=Z(\tet);\ \theta\in \Theta$, by theorem \ref{the 4.2} we have that the operator $T_{Z(\tet)} : \lz\to \l$ defined by
$$(T_{Z(\tet)}x)_n=x_{n+1}-\Phi(\varphi^n\theta,1)x_n \hbox{ for }x\in \lz$$
  is invertible, also the family 
$(T_{Z(\tet)})_{\theta\in\Theta}$ is uniformly correct. Let now $T_{Z(\tet)}^\Psi$ be the operators corresponding to the discretized skew-product $\Gamma^*$ of LSPS $\Gamma$. That means that $T_{Z(\tet)}^\Psi : \lz\to\l $ is defined by 
$$(T_{Z(\tet)}^\Psi x)_n=x_{n+1}-\Psi(\varphi^n\theta,1)x_n \hbox{ for }x\in \lz.$$
We now see that $$T_{Z(\tet)}^\Psi=T_{Z(\tet)}+T_{Z(\tet)}^\Psi-T_{Z(\tet)}=T_{Z(\tet)}\left(I+T_{Z(\tet)}^{-1}(T_{Z(\tet)}^\Psi-T_{Z(\tet)})\right).$$
These identities yield that  $T_{Z(\tet)}^\Psi$ is invertible provided that $\|T_{Z(\tet)}^\Psi-T_{Z(\tet)}\|<\frac{1}{\|T_{Z(\tet)}^{-1}\|}$.

We next  estimate  the norm $\|T_{Z(\tet)}^\Psi-T_{Z(\tet)}\|$. In fact, for each $x\in \lz$ we have that
\begin{eqnarray*}
             \|T_{Z(\tet)}^{\Psi}x-T_{Z(\tet)}x\|
             &=&\sup_{n\in\n}\|\Psi(\p^n{\tet};1)x_n-\P(\p^n{\tet};1)x_n\|\\
             &\leq&\sup_{n\in\n}\|\Psi(\p^n{\tet};1)-\P(\p^n{\tet};1)\|\|x\|\\ 
          &\leq& \delta \|x\|.                                                            
\end{eqnarray*}
Therefore, $\|T_{Z(\tet)}^\Psi-T_{Z(\tet)}\|\le\delta$. Since the family $(T_{Z(\tet)})_{\theta\in\Theta}$ is uniformly correct we have that the exists $\nu$ independent of $\theta$ such that $\|T_{Z(\tet)}\|\ge \nu$. Hence, if $\delta<\frac{1}{\nu}$ then  the operator $T_{Z(\tet)}^\Psi$ is invertible.
Moreover,    $\|T_{Z(\tet)}^{\Psi}\|\geq \|T_{Z(\tet)}\|-\|T_{Z(\tet)}^\Psi-T_{Z(\tet)}\|\geq\|T_{Z(\tet)}\|-\delta\ge \nu-\delta$. It follows that, if $\delta<\nu$ then the family $(T_{Z(\tet)}^\Psi)_{\theta\in\Theta}$ is also  uniformly correct.

Thus, if $\delta<\min\{\nu,\frac{1}{\nu}\}$ then, by theorem \ref{the 4.2}, 
the LSPS  $\Ga$ also  has an exponential  dichotomy. 
\end{proof}

\bibliographystyle{amsplain}
\begin{thebibliography}{10}
\bibitem{aulmin}
 B. Aulbach, N.V. Minh, { Nonlinear semigroups and the existence and 
 stability of semilinear nonautonomous evolution equations}, {\it 
Abstract
 Appl. Anal.} {\bf 1} (1996), 351-380.
\bibitem{cholei} 
S.N. Chow, H. Leiva, {Existence and roughness of the exponential 
dichotomy for skew-produt semiflows in Banach spaces}, {\it J. Diff. Eq.} 
{\bf 120} (1995), 429-477.
\bibitem{LS} Y. Latushkin , R. Schnaubelt, Evolution semigroups, translation
  algebras, and exponential dichotomy of cocycles, J. Diff. Equ. {\bf 159} 
  (1999), 321-369. 
\bibitem{hal} J. Hale, L.T. Magalh\~aes, W.M. Oliva, ``Dynamics in Infinite Dimensions", Appl. Math. Sci. {\bf
47}, Springer-Verlag 2002.
\bibitem{Hen} D. Henry, {"Geometric Theory of Semilinear Parabolic 
Equations",} Lecture Notes in Mathematics, No.~840, Springer, 
Berlin-Heidelberg-New York 1981.
\bibitem{huyha} N.T. Huy, V.T.N. Ha, {Exponential dichotomy of difference equations in $l_p$-phase spaces on the half -line.} {\it Advances in Difference Equations}, to appear.

\bibitem{huyminh} N.T. Huy, N.V. Minh, {Exponential dichotomy of difference equations and application to evolution equations on the half -line.} , {\it Computers and Math.with Appl} {\bf 42} (2001), 301-311.
\bibitem{huy}N.T. Huy, {Exponentially dichotomous operators and exponential dichotomy of evolution equations on the half -line,}   {\it Integr. Equ. Oper. Theory} {\bf 48} (2004), 
497-510.


\bibitem{MRS} N.V. Minh, F. R\"abiger, R. Schnaubelt, {Exponential stability
exponential expansiveness and exponential Dichotomy of evolution equation on the
half line,} {\it Integr. Eq. Oper. Theory} {\bf 32}(1998), 332-353.


\bibitem{sacsel}
R.J. Sacker, G.R. Sell,  Dichotomies for linear 
evolutionary
equations in Banach spaces, J. Diff.
 Equ. {\bf 113} (1994), 17-67.
 \bibitem{sacsel2}
 R.J. Sacker, G.R. Sell, The spectrum of an invariant
submanifold, J. Diff. Equ. {\bf 38} (1980),
 135-160.
\bibitem{sacsel3}
 R.J. Sacker, G.R. Sell, Singular perturbations and 
conditional
stability, J. Math. Anal. Appl. {\bf  76} (1980),
 406-431.
\bibitem{sacsel4}
 R.J. Sacker, G.R. Sell, A spectral theory for linear
differential systems, J. Diff. Equ. {\bf 27} (1978), 320-358.
 \bibitem{sac}
 J.R. Sacker, Existence of dichotomies and invariant splittings 
for
linear differential systems, IV. J. Diff.
 Equ. {\bf  27} (1978),  106-137.
\bibitem{sacsel5}
 R.J. Sacker, G.R. Sell, Lifting properties in skew-product
flows with applications to differential equations, Mem.
 Amer. Math. Soc. {\bf 11} (1977).
\bibitem{sacsel6}
 R.J. Sacker, G.R. Sell, Existence of dichotomies and 
invariant
splittings for linear differential systems, III. J.
 Diff. Equ. {\bf 22} (1976), 497-522.
\bibitem{sacsel7}
 R.J. Sacker, G.R. Sell, Existence of dichotomies and 
invariant
splittings for linear differential systems, II. J.
 Diff. Equ. {\bf 22} (1976),  478-496.
\bibitem{SY} G.R. Sell, Y. You,  "Dynamics of Evolutionary Equations", Appl. Math. Sci. {\bf 143}, Springer-Verlag 2002.
\end{thebibliography}
\end{document}

