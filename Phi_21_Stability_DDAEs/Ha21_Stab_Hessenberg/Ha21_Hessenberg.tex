% Editing tex file using vim in bash mode and then commit it with git

%%%%%%%%%%%class file
\documentclass[smallextended]{svjour3}       % onecolumn (second format)
\usepackage[10pt]{extsizes}

%\smartqed % flush right qed marks, e.g. at end of proof
%%%%%%%%%%%%%%%%%
%call  packages
\usepackage[sort&compress]{natbib}
\usepackage{lipsum}

%%%%%%%%%%%%%%%%%%%%%%%%%%%%%%%%
\pdfoutput=1
\usepackage{algorithm}
\usepackage{algpseudocode}
\usepackage{algcompatible}

%\usepackage[notcite]{showkeys}
%\usepackage{color}
%\usepackage{showlabels}
%\renewcommand{\showlabelfont}{\small\slshape\color{red}}

\usepackage{lineno}
\linenumbers

\usepackage{graphicx}
\usepackage{amssymb,amsmath,bm,mathrsfs}
\let\proof\relax\let\endproof\relax
% Using the package amsthm leads to confliction ''The \begin{proof} is already defined''
\usepackage{amsthm}
\usepackage{mathabx}
\usepackage{hyperref} %$#

%%%%%%%%%%%%%%%%%

%Math environments===========================================================
%\newtheorem{theorem}{Theorem}[section]
%\newtheorem{lemma}[theorem]{Lemma}
%\theoremstyle{definition}
%\newtheorem{definition}[theorem]{Definition}
%\newtheorem{example}[theorem]{Example}
%\newtheorem{corollary}[theorem]{Corollary}
%\newtheorem{proposition}[theorem]{Proposition}
%\newtheorem{remark}[theorem]{Remark}
%\newtheorem{conjecture}[theorem]{Conjecture}
\newtheorem{assumption}[theorem]{Assumption}
\newtheorem{hypo}[theorem]{Hypothesis}
\renewcommand{\labelenumi}{\roman{enumi}}
\numberwithin{equation}{section}
\newtheorem{procedure}[theorem]{Procedure}

\newcommand {\rank}     {\mathop{\rm rank}\nolimits}
\newcommand {\corank}   {\mathop{\rm corank}\nolimits}
\newcommand {\range}  {\mathop{\rm range}\nolimits}
\newcommand {\corange}  {\mathop{\rm corange}\nolimits}
\newcommand {\kernel}   {\mathop{\rm kernel}\nolimits}
\newcommand {\cokernel} {\mathop{\rm cokernel}\nolimits}
\newcommand {\basis}    {\mathop{\rm basis}\nolimits}
\newcommand {\sigmin}   {\mathop{\sigma_{\rm min}}\nolimits}
\newcommand {\ind}      {\mathop{\rm ind}\nolimits}
\newcommand {\opt}      {\mathop{\rm opt}\nolimits}
\newcommand{\diag}{\mbox{\rm diag}}
\newcommand{\upt}{\mbox{\rm up}}
\newcommand{\re}{\mbox{\rm Re}}
\newcommand{\im}{\mbox{\rm Im}}
\newcommand{\dps}{\displaystyle}
\newcommand{\sct} {{\,\stackrel{c}{\sim}\,}}
\newcommand{\sue} {\,{\stackrel{u}{\sim}\,}}
\newcommand{\suc} {\,{\stackrel{uc}{\sim}\,}}
%\renewcommand{\comment}[1]{}
%\renewcommand{\proof}{\par\noindent{\bf Proof}. \ignorespaces}
%\newcommand{\eproof}{\space
%	{\ \vbox{\hrule\hbox{\vrule height1.3ex\hskip0.8ex\vrule}\hrule}} \ignorespaces} %\\[0.2cm]}
%
%=======================================================================================
% new def-s and commands
%\include{HaMe12_Feb18_command}

\def\bbI{\mathbb{I}}
\def\bbL{\mathbb{L}}
\def\bbM{\mathbb{M}}

\def\om{\omega}
\def\leq{\leqslant}
\def\rar{\rightarrow}
\def\Rar{\Rightarrow}
\def\td{\Leftrightarrow}
\def\r{\hro{R}}
\def\C{\hro{C}}
\def\hro{\mathbb}
\def\N{\hro{N}}

\def\a{\alpha}
\def\b{\beta}
\def\B{\mathcal B}
\def\lb{\lambda}
\def\Lb{\Lambda}
\def\vphi{\varphi}
\def\de{\delta}
\def\De{\Delta}
\def\ga{\gamma}
\def\Si{\Sigma}
\def\si{\sigma}
\def\ka{\kappa}
\def\tka{\tilde{\kappa}}
\def\CE{\mathcal{E}}
\def\tE{\tilde{E}}
\def\hE{\hat{E}}
\def\tA{\tilde{A}}
\def\hA{\hat{A}}
\def\bA{\breve{A}}
\def\tB{\tilde{B}}
\def\tD{\tilde{D}}
\def\hB{\hat{B}}
\def\hr{\hat{r}}
\def\hv{\hat{v}}
\def\tr{\tilde{r}}
\def\trho{\tilde{\rho}}

\def\nab{\nabla}

\def\cB{\mathcal B}
\def\cM{\mathcal M}
\def\tcM{\tilde{\mathcal M}}
\def\cN{\mathcal N}
\def\cX{\mathcal X}
\def\cS{\mathcal S}
\def\tU{\tilde{U}}
\def\cU{\mathcal U}
\def\cL{\mathcal L}

\def\tN{\tilde{N}}
\def\hN{\hat{N}}
\def\tk{\tilde{k}}
\def\hk{\hat{k}}
\def\tx{\tilde{x}}
\def\tX{\tilde{X}}
\def\hX{\hat{X}}
\def\tY{\tilde{Y}}
\def\hY{\hat{Y}}
\def\ty{\tilde{y}}
\def\tv{\tilde{v}}
\def\tw{\tilde{w}}

\def\tM{\tilde{M}}
\def\tm{\tilde{m}}
\def\bM{\breve{M}}
\def\hM{\hat{M}}
\def\bm{\breve{m}}

\def\tC{\tilde{C}}
\def\hC{\hat{C}}
\def\hD{\hat{D}}

\def\tH{\tilde{H}}
\def\tF{\tilde{F}}
\def\tG{\tilde{G}}
\def\hG{\hat{G}}
\def\cG{{\cal G}}
\def\baf{\bar{f}}
\def\tg{\tilde{g}}
\def\hg{\hat{g}}
\def\tK{\tilde{K}}

\def\cW{{\cal W}}

\def\tS{\tilde{S}}
\def\tZ{\tilde{Z}}

\def\tx{\tilde{x}}
\def\tf{\tilde{f}}
\def\hf{\hat{f}}
\def\brf{\breve{f}}
\def\bX{\breve{X}}

\def\chA{\widecheck{A}}
\def\chB{\widecheck{B}}
\def\chC{\widecheck{C}}
\def\chD{\widecheck{D}}
\def\chG{\widecheck{G}}
\def\chU{\widecheck{U}}

\def\lsim{\overset{\ell}{\sim}}
% The inverse shift operator
\def\ide{\Delta_{-1}}


\def\be{\begin{equation}}
\def\ee{\end{equation}}         

\newcommand{\ben}{\begin{eqnarray}}
\newcommand{\een}{\end{eqnarray}}

\newcommand{\bsen}{\begin{subeqnarray}}
\newcommand{\esen}{\end{subeqnarray}}

\newcommand{\bens}{\begin{eqnarray*}}
\newcommand{\eens}{\end{eqnarray*}}

\def\bc{\begin{cases}}
\def\ec{\end{cases}}

\newcommand{\bsq}{\begin{subequations}}
\newcommand{\esq}{\end{subequations}}

\newcommand{\m}[1]{
	\begin{bmatrix}
		#1 
	\end{bmatrix}
}

\renewcommand{\pm}[1]{
	\begin{matrix}
		#1 
	\end{matrix}
}

\newcommand{\n}[1]{
	\|	#1 \|
}

\def\Px{P_{\mathrm{x}}}
\def\Py{P_{\mathrm{y}}}

\def\BEA{\cB^{EA}(\r_+,\r^n)}

%===============================================================================
\begin{document}
	
%\title{Regularization of Second Order Singular Difference Equations\footnotemark[1]}
%\titlerunning{Regularization of Second Order Singular Difference Equations}% Part of RIGHT running header
\title{On the Stability Analysis of linear, time-delayed Hessenberg Differential-Algebraic Equations \footnotemark[1]}
\titlerunning{On the Stability Analysis of linear, time-delayed Hessenberg Differential-Algebraic Equations} % Part of RIGHT running header
\author{ {\sc Phi Ha %\thanks{This work was supported by the research project Naforsted, and was done during the first author's visit at VIASM}
	and Do Duc Thuan}}
	%\authorrunning{Short author list}% Part of LEFT running header

\institute{Phi Ha and Do Duc Thuan \at Institute of Math-Mechanics-Informatics, Hanoi University of Science, VNU \\ 	Nguyen Trai Street 334, Thanh Xuan, Hanoi, Vietnam\\ 
		\email{\{haphi.hus;linhvu\}@vnu.edu.vn}
	}
\date{Version 1 : \today \\ Received: date / Accepted: date}      
	
\maketitle
	
\begin{abstract}
In this paper we discuss the stability analysis for linear Hessenberg Differential-Algebraic Equations with time delay.  
First we discuss the classification of these systems, which is followed by the stability analysis for not only non-advanced but also 
for \emph{weakly-advanced} systems. The idea is to transform a given system to an equivalent regular, impulse-free system via an \emph{index reduction procedure}, 
which preserves the spectrum of the original system. 
Then, we introduce a new concept of $C^p$-weak exponential stability and study it via the the spectral method.
Numerical examples are presented to illustrate the advantages of the proposed results.	
\end{abstract}
	
\noindent
{\bf Keywords:} Singular systems; Delay; Spectral.

\noindent
{\bf AMS Subject Classification:} 34A09, 34A12, 65L05, 65H10

%=========================================================================================
\noindent \textbf{Nomenclature} \\[.1cm]
%
\begin{tabular}{|c|c|}
	\hline
	$\N$ ($\N_0$)	&  the set of natural numbers (including $0$)  \\
	$\R$ ($\R_+$)  	&  the set of real (non-negative real) numbers \\
	$\C$ ($\C_{-}$) & the set of complex numbers (the set $\{\lb \in \C \ | \ Re \lb <0\}$) \\
	$I$ ($I_n$)		&  the identity matrix (of size $n \times n$) \\
	$x^{(j)}$       &  the $j$-th derivative of a function $x$ \\
	$C^p([-\tau,0],\r^n)$ & the space of $p$-times continuously differentiable functions \\
					& from $[-\tau,0]$ to $\r^n$ (for $ 0 \leq p < \infty$) \\
	$\|\cdot\|_{p}$ & the $p$-norm of the Banach space $C^p([-\tau,0],\r^n)$, i.e. \\
	                & $\n{f}_p := \sum_{j=1}^{p} \underset{t\in [-\tau,0]}{\sup} \n{f^{(p)}(t)}$ \\ 
	$\|\cdot\|_{\infty}$ & the sup-norm of the Banach space $C^0([-\tau,0],\r^n)$ \\ 
    $A(i,:)$        & the $i$-th row of matrix $A$ (in MATLAB notation) \\
    $A(i:j,:)$      & the rows of $A$, ranging from the $i$-th row to the $j$-th row (for $i\leq j$) \\
    $\Row(i)$        & the $i$-th block row equation of a system \\
    $\Row(i:j)$      & the block row equations, ranging from the $i$-th row to the $j$-th row (for $i\leq j$) \\
    $\De$             & the shift backward operator $\De: x(t) \mapsto x(t-\tau)$  \\
\hline
\end{tabular}

%=========================================================================================

\section{Introduction and Preliminaries}\label{intro}

In the present paper we study the stability analysis of linear, time invariant {\it delay dif\-fe\-ren\-tial-al\-ge\-braic equations (DDAEs)} of the following form
%
\begin{equation}\label{H-DDAE}
 E \dot{x}(t) = A^{(0)} x(t) + A^{(1)} x(t-\tau), 
\end{equation}
%
for all $t\in [0,\infty)$, where the matrix coefficients belong to $\R^{n,n}$, $x:[-\tau,\infty) \rar \R^n$, and $\tau>0$ is a constant delay. 
However, we do not aim at the stability of general system, but only for a class of Hessenberg system, where the matrix coefficients have the special structure as follows
%
{\scriptsize
\begin{equation}
E \!=\! \m{I &  &  &  &  & \\ & I  &  &  &  & \\  &  &  & \ddots & & \\ &  &  &  & I & \\ &  &  &   &   & 0 }, A^{(0)} \!=\! \HkA{0}, A^{(1)} \!=\! \HkA{1},
\end{equation}
}
%
and the matrix product
%
\[ A^{(0)}_{k,k-1} A^{(0)}_{k-1,k-2}  \cdots A^{(0)}_{2,1} A^{(0)}_{1,k} \]
%
is nonsingular. Here $k \geq 2$ and we say that the Hessenberg DDAE \eqref{H-DDAE} has an index $k$. On the other hand, index-1 (Hessenberg) systems take the form
%
\begin{equation}\label{H-DDAE index 1}
	\m{I & 0 \\ 0 & 0} \dot{x}(t) = \m{\HA{0}_{11} & \HA{0}_{12} \\ \HA{0}_{21} & \HA{0}_{22}} x(t) + \m{\HA{1}_{11} & \HA{1}_{12} \\ \HA{1}_{21} & \HA{1}_{22}} x(t-\tau),
\end{equation}
%
where $\HA{0}_{22}$ is nonsingular.

{\textbf{write some more ... Hessenberg differential-algebraic equations arises from ... \\
In the following example we demonstrate some difficulties that may arise in the stability analysis of DDAEs. \\
}}

To achieve uniqueness of solutions for DDAEs of the form \eqref{H-DDAE} one typically has to prescribe an initial function, which takes the form
%
\begin{equation}\label{initial function}
	x|_{[-\tau,0]}=\vphi: [-\tau,0] \rightarrow \R^{n}.
\end{equation}
%
Throughout this paper, we use the following solution concept. % for \eqref{H-DDAE}.
%
\begin{definition}\label{solution}
i) A function $x:[-\tau,\infty)\rar\R^n$ is called a \emph{piecewise differentiable solution} of \eqref{H-DDAE}, if $Ex$ is piecewise continuously 
	differentiable, $x$ is continuous and satisfies \eqref{H-DDAE} at every  $t\in [0,\infty) \setminus \underset{j\in \N_0}{\cup} \{j\tau\}$. \\
ii) An initial function $\vphi$ is called \emph{consistent} with \eqref{H-DDAE} if the associated initial value problem (IVP) \eqref{H-DDAE}, \eqref{initial function} has at least one solution. \\
iii) System \eqref{H-DDAE} is called \emph{solvable} (resp. \emph{regular}) if for every consistent initial function $\vphi$,
	the associated IVP \eqref{H-DDAE}, \eqref{initial function} has a solution (resp. has a unique solution). \\
iv) The set $\si(E,\HA{0},\HA{1}):= \{\lb \in \C \ | \ \det(\lb E -\HA{0}-e^{-\lb \tau}\HA{1}) = 0\}$ is called the \emph{spectrum} of \eqref{H-DDAE}. 
\end{definition}
%

%=================================================================
% \section{}\label{Sec2}

%=================================================================
\section{Main Results}\label{Sec3}

\subsection{The case of index $k = 2$, $3$}\label{Sec3.1}
In this part we demonstrate the index reduction strategy for Hessenberg DDAEs of index $k\leq 3$ and its consequences to the solvability and stability analysis of system \eqref{H-DDAE}.
For $k=2$, we rewrite the system as follows
%
\begin{equation}\label{H-DDAE index 2}
	\m{0 \\ 0} = -\m{I & 0 \\ 0 & 0} \dot{x}(t) + \HtwoA{0} x(t) + \HtwoA{1} x(t-\tau), \mbox{ for all } t\geq 0,
\end{equation}
%
where $A^{(0)}_{21} A^{(0)}_{12}$ is nonsingular. \\
%
We transform the system to the Hessenberg index 1 form by replacing the second block row equation, denoted by $\Row(2)$, 
by the new one $\Row(2)^{new}$ defined by
%
\begin{equation}\label{eq3.1}
	\Row(2)^{new} = \dfrac{d}{dt} \ \Row(2) + A^{(0)}_{21} \ \Row(1) + A^{(1)}_{21} \De \ \Row(1),
\end{equation}
%
where $\De$ is the shift backward operator, which maps $x(t)$ to $x(t-\tau)$. Here by $\De \ \Row(2)$ we mean that the whole block row equation has been shifted backward, i.e.
%
\[
 0 = A^{(0)}_{21}x(t-\tau) + A^{(1)}_{21} x(t-2\tau), \mbox{ for all } t\geq \tau.
\]
%
Consequently, the equation \eqref{eq3.1} becomes
%
\begin{align}\label{eq3.2}
\notag   0 = & A^{(0)}_{21} \m{A^{(0)}_{11} & A^{(0)}_{12}} x(t) + \left( A^{(0)}_{21} \m{A^{(1)}_{11} & A^{(1)}_{12}} + A^{(1)}_{21} \m{A^{(0)}_{11} & A^{(0)}_{12}} \right) x(t-\tau)  \\
     & + A^{(1)}_{21} \m{A^{(1)}_{11} & A^{(1)}_{12}} x(t-2\tau),  \mbox{ for all } t\geq \tau.
\end{align}
%
Thus, combining this equation with the first equation of \eqref{H-DDAE index 2} gives us the system
%
\begin{align}\label{eq3.3}
\notag 	\m{0 \\ 0} = & -\m{I & 0 \\ 0 & 0} \dot{x}(t) + \m{A^{(0)}_{11} & A^{(0)}_{12} \\ A^{(0)}_{21}A^{(0)}_{11} & A^{(0)}_{21}A^{(0)}_{12} } x(t) + \m{A^{(0)}_{11} & A^{(0)}_{12} \\ * & *} x(t-\tau)\\
					 & + \m{0 & 0 \\ A^{(1)}_{21} A^{(1)}_{11} & A^{(1)}_{21} A^{(1)}_{12} }  x(t-2\tau), \mbox{ for all } t\geq \tau,
\end{align}
%
which is clearly an index 1 system, since $A^{(0)}_{21} A^{(0)}_{12}$ is nonsingular.

\begin{remark}\label{rem 1}
We notice, that if we rewrite system \eqref{H-DDAE index 2} in the operator form 
%
\begin{equation}\label{eq3.4}
 0 = \cP(\dfrac{d}{dt},\De) x := \left( -\m{I & 0 \\ 0 & 0} \dfrac{d}{dt} + \HtwoA{0} + \HtwoA{1} \ \De \right) x(t),
\end{equation}
%
then system \eqref{eq3.3} is obtained by simply acting the operator $\sm{I & \quad 0 \\ A^{(0)}_{21} + A^{(1)}_{21} \De &  \quad \dfrac{d}{dt}}$ on \eqref{H-DDAE index 2}. This leads to a consequence, that \emph{an index reduction step}, which transforming the index-2 system \eqref{H-DDAE index 2} to the index-1 system \eqref{eq3.3}, 
does not alter the non-zero eigenvalues. This is very important, in particular to study the stability analysis, as we will see later in Section \ref{Sec3.3}.  
\end{remark}

\begin{remark}\label{rem 2}
From the numerical viewpoint, in fact we can simplify the index reduction step above by transforming the matrix coefficient in the second row as follows.
%
\begin{align*}
	& A^{(0)}(2,:) := A^{(0)}(2,:) \ A^{(0)}, \\
	& A^{(1)}(2,:) := A^{(0)}(2,:) \ A^{(1)} + A^{(1)}(2,:) \ A^{(0)}, 
\end{align*}
%
and introduce a new matrix coefficient $A^{(2)}$ associated with $x(t-2\tau)$ via
%
\[
  A^{(2)} := \m{ 0 \\ A^{(1)}(2,:)  \ A^{(1)} } \ . 
\]
%
\end{remark}

Now let us consider the case of index-3 Hessenberg DDAEs (i.e., $k=3$) of the form
%
\begin{equation}\label{H-DDAE index 3}
 	\m{0 \\ 0  \\ 0} \!=\! -\m{I & 0 & 0 \\  0 & I & 0 \\ 0 & 0 & 0} \dot{x}(t) +\! \HthreeA{0} x(t) +\! \HthreeA{1} x(t-\tau),
\end{equation}
%
for all $t\geq 0$, where the matrix product $A^{(0)}_{32} A^{(0)}_{21} A^{(0)}_{13}$ is nonsingular. 
Our index reduction procedure consists of two steps: Step 1: reduce an index from $k=3$ to $k=2$; and Step 2: reduce an index from $k=2$ to $k=1$ as above.
Similarly to \eqref{eq3.1}, Step 1 is done by performing a transformation on the last row only, i.e.,
%
\begin{equation}
 \Row(3) \mapsto \Row(3)^{new} := \dfrac{d}{dt} \ \Row(3) + A^{(0)}_{32} \ \Row(2) + A^{(1)}_{32} \De \ \Row(2).
\end{equation}
%
The new system now takes the from
%
\begin{equation}
	0 = \m{I \quad & 0 & \quad 0 \\  0 \quad & I &  \quad 0 \\ 0 \quad &   A^{(0)}_{32}+A^{(1)}_{32} \De & \quad \dfrac{d}{dt}} \cP(\dfrac{d}{dt},\De) x \ .
\end{equation}
%
Continue performing Step 2, as in the case $k=2$, we obtain an index-1 Hessenberg DDAE of the form
%
\begin{align}
\notag	    \m{0 \\ 0  \\ 0} \!=\! & -\m{I & 0 & 0 \\  0 & I & 0 \\ 0 & 0 & 0} \dot{x}(t) \!+\! \m{A^{(0)}_{11} & A^{(0)}_{12} & A^{(0)}_{13} \\ A^{(0)}_{21} & A^{(0)}_{22} & 0 \\ * & * & A^{(0)}_{32} A^{(0)}_{21} A^{(0)}_{13} } x(t) \!+\! \m{A^{(1)}_{11} & A^{(1)}_{12} & A^{(1)}_{13} \\ A^{(1)}_{11} & A^{(1)}_{12} & 0 \\ * & * & *} x(t-\tau) \\
	 	& + \m{0 & 0 & 0 \\ 0 & 0 & 0 \\ * & * & *}x(t-2\tau) +  \m{0 & 0 & 0 \\ 0 & 0 & 0 \\ * & * & *}x(t-3\tau) , \mbox{ for all } t\geq \tau.
\end{align}
%
Here $*$ stands for an arbitrary matrix. Here we notice again, that the index reduction procedure does not alter the non-zero eigenvalues of system \eqref{H-DDAE index 3}.


\subsection{The general case}\label{Sec3.2}

The index reduction procedure presented above can ge directly generalized to index-$k$ Hessenberg DDAEs in the following algorithm.

\begin{algorithm}[H]
	\caption{Index reduction procedure of the index-$k$ Hessenberg DDAE \eqref{H-DDAE}}
	\label{Alg1}
	\textbf{Input:} The system coefficients $E$, $A^{(0)}$, $A^{(1)}$. \\
	\textbf{Output:} The system coefficients $E$, $A^{(0)},\dots,A^{(k)}$ of the new system. 
	\begin{algorithmic}[1]
		\FOR{ j = 1: k-1}
		\State Update the last row of matrices $A^{(0)},\dots,A^{(j)}$ by 
		\begin{align*}
			& A^{(0)} (k,:)   = A^{(0)} (k,:) A^{(0)}, \\
			& A^{(\ell)}(k,:) = A^{(0)}(k,:) A^{(\ell)} + A^{(1)}(k,:) A^{(\ell-1)}. 
		\end{align*} 
		\State	Introduce a new matrix $A^{(j+1)} := \m{ 0 \\ A^{(j)} (k,:) A^{(1)} } \in \R^{n,n}$. 
		%		%
		%		\algstore{myalg}
		%		\end{algorithmic}
		%		\end{algorithm}	
		%		\begin{algorithm}[H]
		%		\begin{algorithmic}[1]
		%		\algrestore{myalg}
		%		%
		\ENDFOR		
	\end{algorithmic}
\end{algorithm}



\begin{theorem}\label{thm 1}
	Consider the index-k Hessenberg DDAE \eqref{H-DDAE} and assume that it is uniquely solvable for all consistent and sufficiently smooth initial function $\phi$. Provided that the solution $x$ is alrealdy known/computed on the interval $[0,(k-1)\tau]$, then system \eqref{H-DDAE} has exactly the same solution $x$ on $[(k-1)\tau,\infty)$
	as the index-1 system of the form
%
{\small
\begin{align}\label{transfored system}
	\notag	
	& \m{I &  &  &  &  & \\ & I  &  &  &  & \\  &  &  & \ddots & & \\ &  &  &  & I & \\ \hline &  &  &   &   & 0 } \dot{x}(t) 
	\!=\!   
	\m{ A^{(0)}_{11} & A^{(0)}_{12} & \hdots & A^{(0)}_{1,k-1} 		& \quad A^{(0)}_{1,k} 	\\ 
		A^{(0)}_{21} & A^{(0)}_{22} & \hdots & A^{(0)}_{2,k-1} 		& \quad 0 				\\ 
					 & \ddots        & \ddots & \vdots 				& \quad \vdots         	\\ 
					 &               & \ddots & A^{(0)}_{k-1,k-1} 	& \quad 0 				\\ \hline   												
		*		  	 &   *           &   *    & *			 		& \quad \mathbf{\HA{0}_{k,k}}	}
	x(t)
	\\
	& +
	\m{ A^{(1)}_{11} & A^{(1)}_{12}  & \hdots & \quad A^{(1)}_{1,k-1} 		& \quad A^{(1)}_{1,k} 	\\ 
		A^{(1)}_{21} & A^{(1)}_{22}  & \hdots & \quad A^{(1)}_{2,k-1} 		& \quad 0 				\\ 
					 & \ddots        & \ddots & \quad \vdots 				& \quad \vdots         	\\ 
		 			 &               & \ddots & \quad A^{(1)}_{k-1,k-1} 	& \quad 0 				\\ \hline   												
		*			 &   *           & *      & \quad 	*			 		& \quad * 				}
	x(t-\tau) 
	+ \sum_{j=2}^{k}  \m{0 & 0 & \dots & 0 & 0 \\ 0 & 0 & \dots & 0 & 0 \\ 0 & 0 & \dots & 0 & 0 \\\hline * & * & \dots & * & *} x(t-j\tau) , 
\end{align}
}
%
for all $t\geq \tau$, where $*$ stands for an arbitrary matrix, and the matrix
%
\[
 \mathbf{\HA{0}_{k,k}} :=  A^{(0)}_{k,k-1} A^{(0)}_{k-1,k-2}  \cdots A^{(0)}_{2,1} A^{(0)}_{1,k}
\]
%
is nonsingular. Furthermore, the transformed system \eqref{transfored system} preserves all non-zero eigenvalues of system \eqref{H-DDAE}.
\end{theorem}

\begin{corollary}\label{coro 1} 
	Consider the index-k Hessenberg DDAE \eqref{H-DDAE} and assume that it is uniquely solvable for all consistent and sufficiently smooth initial function $\phi$.
	In order to have a continuous, piecewise differentiable solution $x|_{[0,\infty)}$, the smoothness requirement for $\phi$ is upper-bounded by $(k-1)^2$. 
\end{corollary}

\subsection{Stability analysis}\label{Sec3.3}

We recall the stability concept for DDAEs as follows.
%
\begin{definition}\label{def3.2}(\cite{Mic11,XuDSL02}) \ The null solution $x=0$ of the DDAE \eqref{H-DDAE} is called \emph{exponentially stable} if there exist positive constants 
	$\de$ and $\ga$ such that for any consistent initial function $\vphi \in C([-\tau,0],\R^n)$, the solution $x=x(t,\vphi)$ of the corresponding 
	IVP to \eqref{H-DDAE} satisfies
	%
	\[
	\|x(t)\| \leq \de e^{-\ga t} \|\vphi\|_{\infty}, \ \mbox{ for every } \ t\geq 0.
	\]
	%
\end{definition}
%

\begin{definition} The DDAE \eqref{H-DDAE} is called \\
i)  \emph{non-advanced} (or \emph{impulse-free}) if for any consistent $\vphi \in C([-\tau,0],\R^n)$, there exists a unique solution $x$ to the corresponding IVP for \eqref{H-DDAE}. \\ 
ii) \emph{$C^k$-weakly advanced} if for any consistent $\vphi \in C^p([-\tau,0],\R^n)$, there exists a unique solution $x$ to the corresponding IVP for \eqref{H-DDAE}. 
\end{definition}

\begin{example}
It is well-known, see e.g. \cite{AscP95,ZhuP98,CamL09}, that for index-$2$ Hessenberg system \eqref{H-DDAE index 2}, the system is non-advanced if $A^{(1)}_{21} = 0$.  
For index-$2$ Hessenberg system \eqref{H-DDAE index 2}, the system is non-advanced if $A^{(1)}_{32} = 0$ and $A^{(1)}_{21} = 0$. 
From our discussion in Section \ref{Sec3.1}, we see that system \eqref{H-DDAE index 2} is $C^2$-weakly advanced if $A^{(1)}_{21} \m{A^{(1)}_{11} & A^{(1)}_{12}} \not= 0$. 
In case of the index-$3$ system \eqref{H-DDAE index 3}, it is $C^4$-weakly advanced if $A^{(1)}_{32} \m{A^{(1)}_{21} & A^{(1)}_{22}} \not= 0$. 
Furthermore, it is $C^2$-weakly advanced if $A^{(1)}_{32} \m{A^{(1)}_{21} & A^{(1)}_{22}} = 0$.
Due to Corollary \ref{coro 1}, we see that the index-$k$ Hessenberg system \eqref{H-DDAE} is $C^{(k-1)^2}$-weakly advanced. 
\end{example}

The characterization for exponential stability of the index-$1$ DDAE \eqref{H-DDAE index 1} is given in the following proposition.
%
\begin{proposition}(\cite{Mic11,DuLMT13}) The index-$1$ DDAE \eqref{H-DDAE index 1} is exponentially stable if and only if the spectrum $\si(E,A^{(0)},A^{(1)})$ lies entirely on the left half plane and is bounded away from the imaginary axis.
	
\end{proposition}

%
\begin{definition}\label{def3.3}
	The null solution $x=0$ of the DDAE \eqref{H-DDAE} is called \emph{$C^p$-weakly exponentially stable ($C^p$-w.e.s)} if there exist an integer 
	$0\leq p < \infty$ and positive constants $\de$ and $\ga$ such that for any consistent initial function $\vphi \in C^p([-\tau,0],\R^n)$, the solution $x=x(t,\vphi)$ of the corresponding IVP for \eqref{H-DDAE} satisfies
	%
	\[ \|x(t)\| \leq \de e^{-\ga t} \|\vphi\|_{p}, \ \mbox{ for all } t\geq 0. \]
	%
\end{definition}
%
Notice that the (classical) exponential stability is exactly $C^0$-w.e.s.. 
Furthermore, even though $C^p$-w.e.s. has been considered for ODEs and PDEs as well, till now there are very few reference for DDAEs, see \cite{Ha18,Ha18b}.

\begin{theorem}
 Consider the index-k Hessenberg DDAE \eqref{H-DDAE} and assume that it is uniquely solvable for all consistent and sufficiently smooth initial function $\phi$. 
 We also consider the transformed system \eqref{transfored system} obtained by applying Algorithm \ref{Alg1} to system \eqref{H-DDAE}. 
 Furthermore, we assume that $0\notin \si(E,A^{(0)},A^{(1)})$. 
 Then system \eqref{H-DDAE} is $C^{(k-1)^2}$-weakly exponentially stable, provided that the spectrum $\si(E,A^{(0)},A^{(1)})$ lies entirely on the left half plane and is bounded away from the imaginary axis.
\end{theorem}


\begin{corollary}
Stability condition
\end{corollary}


\begin{example}
Numerical test
\end{example}


%=================================================================
\section{Conclusion and Outlook}\label{conclusion}


%========================================================================================
\vskip 0.2cm
\textbf{Acknowledgment} The author would like to thank the anonymous referee for his suggestions to improve this paper.


%============================================================================
\bibliographystyle{abbrv}
\bibliography{Phi_July_2020}

%\appendix
%\section{Proof of Lemma \ref{lem5.1}}\label{appendixA}



\end{document}
