%% Version 1 : xx/07/2021 Phi
%% 
%% ---------------------------------------------
%%

\documentclass[preprint,12pt]{elsarticle}

%% Use the option review to obtain double line spacing
%% \documentclass[authoryear,preprint,review,12pt]{elsarticle}

%% Use the options 1p,twocolumn; 3p; 3p,twocolumn; 5p; or 5p,twocolumn
%% for a journal layout:
%% \documentclass[final,1p,times]{elsarticle}
%% \documentclass[final,1p,times,twocolumn]{elsarticle}
%% \documentclass[final,3p,times]{elsarticle}
%% \documentclass[final,3p,times,twocolumn]{elsarticle}
%% \documentclass[final,5p,times]{elsarticle}
%% \documentclass[final,5p,times,twocolumn]{elsarticle}

%% For including figures, graphicx.sty has been loaded in
%% elsarticle.cls. If you prefer to use the old commands
%% please give \usepackage{epsfig}

\usepackage{graphicx}
\usepackage{amssymb,amsmath,bm,mathrsfs}
%\let\proof\relax\let\endproof\relax
% Using the package amsthm leads to confliction ''The \begin{pf} is already defined''
\usepackage{amsthm}
\usepackage{mathabx}

\usepackage{algorithm}
\usepackage{algpseudocode}
\usepackage{algcompatible}

%% The amssymb package provides various useful mathematical symbols
%%\usepackage{amssymb}
%% The amsthm package provides extended theorem environments
%% \usepackage{amsthm}

%% The lineno packages adds line numbers. Start line numbering with
%% \begin{linenumbers}, end it with \end{linenumbers}. Or switch it on
%% for the whole article with \linenumbers.
%% \usepackage{lineno}

\usepackage{lineno}
\linenumbers

\usepackage[right]{showlabels} 
\renewcommand{\showlabelfont}{\ttfamily\scriptsize} 


\usepackage{times}

%\usepackage{tikz}
%\usetikzlibrary{decorations.pathreplacing}
%\makeatletter
%\def\underbrace#1{\@ifnextchar_{\tikz@@underbrace{#1}}{\tikz@@underbrace{#1}_{}}}
%\def\tikz@@underbrace#1_#2{\tikz[baseline=(a.east)] {\node (a) {\(#1\)}; \draw[ultra thick,line cap=round,decorate,decoration={brace,amplitude=5pt}] (a.south east) -- node[below,inner sep=7pt] {\(\scriptstyle #2\)} (a.south west);}}
%\makeatother


%\usepackage[square, numbers, comma, sort&compress]{natbib}
%\usepackage[authoryear]{natbib}

\usepackage[final]{hyperref} %$#

%\hypersetup{
%	colorlinks=true,
%	linkcolor=blue,
%	filecolor=magenta,      
%	urlcolor=red,
%}

\journal{Journal of The Franklin Institute}

\newtheorem{thm}{Theorem}
\newtheorem{lem}[thm]{Lemma}
\newtheorem{example}[thm]{Example}
\newtheorem{ass}[thm]{Assumption}
\newdefinition{rem}{Remark}
\newproof{pf}{Proof}
\newtheorem{definition}[thm]{Definition}
\newtheorem{coro}[thm]{Corollary}
\newtheorem{prop}[thm]{Proposition}


\newcommand {\rank}     {\mathop{\rm rank}\nolimits}
\newcommand {\corank}   {\mathop{\rm corank}\nolimits}
\newcommand {\range}  {\mathop{\rm range}\nolimits}
\newcommand {\corange}  {\mathop{\rm corange}\nolimits}
\newcommand {\kernel}   {\mathop{\rm kernel}\nolimits}
\newcommand {\cokernel} {\mathop{\rm cokernel}\nolimits}
\newcommand {\basis}    {\mathop{\rm basis}\nolimits}
\newcommand {\sigmin}   {\mathop{\sigma_{\rm min}}\nolimits}
\newcommand {\ind}      {\mathop{\rm ind}\nolimits}
\newcommand {\opt}      {\mathop{\rm opt}\nolimits}
\newcommand{\diag}{\mbox{\rm diag}}
\newcommand{\upt}{\mbox{\rm up}}
\newcommand{\re}{\mbox{\rm Re}}
\newcommand{\im}{\mbox{\rm Im}}
\newcommand{\dps}{\displaystyle}
\newcommand{\sct} {{\,\stackrel{c}{\sim}\,}}
\newcommand{\sue} {\,{\stackrel{u}{\sim}\,}}
\newcommand{\suc} {\,{\stackrel{uc}{\sim}\,}}
%\renewcommand{\comment}[1]{}
%\renewcommand{\proof}{\par\noindent{\bf Proof}. \ignorespaces}
%\newcommand{\eproof}{\space
%	{\ \vbox{\hrule\hbox{\vrule height1.3ex\hskip0.8ex\vrule}\hrule}} \ignorespaces} %\\[0.2cm]}
%
%=======================================================================================
% new def-s and commands
%\include{HaMe12_Feb18_command}

\def\bbI{\mathbb{I}}
\def\bbL{\mathbb{L}}
\def\bbM{\mathbb{M}}

\def\om{\omega}
\def\leq{\leqslant}
\def\rar{\rightarrow}
\def\Rar{\Rightarrow}
\def\td{\Leftrightarrow}
\def\r{\hro{R}}
\def\C{\hro{C}}
\def\hro{\mathbb}
\def\N{\hro{N}}

\def\a{\alpha}
\def\b{\beta}
\def\B{\mathcal B}
\def\lb{\lambda}
\def\Lb{\Lambda}
\def\vphi{\varphi}
\def\de{\delta}
\def\De{\Delta}
\def\ga{\gamma}
\def\Si{\Sigma}
\def\si{\sigma}
\def\ka{\kappa}
\def\tka{\tilde{\kappa}}
\def\CE{\mathcal{E}}
\def\tE{\tilde{E}}
\def\hE{\hat{E}}
\def\tA{\tilde{A}}
\def\hA{\hat{A}}
\def\bA{\breve{A}}
\def\tB{\tilde{B}}
\def\tD{\tilde{D}}
\def\hB{\hat{B}}
\def\hr{\hat{r}}
\def\hv{\hat{v}}
\def\tr{\tilde{r}}
\def\trho{\tilde{\rho}}

\def\nab{\nabla}

\def\cB{\mathcal B}
\def\cM{\mathcal M}
\def\tcM{\tilde{\mathcal M}}
\def\cN{\mathcal N}
\def\cX{\mathcal X}
\def\cS{\mathcal S}
\def\tU{\tilde{U}}
\def\cU{\mathcal U}
\def\cL{\mathcal L}

\def\tN{\tilde{N}}
\def\hN{\hat{N}}
\def\tk{\tilde{k}}
\def\hk{\hat{k}}
\def\tx{\tilde{x}}
\def\tX{\tilde{X}}
\def\hX{\hat{X}}
\def\tY{\tilde{Y}}
\def\hY{\hat{Y}}
\def\ty{\tilde{y}}
\def\tv{\tilde{v}}
\def\tw{\tilde{w}}

\def\tM{\tilde{M}}
\def\tm{\tilde{m}}
\def\bM{\breve{M}}
\def\hM{\hat{M}}
\def\bm{\breve{m}}

\def\tC{\tilde{C}}
\def\hC{\hat{C}}
\def\hD{\hat{D}}

\def\tH{\tilde{H}}
\def\tF{\tilde{F}}
\def\tG{\tilde{G}}
\def\hG{\hat{G}}
\def\cG{{\cal G}}
\def\baf{\bar{f}}
\def\tg{\tilde{g}}
\def\hg{\hat{g}}
\def\tK{\tilde{K}}

\def\cW{{\cal W}}

\def\tS{\tilde{S}}
\def\tZ{\tilde{Z}}

\def\tx{\tilde{x}}
\def\tf{\tilde{f}}
\def\hf{\hat{f}}
\def\brf{\breve{f}}
\def\bX{\breve{X}}

\def\chA{\widecheck{A}}
\def\chB{\widecheck{B}}
\def\chC{\widecheck{C}}
\def\chD{\widecheck{D}}
\def\chG{\widecheck{G}}
\def\chU{\widecheck{U}}

\def\lsim{\overset{\ell}{\sim}}
% The inverse shift operator
\def\ide{\Delta_{-1}}


\def\be{\begin{equation}}
\def\ee{\end{equation}}         

\newcommand{\ben}{\begin{eqnarray}}
\newcommand{\een}{\end{eqnarray}}

\newcommand{\bsen}{\begin{subeqnarray}}
\newcommand{\esen}{\end{subeqnarray}}

\newcommand{\bens}{\begin{eqnarray*}}
\newcommand{\eens}{\end{eqnarray*}}

\def\bc{\begin{cases}}
\def\ec{\end{cases}}

\newcommand{\bsq}{\begin{subequations}}
\newcommand{\esq}{\end{subequations}}

\newcommand{\m}[1]{
	\begin{bmatrix}
		#1 
	\end{bmatrix}
}

\renewcommand{\pm}[1]{
	\begin{matrix}
		#1 
	\end{matrix}
}

\newcommand{\n}[1]{
	\|	#1 \|
}

\def\Px{P_{\mathrm{x}}}
\def\Py{P_{\mathrm{y}}}

\def\BEA{\cB^{EA}(\r_+,\r^n)}

\begin{document}

\begin{frontmatter}

%% Title, authors and addresses

%% use the tnoteref command within \title for footnotes;
%% use the tnotetext command for theassociated footnote;
%% use the fnref command within \author or \address for footnotes;
%% use the fntext command for theassociated footnote;
%% use the corref command within \author for corresponding author footnotes;
%% use the cortext command for theassociated footnote;
%% use the ead command for the email address,
%% and the form \ead[url] for the home page:
%% \title{Title\tnoteref{label1}}
%% \tnotetext[label1]{}
%% \author{Name\corref{cor1}\fnref{label2}}
%% \ead{email address}
%% \ead[url]{home page}
%% \fntext[label2]{}
%% \cortext[cor1]{}
%% \address{Address\fnref{label3}}
%% \fntext[label3]{}

\title{On the stability analysis of arbitrarily high-index singular systems with multiple delays}

%% use optional labels to link authors explicitly to addresses:
%% \author[label1,label2]{}
%% \address[label1]{}
%% \address[label2]{}

\author[add1]{Ha Phi\corref{cor1} %\fnref{fn1}
}
\ead{haphi.hus@vnu.edu.vn}

%\author[add1]{Ha Trung\fnref{fn2}}
%\ead{hatrung.hus@vnu.edu.vn}


\address[add1]{Faculty of Mathematics, Mechanics, and Informatics, VNU University of Science, Vietnam National University, Hanoi, Vietnam.}

\cortext[cor1]{Corresponding author}
%\fntext[fn1]{Footnote of first author.}
%\fntext[fn2]{The second author was supported the National Foundation for Science and Technology Development (NAFOSTED) under the project number 101.01-2017.302. He also would like to thank the Vietnam Institute for Advanced Study in Mathematics (VIASM) for their kind hospitality during his research visit.}

\begin{abstract}
This paper is devoted to the stability analysis for the class of arbitrarily high-index (continuous-time) singular linear systems with multiple delays. 
By transforming the originally given system to an equivalent regular, impulse-free system, the global exponential stability problem is addressed by both approaches: 
spectral and Lyanpunov-Krasovskii. Characterizations for the stability are developed in both the spectral condition and the linear matrix inequality (LMI) setting. 
Moreover, an estimate of the convergence rate of such stable systems is presented. 
Numerical examples are presented to illustrate the advantages of the proposed results.	
\end{abstract}

% \newpageafter{abstract}

%%Graphical abstract
%%\begin{graphicalabstract}
%\includegraphics{grabs}
%%\end{graphicalabstract}

%%Research highlights
%\begin{highlights}
%\item Research highlight 1
%\item Research highlight 2
%\end{highlights}

\begin{keyword}
%% keywords here, in the form: keyword \sep keyword
Singular systems \sep Delay \sep LMIs \sep Spectral \sep Stabilization \sep Feedback.
%% PACS codes here, in the form: \PACS code \sep code

%% MSC codes here, in the form: \MSC code \sep code
%% or \MSC[2008] code \sep code (2000 is the default)
\MSC 34D20 \sep 93D05 \sep 93D20
\end{keyword}

\end{frontmatter}

%% \linenumbers

%% main text

\section{Introduction}\label{intro}
Consider the linear singular time-delay system of the form
%
\begin{align}\label{delay-descriptor}
	E\dot{x}(t) &= A_0 x(t) + \sum_{i=1}^{m} A_i x(t-\tau_i) + B u(t), \ \mbox{ for all } t\in [t_0,\infty ), \\
	x(t) &= \phi(t), \mbox{ for all } t_0-\tau_m \leq t \leq t_0 ,
\end{align}
%
where $E \in \R^{n,n}$ is allowed to be singular. Here the state is $x:[t_0-\tau_m,\infty ) \rar \r^n$, and the (constant) time-delays satisfy $ 0 < \tau_1 < \tau_2 < ... < \tau_m$. The capital letters are real-valued matrices of appropriate dimensions. The system is called \emph{free (or DDAE)} if we let $u\equiv 0$, i.e., the system reads
%
\begin{equation}\label{free system}
	E\dot{x}(t) = A_0 x(t) + \sum_{i=1}^{m} A_i x(t-\tau_i) \ .
\end{equation}%

\textbf{
The motivation for the system description \ref{delay-descriptor} in the context
of designing controllers lies in its generality in modelling interconnected systems.
}

The rest of the paper is organized as follows. In Section \ref{pre}, some definitions concerning about the solution and the system classification are stated. Auxiliary Lemmas about the solution's presentation and the non-advanced test are also recalled. In Section \ref{Stability}, our first main results about the stability of arbitrarily high-index system are given, making use of both approaches above. 
% In Sections \ref{Stabilization}, we discuss the stabilization problem via the Lyapunov-Krasovskii functional method. 
Finally, in Section \ref{conclusion}, numerical examples and the conclusion are given.

%=================================================================

\section{Preliminaries} \label{pre}

To keep the brevity of this research, we refer the interested readers to \cite{AscP95,Cam95c,ShaG06,Ha15,HaM16} for the solvability analysis of the IVP \eqref{delay-descriptor}.

%
\begin{definition}\label{def3.2}
	The null solution $x=0$ of the free system \eqref{free system} is called \emph{exponentially stable} if there exist positive constants 
	$\de$ and $\ga$ such that for any consistent initial function $\vphi \in C([-\tau,0],\R^n)$, the solution $x=x(t,\vphi)$ of the corresponding 
	IVP to \eqref{free system} satisfies
	%
	\[
	\|x(t)\| \leq \de e^{-\ga t} \|\vphi\|_{\infty}, \ \mbox{ for every } \ t\geq 0.
	\]
	%
\end{definition}
%

%
\begin{definition}\label{regularity} i) Consider the DDAE \eqref{delay-descriptor}. The matrix pair $(E,A_0 )$ is called \emph{regular} if the \emph{polynomial} $\det(\lb E - A_0 )$ is not identically zero. \\
ii) The sets $\si(E,A_0,...,A_m):= \{\lb \in \C \ | \det(\lb E -A_0 - e^{-\lb \tau_i} A_i) \!=\! 0\}$ is called the \emph{spectrum} of \eqref{delay-descriptor}. 
\end{definition}
%

Provided that the pair $(E,A_0 )$ is regular, we can transform them to the Kronecker-Weierstra\ss \ canonical form as follows. 

\begin{lem}\label{KW lemma}(\cite{Dai89,KunM06}) Provided that the matrix pair $(E,A_0 )$ is regular, then there exist regular matrices $W$, $T\in \R^{n,n}$ such that
%
\begin{equation}\label{KW form}
	(W E T, W A_0 T ) = \left(  \m{I & 0 \\ 0 & N } , \ \m{ J & 0 \\ 0 & I} \right) \ ,
\end{equation}
%
where $N$ is a nilpotent, upper triangular matrix of nilpotency index $\nu$. We also say that the pair $(E,A_0 )$ has an \emph{index} $\nu$, i.e., $\ind(E,A_0 ) = \nu$.
Furthermore, the system \eqref{delay-descriptor} is called {\rm impulse-free (index 1, or strangeness-free)} if $N=0$.
\end{lem}

\begin{rem}
In general, the two concepts index and stability are independent. In fact, Examples 5 in \cite{HaN21} has illustrated that there exist systems with arbitrarily high-index (and hence, not impulse-free) which are stable.
\end{rem}

\begin{lem}\label{Neumann inverse}
For a nilpotent, upper triangular matrix $N$ of nilpotency index $\nu$, the matrix $I-\lb N$ is invertible for all $\lb \in \C$, and $\det(I-\lb N) = 1$.
Furthermore, the following identity holds true.
%
\[
(I-\lb N)^{-1} = I + \sum_{i=1}^{\nu} (\lb N)^i. 
\]
%
\end{lem}
\begin{pf}
The proof is simple and can be found in classical matrix theory textbooks, for example \cite{HorJ90}.
\end{pf}

\subsection{System classification}
It is well-known (see e.g. \cite{BelC63,HalL93}) that in general, time-delayed systems has been classified into three different types (retarded, neutral, advanced). For example, the 
time-delayed equation
%
\[
a_0 \dot{x}(t) + a_1\dot{x}(t-\tau) + b_0 x(t) + b_1 x(t-\tau) = f(t)
\]
%
is retarded if $a_0\not= 0$ and $a_1=0$; is neutral if $a_0\not= 0$, $a_1\not= 0$; is advanced if $a_0=0$, $a_1\not= 0$, $b_0 \not=0$. This classification is based on the smoothness comparison between $x(t)$ and $x(t-\tau)$. In literature, not only the theoretical but also the numerical solution has been studied mainly for retarded and neutral systems, due to their appearance in various applications. For this reason, in \cite{Ha15,HaM16,Ung18} the authors proposed a concept of \emph{non-advancedness} for the free system (see Definition \ref{def2} below). 
We also notice, that even though not clearly proposed, due to the author's knowledge, so far results for delay-descriptor are only obtained for certain classes of non-advanced systems, e.g. \cite{AscP95,ShaG06,ZhuP97,ZhuP98,Mic11,Phat14,Sau16,Cui.et.al'17,Ngoc18}.
%
\begin{definition}\label{def2}
	A regular delay-descriptor system \eqref{delay-descriptor} is called \emph{non-advanced} if for any consistent and continuous initial function $\vphi$, there exists a continuous, piecewise differentiable solution $x(t)$.
\end{definition} 
%

%\subsection{A simple check for the non-advancedness}
Making use of Lemma \ref{KW lemma}, we change the variable $x\!=\!Ty$ and scale the whole system \eqref{free system} with $W$ to obtain the transformed system 
%
\begin{equation}\label{eq9}
	\m{I & 0 \\ 0 & N } \dot{y}(t) = \m{ J & 0 \\ 0 & I} y(t) + \sum_{i=1}^{m} \tAi{i} y(t-\tau_i),
\end{equation}
%
where $WA_iT = \tAi{i}$ for all $i=1,...,m$. The following lemma gives us the necessary and sufficient condition for the non-advancedness of system \eqref{free system}.

\begin{lem}\label{lem4}
i) System \eqref{free system} is non-advanced if and only if the matrix coefficients of the transformed system \eqref{eq9} satisfy
	%
	\begin{equation}\label{non-advanced cond.}
		N \m{\tA_{i,3} & \tA_{i,4}} = \m{0 & 0} \ \mbox{ for all } \ i=1,\dots,m.
	\end{equation}
	%
ii) Consequently, system \eqref{eq9} has exactly the same solution as the so-called \emph{index-reduced system} 
%
\begin{equation}\label{index reduced system}	
	\tE  \dot{y}(t) = \tA_0 y(t) + \sum_{i=1}^{m} \tA_i y(t-\tau_i),
\end{equation}
%
where 
%
\begin{equation*}
	\tE := \m{I & 0 \\ 0 & \mathbf{0}}, \ \tA_0:= \m{ J & 0 \\ 0 & I}, \ \tA_i := \tAi{i}, \ i=1,...,m.
\end{equation*}
%
\end{lem}
\begin{pf} 
Partitioning $y:=\sm{y_1 \\ y_2}$ conformably, we can rewrite system \eqref{eq9} as follows 
%
	\begin{align}\label{eq14.2}
	  \dot{y}_1 &= J y_1 + \sum_{i=1}^{m} \m{\tA_{i,1} & \tA_{i,2}} y(t-\tau_i), \notag \\
	N \dot{y}_2 &= y_2 + \sum_{i=1}^{m} \m{\tA_{i,3} & \tA_{i,4}} y(t-\tau_i), 
	\end{align}
%
The second equation has a unique solution
%
\[
y_2(t) = -  \m{\tA_{i,3} & \tA_{i,4}} y(t-\tau_i) - \sum_{j=1}^{\nu} \sum_{i=1}^{m} N^i \m{\tA_{i,3} & \tA_{i,4}} y^{(j)}(t-\tau_i).
\]
%
Since the system \eqref{free system} is non-advanced, then so is system \eqref{eq9}. 
Consequently, $y(t)$ must not depend on $y^{(j)}(t-\tau_i)$ for all $j=1,...,\nu$ and $i=1,...,m$, which implies the identity \eqref{check advanced}.
Then, the second claim is trivially followed.
\end{pf}

\begin{rem}
From Lemma \ref{lem4} ii), we see that if system \eqref{free system} is non-advanced, then there is a linear, bijective mapping $x\mapsto y = T^{-1} x$ (where $T$ is the matrix given in the Kronecker-Weierstra\ss \ form \eqref{KW form}) between the solution set of the high-index system 
\eqref{free system} and the impulse-free system \eqref{index reduced system}. This will play the key role in the stability analysis in Section \ref{Stability}.
\end{rem}

\begin{rem} Since the numerical computation of the Kronecker-Weierstra\ss \ form \eqref{KW form} is quite complicated and unstable (see \cite{Van79}), 
Lemma \ref{lem4} has more theoretical than numerical meaning for checking the non-advancedness of \eqref{free system}. 
Below we will construct another test, which is more practical.  
\end{rem}

Assume that the pair $(E,A_0 )$ is regular with index $\ind(E,A_0 )=\nu$. We want to give a simple check whether the system \eqref{free system} is non-advanced or not. In analoguous to the case of DAEs, see e.g. \cite{BreCP96,KunM06}, we aim to extract the so-called \emph{underlying delay equation} of the form 
%
\begin{equation}\label{underlying DDEs}
	\dot{x}(t) = \bfA_0 x(t) + \sum_{i=1}^{m}\bfA_{i} x(t-\tau_i) + \sum_{i=1}^{m}\bfF_{i} \dot{x}(t-\tau_i),
\end{equation}
%
from an augmented system consisting of system \eqref{free system} and its derivatives, which read in details
%
\[
\dfrac{d^j}{dt^j} \left(  E\dot{x}(t) - A_0 x(t) - \sum_{i=1}^{m} A_i x(t-\tau_i) \right) = 0, \mbox{ for all } j=0,1,\dots,\nu.
\]
%
We rewrite these equations into the so-called \emph{inflated system}
%
\begin{align}\label{inflated}
	& 
	\underbrace{
	\m{E & & & & \\ -A_0 & E & & & \\ & & \ddots & \ddots & \\ & & & -A_0 & E}
		}_{\cE}	
		\m{\dot{x} \\ \ddot{x} \\ \vdots \\ x^{(\nu+1)}}
		= 
	\underbrace{
	\m{A_0 & 0 & \hdots & & 0 \\ 0 & 0 & \hdots & & 0 \\ \vdots & \vdots & & & \vdots \\ 0 & 0 & \hdots &  & 0}
	}_{\cA_0}	
	\m{x \\ \dot{x} \\ \vdots \\ x^{(\nu)}} \notag \\
	& \ + 
	\sum_{i=1}^{m}
	\underbrace{
		\m{A_i & & & \\ & A_i & & \\  & & \ddots & \\ & & & A_i}
	}_{\cA_i}	
	\m{x(t-\tau_i) \\ \dot{x}(t-\tau_i) \\ \vdots \\ x^{(\nu)}(t-\tau_i)} \ .
\end{align}
%
Here the matrix coefficients are $\cE,\cA_0,\cA_i \in \R^{(\nu+1) n,(\nu+1) n}$ for all $i=1,...,m$. For the reader's convenience, below we will use MATLAB notations. An underlying delay system \eqref{underlying DDEs} can be extracted from \eqref{inflated} if and only if there exists a matrix $P=\m{P_0 & P_1 & \dots & P_{\nu}}^T$ in $\R^{(\nu+1) n,n}$ such that
%
\begin{align*}
	P^T \cE &= \m{ I_n \ & 0_{n,\nu n} } , \\
	P^T \cA_i &= \m{* \ & * \ & 0_{n,(\nu-1)n}}, \mbox{ for all } i=1,...,m,
\end{align*}
%
where $*$ stands for an arbitrary matrix. Consequently, $P$ is the solution to the following linear systems
%
\[
\m{\cE^T \\ \cA_1(:,2n+1 : end)^T \\ \vdots \\ \cA_m(:,2n+1 : end)^T} P = \m{ \m{I_n \ & 0_{n,\nu n}}^T \\ 0_{(\nu-1)n,n} \\ \vdots \\ 0_{(\nu-1)n,n}} \ .
\]
%
Therefore, making use of Crammer's rule we directly obtain the simple check for the non-advancedness of system \eqref{free system} in the following theorem.
%
\begin{thm}\label{thm check advancedness}
	Consider the zero-input descriptor system \eqref{free system} 
	and assume that the pair $(E,A_0 )$ is regular with index $\ind(E,A_0 )=\nu$. 
	Then, this system is non-advanced if and only if the following rank condition is satisfied 
	%
	\begin{equation}\label{adv. check eq.}
		\rank \m{\cE^T \\ \cA_1(:,2n\!+\!1 : end)^T \\ \vdots \\ \cA_m(:,2n\!+\!1 : end)^T}
		\!=\! \rank \left[ \! \pm{\cE^T \\ \cA_1(:,2n\!+\!1 : end)^T \\ \vdots \\ \cA_m(:,2n\!+\!1 : end)^T} \vline \pm{ \m{I_n \ & 0_{n,\nu n}}^T \\ 0_{(\nu-1)n,n} \\ \vdots \\ 0_{(\nu-1)n,n}} \! \right] . 
	\end{equation}
	%
\end{thm}
%
Theorem \ref{thm check advancedness} applied to the index two case straightly gives us the following corollary.
%
\begin{coro}\label{coro3}
	Consider the zero-input descriptor system \eqref{free system} and assume that the pair $(E,A_0 )$ is regular with index $\ind(E,A_0 )=2$.  	
	Then, system \eqref{free system} is non-advanced if and only if the following identity hold true.
	%
	\begin{equation}\label{check advanced}
		\rank \m{E^T & -A_0^T & 0 \\ 0 & E^T & -A_0^T \\ 0 & 0 & E^T \\ \hline \\[-.4cm] 0 & 0 & A_1^T \\ \vdots & \vdots & \vdots \\ 0 & 0 & A_m^T} =
		n + \rank \m{E^T & -A_0^T \\ 0 & E^T \\ \hline \\[-.4cm] 0 & A_1^T \\ \vdots & \vdots \\ 0 & A_m^T} \ .
	\end{equation}
	%
\end{coro} 


%=================================================================
\section{Stability}\label{Stability}

\subsection{Spectral method}
The stability analysis of the null solution of (1) in this work is based on a spectrum determined growth property of the
solutions, which allows us to infer stability information from the location of the characteristic roots. For instance,
exponential stability will be related to a strictly negative spectral abscissa (the supremum of the real parts of the
characteristic roots). As we shall see, the spectral abscissa of (1) may not be a continuous function of the delays. Moreover,
this may lead to a situation where infinitesimal delay perturbations destabilise an exponentially stable system.
These properties are very similar to the spectral properties of neutral equations (see, e.g. [2, Section 2]), which are
known to be closely related to DDAEs [3]. 

\begin{prop}\label{Wim11}(\cite{Mic11,DuLMT13})
Consider the linear, homogeneous DDAE \eqref{free system}. Furthermore, assume that it is regular, impulse-free. Then it is stable if and only if the corresponding spectrum of this system lies entirely on the left half plane and it is bounded away from the imaginary axis.
\end{prop}

The following lemma plays the key role in the proof of the main Theorem \ref{Thm3.1} below.

\begin{lem}\label{key lemma}
Consider the linear, homogeneous DDAE \eqref{free system}. Furthermore, assume that it is non-advanced. Then system \eqref{free system} has the same spectrum (without counting multiplicity) as the index-reduced system \eqref{index reduced system}.
\end{lem}
\begin{pf} We will show that both systems \eqref{free system} and \eqref{index reduced system} have the same spectrum (without counting multiplicity) as the 
system \eqref{eq9}. Due to the variable transformation $x=Ty$ and the identity
%
\[
W \left( \lb E -A_0 - e^{-\lb \tau_i} A_i \right) T = \m{I & 0 \\ 0 & N } - \m{ J & 0 \\ 0 & I} - e^{-\lb \tau_i} \tAi{i},  
\]
%
it is straightforward that 
%
\begin{equation}\label{eq11}
\si(E,A_0,...,A_m) \!=\! \si \left( \m{I & 0 \\ 0 & N },\m{ J & 0 \\ 0 & I},\tAi{1},...,\tAi{m} \right).
\end{equation}
%
Now let us consider the right hand side of \eqref{eq11}, due to Lemma \ref{Neumann inverse} we see that for an arbitrary $\lb \in \C$
%
\begin{align*}
& \det\left( \m{I & 0 \\ 0 & N } - \m{ J & 0 \\ 0 & I} - \sum_{i=1}^{m} e^{-\lb \tau_i} \tAi{i} \right) \\
& = \det\left( \m{I & 0 \\ 0 & (I\!-\!\lb N)^{\!-\!1} } \cdot 
\m{I\!-\!J\!-\! \sum_{i=1}^{m} e^{\!-\!\lb \tau_i} \tA_{i,1} & \!-\! \sum_{i=1}^{m} e^{\!-\!\lb \tau_i} \tA_{i,2} \\ 
	\!-\! \sum_{i=1}^{m} e^{\!-\!\lb \tau_i} \tA_{i,3} & \lb N \!-\! I \!-\! \sum_{i=1}^{m} e^{\!-\!\lb \tau_i} \tA_{i,4} } \right) .
\end{align*}
%
Due to Lemma \ref{Neumann inverse}  and the identity \eqref{non-advanced cond.}, we have
%
\begin{align*}
 & (I+\sum_{i=1}^{\nu} (\lb N)^i) \cdot \sum_{i=1}^{m} e^{-\lb \tau_i} \tA_{i,3} =  \sum_{i=1}^{m} e^{-\lb \tau_i} \tA_{i,3}, \\
 & (I+\sum_{i=1}^{\nu} (\lb N)^i) \cdot \left( \lb N - I - \sum_{i=1}^{m} e^{-\lb \tau_i} \tA_{i,4} \right) =  - I - \sum_{i=1}^{m} e^{-\lb \tau_i} \tA_{i,4} .
\end{align*}
%
Hence, it follows that for any $\lb \in \C$
%
\begin{align*}
	& \det\left( \m{I & 0 \\ 0 & N } - \m{ J & 0 \\ 0 & I} - \sum_{i=1}^{m} e^{-\lb \tau_i} \tAi{i} \right) \\
	& = \det \left( 
	\m{I-J- \sum_{i=1}^{m} e^{-\lb \tau_i} \tA_{i,1} & - \sum_{i=1}^{m} e^{-\lb \tau_i} \tA_{i,2} \\ 
		- \sum_{i=1}^{m} e^{-\lb \tau_i} \tA_{i,3} 
		&  - I - \sum_{i=1}^{m} e^{-\lb \tau_i} \tA_{i,4} 
	   }
	\right),
\end{align*}
%
which yields that
%
\begin{equation}\label{eq12}
	\si \left( \m{I & 0 \\ 0 & N },\m{ J & 0 \\ 0 & I},\tAi{1},...,\tAi{m} \right) \!=\! \si(\tE,\tA_0,...,\tA_m) .
\end{equation}
%
From \eqref{eq11} and \eqref{eq12} we have $\si(E,A_0,...,A_m) \!=\! \si(\tE,\tA_0,...,\tA_m)$.
%
\hfill $\square$
\end{pf}

\begin{thm}\label{Thm3.1}
	Consider the free system \eqref{free system}. Furthermore, we assume that the matrix pair $(E,A_0)$ is regular. Then, \eqref{free system} is exponentially stable if and only if the following assertions hold.
	\begin{enumerate}
		\item[i)] System \eqref{free system} is non-advanced. 
		\item[ii)] The spectrum $\si(E,A_0,\dots,A_m)$ lies entirely on the left half plane and it is bounded away from the imaginary axis.  
	\end{enumerate} 
\end{thm}
\begin{pf}
$``\Rightarrow"$ Assume that system \eqref{free system} is exponentially stable. Clearly, it is non-advanced, so we only need to prove ii).
Furthermore, due to Lemma \ref{lem4}ii),  system \eqref{free system} is stable if and only if the index-reduced system \eqref{index reduced system} is also stable. 
Thus, the spectrum $\si(\tE,\tA_0,\dots,\tA_m)$ lies entirely on the left half plane and it is bounded away from the imaginary axis,
and hence, due to Lemma \ref{key lemma} we obtain the desired claim. \\
%
$``\Leftarrow"$ Since the index-reduced system \eqref{index reduced system} is impulse-free, Proposition \ref{Wim11} applied to it implies that the index-reduced system \eqref{index reduced system} is exponentially stable,
and so is system \eqref{free system}. This completes the proof.
\hfill $\square$
\end{pf}

\begin{rem}
Again, we notice that due to the complication in computing the Kronecker-Weierstra\ss \ form \eqref{KW form}, we will not compute the spectrum $\si(E,A_0,\dots,A_m)$ based on \eqref{KW form}. 
Instead, we refer the reader to the  spectral discretisation approach in \cite{Mic11}. Nevertheless, since this method has only been developed for impulse-free (or index-1) system, we need the pre-processing step as in Lemma \ref{lem6} below.
\end{rem}

Let us consider the (reordered) QZ-decomposition (\cite{GolV96}) of the matrix pair $(E,A_0)$ as follows
%
\begin{equation}\label{eq15}
Q E Z^T = \m{\Si_E & \hE_{2} \\ 0 & N_E }, \ Q A_0 Z^T = \m{ J_A & \hA_{2} \\ 0 & \Si_A}, \ Q A_i Z^T = \hAi{i}	,
\end{equation}
%
where $Q$ and $Z$ are orthogonal matrices, $\Si_E$ and $\Si_A$ are nonsingular, upper triangular matrices, $N_E$ is a nilpotent, upper triangular matrix.

Using the same argument as in Lemma \ref{lem4}, we have the following lemma. 

\begin{lem}\label{lem6}
Consider the free system \eqref{free system} and the QZ-decomposition \eqref{eq15}. Then, the following assertions hold true.\\
i) System \eqref{free system} is non-advanced if and only if $N_E \Si_A^{-1} \m{\hA_{i,3} & \hA_{i,4}} = 0$ for all $i=1,...,m$. \\
ii) If this is the case, then there is a linear, bijective mapping $x\mapsto y = Z x$ (where $Z$ is the matrix given in \eqref{eq15}) between the solution set of the high-index system 
\eqref{free system} and the following index-reduced system
%
\begin{equation}\label{impulse free system}
\m{\Si_E & \hE_{2} \\ 0 & \mathbf{0} } \dot{y}(t) = \m{ J_A & \hA_{2} \\ 0 & \Si_A} y(t) + \sum_{i=1}^{m} \hAi{i} y(t-\tau_i) \ .
\end{equation}
%
\end{lem}
\begin{pf}
The proof is essentially the same as the proof of Lemma \ref{lem4} and will be omitted to keep the brevity of this research.
\end{pf}

\begin{example}
To illustrate the advantage of the proposed method, we consider the following system, motivated from \cite{HaiB09}.
%
\begin{align}\label{eq17}
	\notag \m{-1 & 2 & 0.2648\\
		-2 & 4 & 0.8476\\
		0 & 0 & 0
	} \dot{x}(t)
	=& \m{4.7 & 0.4 & 0.1192\\
		-4.9 & 0.8 & 1.1783\\
		0 & 0 & 0.6473
	} x(t) + \m{0.7 & -0.95 & 0.6456\\
		1.1 & -1.75 & 1.7706\\
		0 & 0 & 0
	} x(t-0.2)  \\
	&+  
	\m{1 & -0.8 & 0.6393\\
		1.4 & -1.3 & 1.8234\\
		0 & 0 & 0
	} x(t-2) . 
\end{align}
%
We notice that the matrix pair $(E,A_0)$ in system \eqref{eq17} has index $\nu=2$, and hence the system is not impulse-free. 
Using the MATLAB Toolbox TDS\_STABIL (\cite{Mic10,Mic11}) we obtain the dominant eigenvalues of the original system \eqref{eq17} and that of
the index-reduced system \eqref{impulse free system}. The result is presented in Figure \ref{fig1}. 
Clearly, we see that without the index-reduced step, the spectrum is not properly computed and hence, is not reliable to determine the
stability of system \eqref{eq17}. 

\begin{figure}
	\centering
	\includegraphics[scale = 0.8]{Example13_Ha21}
	\caption{Spectrum of the system \eqref{eq17} (left) and the index-reduced system \eqref{impulse free system} (right), using the MATLAB Toolbox TDS\_STABIL (\cite{Mic10}).}
	\label{fig1}
\end{figure}
\end{example}


\subsection{Lyapunov-Krasovskii functional method}           

Adopting the Lyapunov-Krasovskii approach, (sufficient) stability conditions for many classes of singular systems with different types of delays (single, multiple, time-varying, etc.) have been 
proposed, see for example, \cite{HaiB09,Nor.et.al.21,CheP13,XuDSL02,Saw21,Fri02,Sun03,Ngoc20,Che20}.
We, again, notice that all the conditions on the references mentioned above are only valid for impulse-free system. We recall one important result in the following proposition.

\begin{prop}\label{Xu02}(\cite{Fri02,XuDSL02})
Consider the linear, homogeneous DDAE \eqref{free system}. Furthermore, assume that it is regular, impulse-free. Then it is stable if there exist
matrices $Q_i \succ 0$ and matrices $P_i$, $i=1,...,m$ such that following LMI are satisfied
%
\begin{equation}\label{LMI}
M:= \m{AP^T+PA^T+Q & \vline & A_1 P_1^T & ... 	& A_m P_m^T \\ \hline 
	P_1 A_1^T  & \vline & -Q_1      &     	&           \\
	    \vdots & \vline &           & \ddots &			\\
	P_m A_d^T  & \vline &        	& 		& -Q_m	} 
	\prec 0 \ .
\end{equation}
%
\end{prop}

Similarly, in order to apply these results for arbitrarily-high index system, first we transform system \eqref{free system} to the index-reduced form \eqref{index reduced system}.
We illustrate the advantage of this strategy in the following example.

\begin{example}\label{exam14}
Let us reconsider system \eqref{eq17}, which is not impulse-free and having an index $\nu(E,A)=2$. If we directly apply the MATLAB LMI-Toolbox or the package CVX \cite{cvx,GraB08} to the 
system \eqref{eq17} then the following warning messages appear ``Marginal infeasibility: these LMI constraints may be feasible but are not strictly feasible" or ``Status: Inaccurate/Solved". 
Indeed, the matrix $M$ obtained from LMI-Toolbox/CVX package is not negative definite.  
\end{example}


\begin{rem}\label{rem15}
	In comparison to the stability result obtained in \cite{Cui.et.al'17}, we do not make use of the Drazin inverse, and hence, the computation is stable and more reliable.
\end{rem}


%=================================================================
%\section{Robust stabilization}\label{Stabilization}


%=================================================================
\section{Conclusion and Outlook}\label{conclusion}


%========================================================================================
\vskip 0.2cm
\textbf{Acknowledgment} The author would like to thank the anonymous referee for his suggestions to improve this paper.
%=========================================================================================

%\bibliographystyle{abbrv}
%\bibliographystyle{unsrtnat}
\bibliographystyle{elsarticle-num}
\bibliography{Phi_July_2020}
   
\end{document}
\endinput

%%
%% End of file `elsarticle-template-num.tex'.
