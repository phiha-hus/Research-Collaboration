%Report for the revised version
\NeedsTeXFormat{LaTeX2e}
\documentclass[12pt,onesite]{article} 
\usepackage{amsmath,amsxtra,amssymb,latexsym,amscd}
\hoffset= 0.0 cm
\voffset=-2.0 cm
\textwidth=14 cm
\textheight=23 cm
%\voffset -0.7in
\parskip 4pt

\begin{document}
%\pagestyle{empty}
\setlength{\baselineskip}{16truept}


\begin{center} 
{\large \bf Reply to Referees' Reports}

\end{center}
\vspace{10 mm}

\noindent Title: {\bf Index Reduction for Second Order Singular Systems of Difference Equations}

\noindent Authors: {\bf Vu Hoang Linh and Ha Phi}

\vspace{5 mm}

We would like to thank the referee for careful reading, very useful comments, and suggestions. We have taken all the comments into consideration during the course of revising the manuscript.  In order to make the paper concise and to highlight the main results, we have restructured and rewritten the manuscript substantially.

Our changes and replies are as follows.

\noindent {\bf Major changes}
\begin{itemize}

\item	Compared to the old version, we have modified/removed unnecessary parts in Section 2. In more details, we did shorten the paper by referring to Lemma 2.7 ([8]) for the proof of Lemma 2.3. Furthermore, we moved Lemmas 2.7, 2.8 and Remark 2.9 to Section 4, because they are used only in Sections 4 and 5. Therefore, Algorithm 1 and the first main result (Theorem 3.8) have presented earlier in a clearer way in the revised version. 

\item	In Section 3, Lemma 3.7 to Lemma 3.9 have been combined into one new lemma (Lemma 3.7), in order to bring the main idea to the reader in a faster and clearer way. The proof, which is not too difficult but rather lengthy and technical, has been left to Appendix A. 

\item	In Section 4, we have combined Lemma 4.1 and Proposition 4.3 into one new theorem (Theorem 4.4). This is reasonable, since our result is comparable to Theorem 2.4 (about the condensed form in Loose \& Mehrmann (2008)). Furthermore, not only the condensed form but also the feedback regularization are presented here (part ii).

\item	In Section 5, we have also moved Theorem 5.7 forward in order to highlight the results. The detailed construction to the desired strangeness-form has been presented in the proof of this theorem. Furthermore, the main steps of the construction are still summarized in Algorithm 2.

\item	Concerning the relevance of the topic, in our opinion discrete-time systems play an important role and they frequently appear in real-life applications, as well. In this work, we have not only presented a comparable index theory for linear discrete-time systems, but also improved some recent results for high order continuous-time singular systems.

\item     We have considered very carefully the suggestion {\it ``to first present a self-contained section that summarizes the highlights and advances of this article''}, which is reasonable. However, in our opinion, it would be complicated to present the main results this way. This work contains different problems and approaches with many auxiliary lemmas and notions which should be introduced before the main results. In addition, auxiliary results are important in understanding the two algorithms which are highlights of the paper, too. Therefore, we have restructured the auxiliary results and changed the presentation of the main results substantially as mentioned above. We do hope that now the manuscript in the revised form is concise and presents clearly the main results. 

\end{itemize}
\noindent {\bf Minor changes}
\begin{itemize}

\item	The format “smallmatrix” and also transpose have been used to avoid the row spacing. 

\item	We have checked carefully our use in punctuation marks and capitalization.

\item     We have revised linguistic mistakes and typos.

\end{itemize}
\end{document}