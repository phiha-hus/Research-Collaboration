\section{Proof of Lemma \ref{lem2.10}}\label{App0}

First we prove that any solution to \eqref{eq2.9} is also a solution to \eqref{eq2.11}. Notice that, due to Lemma \ref{lem2.8}, two systems \eqref{eq2.5} and \eqref{eq2.9} have identical solution set. Thus, we only need to prove that \eqref{eq2.9} and \eqref{eq2.11} are equivalent. \\
\textbf{Necessity:} The main idea here is to apply elementary row transformations to system \eqref{eq2.9} to obtain \eqref{eq2.11}. 
Notice that we use only two elementary block row operations: 
%
\begin{enumerate}
\item[i)] scaling a block row equation with a nonsingular matrix,
\item[ii)] add to one row a linear combinations of another rows.
\end{enumerate}
%
Firstly, by scaling the first (resp., second) block row equation of \eqref{eq2.9} with an orthogonal matrix $\m{S^{(2)}_{n} \\ Z^{(2)}_{n}}$ (resp., $\m{S^{(1)}_{n} \\ Z^{(1)}_{n}}$), we obtain an equivalent system to \eqref{eq2.5}, as follows
%
\begin{equation}\label{eqA1}
 \pm{ d_{2} \\ s_2 \\ \hline \\[-0.35cm] d_1 \\ s_1 \\ \hline \\[-0.35cm] r_0 \\ v \\ \hline \\[-0.35cm]  r_0 \\ r_1 \\ r_0 } \
\m{
	S^{(2)}_{n} A_{n,1} & S^{(2)}_{n} B_{n,1}    & S^{(2)}_{n} C_{n,1} 	\\
	Z^{(2)}_{n} A_{n,1} & Z^{(2)}_{n} B_{n,1}    & Z^{(2)}_{n} C_{n,1}  \\ \hline \\[-0.35cm]
	0 			& S^{(1)}_{n} B_{n,2}  	 & S^{(1)}_{n} C_{n,2}  \\
	0 			& Z^{(1)}_{n} B_{n,2}    & Z^{(1)}_{n} C_{n,2} 	\\ \hline \\[-0.35cm]
	0			& 0				 & C_{n,3}     \\
	0 		    & 0              & 0 			\\	\hline 
	0			& C_{n+1,3}					    & 0		     \\
	B_{n+1,2}  	& C_{n+1,2}  			   		& 0	\\
	C_{n+2,3}	& 0				 			    & 0     		
} \m{x(n+2) \\ x(n+1) \\ x(n)} \!=\! 
\m{ S^{(2)}_{n} f_1(n) \\ Z^{(2)}_{n} f_1(n) \\ \hline \\[-0.35cm] S^{(1)}_{n} f_2(n) \\ Z^{(1)}_{n} f_2(n) \\ \hline \\[-0.35cm] f_3(n) \\ f_4(n) \\ \hline f_3(n+1) \\  f_2(n+1) \\ f_3(n+2)		}.
\end{equation}
%		

By adding the seventh row scaled with $Z^{(3)}_{n}$ to the fourth row of \eqref{eqA1} and making use of \eqref{eq2.10a} we obtain the first hidden constraint
%
\begin{equation*}
Z^{(1)}_{n} C_{n,2} x(n) = Z^{(1)}_{n} f_2(n) + Z^{(3)}_{n} f_3(n+1),
\end{equation*}
%
which is exactly the fourth row of \eqref{eq2.11}.

We continue by adding the seventh row scaled with $Z^{(4)}_{n}$ and the eighth row scaled with $Z^{(5)}_{n}$ to the second row of \eqref{eqA1} and making use of \eqref{eq2.10b} to obtain
%
\begin{align*}
&& \left( Z^{(2)}_{n} B_{n,1} + Z^{(4)}_{n} C_{n+1,2} \right) x(n+1) + Z^{(2)}_{n} C_{n,1} x(n) \notag \\ 
&& = Z^{(2)}_{n} f_1(n) + Z^{(4)}_{n} f_2(n+1) + Z^{(5)}_{n} f_3(n+2).
\end{align*}
%
This is exactly the second row of \eqref{eq2.11}. Therefore, any solution to \eqref{eq2.5} is also a solution to \eqref{eq2.11}. \\
%
\textbf{Sufficiency:} Let $x$ be an arbitrary solution to \eqref{eq2.11}. Thus, $x$ is also a solution to the shifted system
%
\begin{align}\label{eqA2}
& \pm{ d_{2} \\ s_2 \\ \hline \\[-0.35cm] d_1 \\ s_1 \\ \hline \\[-0.35cm] r_0 \\ v \\ \hline \\[-0.35cm]  r_0 \\ r_0 } \ 
\m{S^{(2)}_{n} A_{n,1} & S^{(2)}_{n} B_{n,1}      			   & S^{(2)}_{n} C_{n,1} 	\\
	0			& Z^{(2)}_{n} B_{n,1} + Z^{(4)}_{n} C_{n+1,2}  & Z^{(2)}_{n} C_{n,1}  \\ \hline \\[-0.35cm]
	0 			& S^{(1)}_{n} B_{n,2}  	   			   & S^{(1)}_{n} C_{n,2}  \\
	0 			& 0                			   & Z^{(1)}_{n} C_{n,2} 	\\ \hline \\[-0.35cm]
	0			& 0				 			   & C_{n,3}     \\
	0 		    & 0              			   & 0 		\\	\hline 
	0			& C_{n+1,3}					    & 0		     \\
	C_{n+2,3}	& 0				 			    & 0     	
} 
\m{x(n+2) \\ x(n+1) \\ x(n)} \notag \\  
&= \m{ S^{(2)}_{n} f_1(n) \\ Z^{(2)}_{n} f_1(n) + Z^{(4)}_{n} f_2(n+1) + Z^{(5)}_{n} f_3(n+2) \\ \hline \\[-0.35cm] S^{(1)}_{n} f_2(n) \\ Z^{(1)}_{n} f_2(n) + Z^{(3)}_{n} f_3(n+1) \\ \hline f_3(n) \\  f_4(n) 
	\\ \hline f_3(n+1) \\  f_3(n+2)	}, \ \mbox{ for all } n\geq n_0.
\end{align}
%
Since elementary matrix row operations are reversible, we can reverse the transformations performed in the necessity part. Consequently, we see that any solution to \eqref{eqA2} is also a solution to \eqref{eqA1}, and hence, this completes the proof.
%By adding the seventh row scaled with $-Z^{(3)}_{n}$ to the fourth row of \eqref{eqA2} and making use of \eqref{eq2.10a} we get
%%
%\[
%\pm{ d_{2} \\ s_2 \\ \hline \\[-0.35cm] d_1 \\ s_1 \\ \hline \\[-0.35cm] r_0 \\ v \\ \hline \\[-0.35cm]  r_0 \\ r_0 } \ 
%\m{S^{(2)}_{n} A_{n,1} & S^{(2)}_{n} B_{n,1}      			   & S^{(2)}_{n} C_{n,1} 	\\
%	0			& Z^{(2)}_{n} B_{n,1} + Z^{(4)}_{n} C_{n+1,2}  & Z^{(2)}_{n} C_{n,1}  \\ \hline \\[-0.35cm]
%	0 			& S^{(1)}_{n} B_{n,2}  	   			   & S^{(1)}_{n} C_{n,2}  \\
%	0 			& Z^{(1)}_{n} B_{n,2}     			   & Z^{(1)}_{n} C_{n,2} 	\\ \hline \\[-0.35cm]
%	0			& 0				 			   & C_{n,3}     \\
%	0 		    & 0              			   & 0 		\\	\hline 
%	0			& C_{n+1,3}					    & 0		     \\
%	C_{n+2,3}	& 0				 			    & 0     	
%} 
%\m{x(n+2) \\ x(n+1) \\ x(n)} =  
%\]
%%
%\begin{equation}\label{eqA3}
%= \m{ S^{(2)}_{n} f_1(n) \\ Z^{(2)}_{n} f_1(n) + Z^{(4)}_{n} f_2(n+1) + Z^{(5)}_{n} f_3(n+2) \\ \hline \\[-0.35cm] S^{(1)}_{n} f_2(n) \\ Z^{(1)}_{n} f_2(n) \\ \hline f_3(n) \\  f_4(n) 
%	\\ \hline f_3(n+1) \\  f_3(n+2)	}, \ \mbox{ for all } n\geq n_0.
%\end{equation}
%%
%Since the matrix $\m{S^{(1)}_{n} \\ Z^{(1)}_{n}}$ is orthogonal, from the third and fourth block row equations of \eqref{eqA3} we have
%%
%\[
%B_{n,2} x(n+1)  + C_{n,2} x(n) = f_2(n) \ .
%\]
%%
%Shift-forward this equation and also include this into \eqref{eqA3}, we obtain the system
%%
%\[
%\pm{ d_{2} \\ s_2 \\ \hline \\[-0.35cm] r_1 \\ \hline \\[-0.35cm] r_0 \\ v \\ \hline \\[-0.35cm]  r_0 \\ r_0 \\ r_1} \ 
%\m{S^{(2)}_{n} A_{n,1} & S^{(2)}_{n} B_{n,1}      			   & S^{(2)}_{n} C_{n,1} 	\\
%	0			& Z^{(2)}_{n} B_{n,1} + Z^{(4)}_{n} C_{n+1,2}  & Z^{(2)}_{n} C_{n,1}  \\ \hline \\[-0.35cm]
%	0 			& B_{n,2}     			   					   & C_{n,2} 	\\ \hline \\[-0.35cm]
%	0			& 0				 			   				   & C_{n,3}     \\
%	0 		    & 0              			    & 0 		\\	\hline 
%	0			& C_{n+1,3}					    & 0		     \\
%	C_{n+2,3}	& 0				 			    & 0   		 \\  	
%	B_{n+1,2}  	& C_{n+1,2}  			   		& 0			\\
%} 
%\m{x(n+2) \\ x(n+1) \\ x(n)} =  
%\]
%%
%\begin{equation}\label{eqA4}
%= \m{ S^{(2)}_{n} f_1(n) \\ Z^{(2)}_{n} f_1(n) + Z^{(4)}_{n} f_2(n+1) + Z^{(5)}_{n} f_3(n+2) \\ \hline \\[-0.35cm] f_2(n) \\ \hline f_3(n) \\  f_4(n) \\ \hline f_3(n+1) \\  f_3(n+2) \\  f_2(n+1) }, \ \mbox{ for all } n\geq n_0.
%\end{equation}
%% 
%Since elementary matrix row operations are reversible, we see that any solution to \eqref{eqA2} is also a solution to \eqref{eqA1}, and hence, this completes the proof.

%\section{Proof of Theorem \ref{thm2}}\label{App1}
%Within this proof, we only transform system \eqref{eq2.9} using two elementary block row operations: 1st) scaling a block row equation with a nonsingular matrix, and 2nd) add to one row a linear combinations of another rows. Furthermore, we always keep three last block row equations unchanged. Since the proof is quite technical, we will split it in several steps.\\
%A) In this step we will remove the hidden redundancy in the first block row of system \eqref{eq2.9}.
%By scaling the first block equation of \eqref{eq2.9} with $\m{S^{(2)}_{n} \\ Z^{(2)}_{n}}$, we obtain
%%
%\begin{equation}\label{eqA1}
%\pm{ d_{2} \\ s_2 \\ \hline  \\[-0.35cm] r_1 \\ r_0 \\ v \\ \hline r_0 \\ r_1 \\ r_0 } \quad 
%\m{	S^{(2)}_{n} A_{n,1} & S^{(2)}_{n} B_{n,1}    & S^{(2)}_{n} C_{n,1} 	\\
%	Z^{(2)}_{n} A_{n,1} & Z^{(2)}_{n} B_{n,1}    & Z^{(2)}_{n} C_{n,1}  \\ \hline \\[-0.35cm] 
%	0 			&  B_{n,2}  	 &  C_{n,2}  \\
%	0			& 0				 & C_{n,3}     \\
%	0 		    & 0              & 0    \\ \hline \\[-0.35cm]
%	0			& C_{n+1,3}	 	 & 0     \\
%	B_{n+1,2}  	& C_{n+1,2}  	 & 0	\\
%	C_{n+2,3}	& 0				 & 0   } 
%\m{x(n+2) \\ x(n+1) \\ x(n)} \!=\! 
%\m{ S^{(2)}_{n} f_1(n) \\ Z^{(2)}_{n} f_1(n) \\ \hline  \\[-0.35cm] f_2(n) \\ f_3(n) \\ f_4(n) \\ \hline f_3(n+1) \\ f_2(n+1) \\  f_3(n+2) }.  
%\end{equation}
%%
%Add both $Z^{(4)}_{n}$ times the seventh row and $Z^{(5)}_{n}$ times the eighth row to the second row, and make use of \eqref{eq2.10b}, we obtain
%%
%\bens
%&& \left( Z^{(2)}_{n} B_{n,1} + Z^{(4)}_{n} C_{n+1,2} \right) x(n+1) + Z^{(2)}_{n} C_{n,1} x(n) \\ 
%&& =  Z^{(2)}_{n} f_1(n) + Z^{(4)}_{n} f_2(n+1) + Z^{(5)}_{n} f_3(n+2)  \ .
%\eens
%%
%Using the notation in \eqref{eq2.11}, we rewrite our obtained system as follows.
%%
%\be\label{eqA2}
%\pm{ d_{2} \\ s_2 \\ \hline  \\[-0.35cm] r_1 \\ r_0 \\ v \\ \hline r_0 \\ r_1 \\ r_0 } \quad
%\m{S^{(2)}_{n} A_{n,1}  & S^{(2)}_{n} B_{n,1}      					   & S^{(2)}_{n} C_{n,1} 	\\
%	0					& K_{n,1} 									   & H_{n,1}  \\ \hline \\[-0.35cm]
%	0 					& B_{n,2}			  	   					   &  C_{n,2}  \\
%	0					& 0				 			   				   & C_{n,3}     \\
%	0 				    & 0              			   				   & 0 	\\ \hline \\[-0.35cm]
%	0					& C_{n+1,3}	 	 							   & 0     \\
%	B_{n+1,2}  			& C_{n+1,2} 								   & 0	\\
%	C_{n+2,3}			& 0				 							   & 0     } 
%\m{x(n+2) \\ x(n+1) \\ x(n)}
%=
%\m{ S^{(2)}_{n} f_1(n) \\ \gamma(n+2) \\ \hline  \\[-0.35cm] f_2(n) \\ f_3(n) \\ f_4(n) \\ \hline f_3(n+1) \\ f_2(n+1) \\  f_3(n+2) },
%\ee  
%%
%for all $n\geq n_0$. Here $r_2 = d_2+s_2$. \\ 
%B) In this step we will remove the hidden redundancy in the second and third block row of system \eqref{eqA2}. Apply Lemma \ref{lem1.2} to the matrix pair $(K_{n,1}, C_{n+1,3})$ we can find an unitary matrix $\m{S^{(6)}_{n} \\ Z^{(6)}_{n}} \in \r^{s_2,s_2}$ and $Z^{(7)}_{n}$ such that the pair $(S^{(6)}_{n}, C_{n+1,3})$ has no hidden redundancy and 
%%
%\be\label{eqA3}
%Z^{(6)}_{n} K_{n,1} + Z^{(7)}_{n} C_{n+1,3} = 0 \ .
%\ee
%%
%By scaling the second (resp. third) block equations of \eqref{eqA2} with $\m{S^{(1)}_{n} \\ Z^{(1)}_{n}}$ (resp. $\m{S^{(6)}_{n} \\ Z^{(3)}_{6}}$), we have
%%
%\be\label{eqA4}
%\pm{ d_{2} \\ s_{21} \\ s_{20} \\ \hline  \\[-0.35cm] d_1 \\ s_1 \\ \hline  \\[-0.35cm] r_0 \\ v \\ \hline r_0 \\ r_1 \\ r_0 } \quad 
%\m{S^{(2)}_{n} A_{n,1}  & S^{(2)}_{n} B_{n,1}    & S^{(2)}_{n} C_{n,1} 	\\
%	0					& S^{(6)}_{n} K_{n,1} 	 & S^{(6)}_{n} H_{n,1}  \\ 
%	0					& Z^{(6)}_{n} K_{n,1} 	 & Z^{(6)}_{n} H_{n,1} \\ \hline \\[-0.35cm]
%	0 					& S^{(1)}_{n} B_{n,2}  	 & S^{(1)}_{n} C_{n,2}  \\
%	0 					& Z^{(1)}_{n} B_{n,2}    & Z^{(1)}_{n} C_{n,2} 	\\ \hline \\[-0.35cm]
%	0					& 0				 		 & C_{n,3}     \\
%	0 				    & 0              		 & 0 	\\ \hline \\[-0.35cm]
%	0					& C_{n\!+\!1,3}	 	 	 & 0     \\
%	B_{n\!+\!1,2}  		& C_{n\!+\!1,2} 		 & 0	\\
%	C_{n\!+\!2,3}		& 0				 		 & 0     } 
%\m{x(n\!+\!2) \\ x(n\!+\!1) \\ x(n)}
%\!=\!
%\m{ S^{(2)}_{n} f_1(n) \\ S^{(6)}_{n} \gamma(n\!+\!2) \\ Z^{(6)}_{n} \gamma(n\!+\!2) \\ \hline  \\[-0.35cm] S^{(1)}_{n} f_2(n) \\ Z^{(1)}_{n} f_2(n) \\ \hline  \\[-0.35cm] f_3(n) \\ f_4(n) \\ \hline f_3(n\!+\!1) \\ f_2(n\!+\!1) \\  f_3(n\!+\!2) }, 
%\ee  
%%
%where $s_2=s_{21}+s_{20}$. Then, we add the eight block row equation of \eqref{eqA4} scaled with $Z^{(7)}_{n}$ to the third block row equation, and make use of \eqref{eqA3} to have
%%
%\be\label{eqA5}
%Z^{(6)}_{n} H_{n,1} x(n) = Z^{(6)}_{n} \gamma(n\!+\!2) + Z^{(7)}_{n} f_3(n\!+\!1) \ .
%\ee
%%
%Similarly, we add the eight block row equation of \eqref{eqA4} scaled with $Z^{(3)}_{n}$ to the fifth block row equation, and make use of \eqref{eq2.10a} to have
%%
%\be\label{eq2.17}
%Z^{(1)}_{n} C_{n,2} x(n) = Z^{(1)}_{n} f_2(n) + Z^{(3)}_{n} f_3(n\!+\!1) \ .
%\ee
%%
%Therefore, we obtain the system 
%%
%\be\label{eqA7}
%\pm{ d_{2} \\ s_{21} \\ s_{20} \\ \hline  \\[-0.35cm] d_1 \\ s_1 \\ \hline  \\[-0.35cm] r_0 \\ v \\ \hline r_0 \\ r_1 \\ r_0 } \
%\m{S^{(2)}_{n} A_{n,1}  & S^{(2)}_{n} B_{n,1}    & S^{(2)}_{n} C_{n,1} 	\\
%	0					& S^{(6)}_{n} K_{n,1} 	 & S^{(6)}_{n} H_{n,1}  \\ 
%	0					& 0					 	 & Z^{(6)}_{n} H_{n,1} \\ \hline \\[-0.35cm]
%	0 					& S^{(1)}_{n} B_{n,2}  	 & S^{(1)}_{n} C_{n,2}  \\
%	0 					& 0 					 & Z^{(1)}_{n} C_{n,2} 	\\ \hline \\[-0.35cm]
%	0					& 0				 		 & C_{n,3}     \\
%	0 				    & 0              		 & 0 	\\ \hline \\[-0.35cm]
%	0					& C_{n\!+\!1,3}	 	 	 & 0     \\
%	B_{n\!+\!1,2}  		& C_{n\!+\!1,2} 		 & 0	\\
%	C_{n\!+\!2,3}		& 0				 		 & 0     } 
%\m{x(n\!+\!2) \\ x(n\!+\!1) \\ x(n)}
%\!=\!
%\m{ S^{(2)}_{n} f_1(n) \\ S^{(6)}_{n} \gamma(n\!+\!2) \\ Z^{(6)}_{n} \gamma(n\!+\!2) \!+\! Z^{(7)}_{n} f_3(n\!+\!1) \\ \hline  \\[-0.35cm] S^{(1)}_{n} f_2(n) \\ Z^{(1)}_{n} f_2(n) \!+\! Z^{(3)}_{n} f_3(n\!+\!1) \\ \hline  \\[-0.35cm] f_3(n) \\ f_4(n) \\ \hline f_3(n\!+\!1) \\ f_2(n\!+\!1) \\  f_3(n\!+\!2) }.
%\ee  
%%
%Throw out the last three block row equations of this system, we obtain 
%%
%\begin{equation}\label{eqA8}
%\pm{ d_{2} \\ s_{21} \\ s_{20} \\ \hline  \\[-0.35cm] d_1 \\ s_1 \\ \hline  \\[-0.35cm] r_0 \\ v } \
%\m{S^{(2)}_{n} A_{n,1}  & S^{(2)}_{n} B_{n,1}    & S^{(2)}_{n} C_{n,1} 	\\
%	0					& S^{(6)}_{n} K_{n,1} 	 & S^{(6)}_{n} H_{n,1}  \\ 
%	0					& 0					 	 & Z^{(6)}_{n} H_{n,1} \\ \hline \\[-0.35cm]
%	0 					& S^{(1)}_{n} B_{n,2}  	 & S^{(1)}_{n} C_{n,2}  \\
%	0 					& 0 					 & Z^{(1)}_{n} C_{n,2} 	\\ \hline \\[-0.35cm]
%	0					& 0				 		 & C_{n,3}     \\
%	0 				    & 0              		 & 0 } 
%\m{x(n\!+\!2) \\ x(n\!+\!1) \\ x(n)}
%\!=\!
%\m{ S^{(2)}_{n} f_1(n) \\ S^{(6)}_{n} \gamma(n\!+\!2) \\ Z^{(6)}_{n} \gamma(n\!+\!2) \!+\! Z^{(7)}_{n} f_3(n\!+\!1) \\ \hline  \\[-0.35cm] S^{(1)}_{n} f_2(n) \\ Z^{(1)}_{n} f_2(n) \!+\! Z^{(3)}_{n} f_3(n\!+\!1)
%\\ \hline  \\[-0.35cm] f_3(n) \\ f_4(n) } \ ,
%\end{equation}
%%
%for all $n\geq n_0$, which is exactly the desired system \eqref{eq2.13}. Thus, any solution to \eqref{eq1.1} is also a solution to \eqref{eqA8}. \\
%C) Now we need to prove that two systems \eqref{eqA8} and \eqref{eq2.9} have the same solution set. 
%Since all elementary block row operations are reversible, in fact, it is sufficient to prove that any solution to \eqref{eqA8} is also a solution to \eqref{eqA7}.
%
%Since \eqref{eqA8} holds for all $n\geq n_0$, it has the same solution set to the shifted system
%%
%\be\label{eqA9}
%\pm{ d_{2} \\ s_{21} \\ s_{20} \\ \hline  \\[-0.35cm] d_1 \\ s_1 \\ \hline  \\[-0.35cm] r_0 \\ v \\ \hline r_0 \\ r_0 } \
%\m{S^{(2)}_{n} A_{n,1}  & S^{(2)}_{n} B_{n,1}    & S^{(2)}_{n} C_{n,1} 	\\
%	0					& S^{(6)}_{n} K_{n,1} 	 & S^{(6)}_{n} H_{n,1}  \\ 
%	0					& 0					 	 & Z^{(6)}_{n} H_{n,1} \\ \hline \\[-0.35cm]
%	0 					& S^{(1)}_{n} B_{n,2}  	 & S^{(1)}_{n} C_{n,2}  \\
%	0 					& 0 					 & Z^{(1)}_{n} C_{n,2} 	\\ \hline \\[-0.35cm]
%	0					& 0				 		 & C_{n,3}     \\
%	0 				    & 0              		 & 0 	\\ \hline \\[-0.35cm]
%	0					& C_{n\!+\!1,3}	 	 	 & 0     \\
%	C_{n\!+\!2,3}		& 0				 		 & 0     } 
%\m{x(n\!+\!2) \\ x(n\!+\!1) \\ x(n)}
%\!=\!
%\m{ S^{(2)}_{n} f_1(n) 														\\ 
%	S^{(6)}_{n} \gamma(n\!+\!2) 											\\ 
%	Z^{(6)}_{n} \gamma(n\!+\!2) \!+\! Z^{(7)}_{n} f_3(n\!+\!1) 				\\ \hline  \\[-0.35cm]
%	S^{(1)}_{n} f_2(n) 														\\ 
%	Z^{(1)}_{n} f_2(n) \!+\! Z^{(3)}_{n} f_3(n\!+\!1) 						\\ \hline  \\[-0.35cm] 
%	f_3(n) 																	\\ 
%	f_4(n) 																	\\ \hline 
%	f_3(n\!+\!1)															 \\ 													 
%	f_3(n\!+\!2) }.
%\ee  
%%
%Subtract the eighth row scaled with $Z^{(3)}_{n}$ to the fifth row, and make use of \eqref{eq2.10a}, we have that 
%%
%\[
%Z^{(1)}_{n} B_{n,2} x(n+1) + Z^{(1)}_{n} C_{n,2} x(n) = Z^{(1)}_{n} f_2(n),
%\]
%%
%and hence, \eqref{eqA9} becomes
%%
%\be\label{eqA10}
%\pm{ d_{2} \\ s_{21} \\ s_{20} \\ \hline  \\[-0.35cm] d_1 \\ s_1 \\ \hline  \\[-0.35cm] r_0 \\ v \\ \hline r_0 \\ r_0 } \
%\m{S^{(2)}_{n} A_{n,1}  & S^{(2)}_{n} B_{n,1}    & S^{(2)}_{n} C_{n,1} 	\\
%	0					& S^{(6)}_{n} K_{n,1} 	 & S^{(6)}_{n} H_{n,1}  \\ 
%	0					& 0					 	 & Z^{(6)}_{n} H_{n,1} \\ \hline \\[-0.35cm]
%	0 					& S^{(1)}_{n} B_{n,2}  	 & S^{(1)}_{n} C_{n,2}  \\
%	0 					& Z^{(1)}_{n} B_{n,2} 	 & Z^{(1)}_{n} C_{n,2} 	\\ \hline \\[-0.35cm]
%	0					& 0				 		 & C_{n,3}     \\
%	0 				    & 0              		 & 0 	\\ \hline \\[-0.35cm]
%	0					& C_{n\!+\!1,3}	 	 	 & 0     \\
%	C_{n\!+\!2,3}		& 0				 		 & 0     } 
%\m{x(n\!+\!2) \\ x(n\!+\!1) \\ x(n)}
%\!=\!
%\m{ S^{(2)}_{n} f_1(n) 														\\ 
%	S^{(6)}_{n} \gamma(n\!+\!2) 											\\ 
%	Z^{(6)}_{n} \gamma(n\!+\!2) \!+\! Z^{(7)}_{n} f_3(n\!+\!1) 				\\ \hline  \\[-0.35cm]
%	S^{(1)}_{n} f_2(n) 														\\ 
%	Z^{(1)}_{n} f_2(n)														\\ \hline  \\[-0.35cm] 
%	f_3(n) 																	\\ 
%	f_4(n) 																	\\ \hline 
%	f_3(n\!+\!1)															 \\ 													 
%	f_3(n\!+\!2) 
%}.
%\ee
%%
%Since $\m{S^{(1)}_{n} \\ Z^{(1)}_{n}}$ is unitary, we can compress the fourth and fifth block row equations of \eqref{eqA10} to achieve
%%
%\[
%\pm{ d_{2} \\ s_{21} \\ s_{20} \\ \hline  \\[-0.35cm] r_1 \\ \hline  \\[-0.35cm] r_0 \\ v \\ \hline r_0 \\ r_0 } \
%\m{S^{(2)}_{n} A_{n,1}  & S^{(2)}_{n} B_{n,1}    & S^{(2)}_{n} C_{n,1} 	\\
%	0					& S^{(6)}_{n} K_{n,1} 	 & S^{(6)}_{n} H_{n,1}  \\ 
%	0					& 0					 	 & Z^{(6)}_{n} H_{n,1} \\ \hline \\[-0.35cm]
%	0 					& B_{n,2}  	 			 & C_{n,2} 	\\ \hline \\[-0.35cm]
%	0					& 0				 		 & C_{n,3}     \\
%	0 				    & 0              		 & 0 	\\ \hline \\[-0.35cm]
%	0					& C_{n\!+\!1,3}	 	 	 & 0     \\
%	C_{n\!+\!2,3}		& 0				 		 & 0     } 
%\m{x(n\!+\!2) \\ x(n\!+\!1) \\ x(n)}
%\!=\!
%\m{ S^{(2)}_{n} f_1(n) 														\\ 
%	S^{(6)}_{n} \gamma(n\!+\!2) 											\\ 
%	Z^{(6)}_{n} \gamma(n\!+\!2) \!+\! Z^{(7)}_{n} f_3(n\!+\!1) 				\\ \hline  \\[-0.35cm]
%	f_2(n) 																	\\ 	 \hline  \\[-0.35cm] 
%	f_3(n) 																	\\ 
%	f_4(n) 																	\\ \hline 
%	f_3(n\!+\!1)															 \\ 													 
%	f_3(n\!+\!2) 
%}.
%\]
%%
%Finally, by shifting the fourth block row equation, we then have exactly the system \eqref{eqA7}, and this completes the proof.