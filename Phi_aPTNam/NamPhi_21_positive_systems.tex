%
% This is a general template file for the LaTeX package SVJour3
% for Springer journals.          Springer Heidelberg 2010/09/16
%
% Copy it to a new file with a new name and use it as the basis
% for your article. Delete % signs as needed.
%
% This template includes a few options for different layouts and
% content for various journals. Please consult a previous issue of
% your journal as needed.
%
%%%%%%%%%%%%%%%%%%%%%%%%%%%%%%%%%%%%%%%%%%%%%%%%%%%%%%%%%%%%%%%%%%%
%
%
%\documentclass{svjour3}                     % onecolumn (standard format)
%\documentclass[smallcondensed]{svjour3}     % onecolumn (ditto)
\documentclass[smallextended]{svjour3}       % onecolumn (second format)
%\documentclass[twocolumn]{svjour3}          % twocolumn
%
\smartqed  % flush right qed marks, e.g. at end of proof
%
\usepackage{amsmath,amssymb}

%% In order to avoid the cònliction LaTeX Error: Command \proof already defined.
\let\proof\relax 
\let\endproof\relax
%%
\usepackage{amsthm}
\usepackage{subeqnarray}
\usepackage[backref=page]{hyperref}
%\usepackage[square, numbers, comma, sort&compress]{natbib}
\usepackage[authoryear]{natbib}
\usepackage{graphicx}
\usepackage[active]{srcltx}
%\usepackage{epstopdf} 

\usepackage{lineno}
\linenumbers

\usepackage[right]{showlabels}
%
% \usepackage{mathptmx}      % use Times fonts if available on your TeX system
%
% insert here the call for the packages your document requires
%\usepackage{latexsym}
% etc.
%
% please place your own definitions here and don't use \def but
% \newcommand{}{}
%
% Insert the name of "your journal" with
% \journalname{myjournal}
%
\newcommand {\rank}     {\mathop{\rm rank}\nolimits}
\newcommand {\corank}   {\mathop{\rm corank}\nolimits}
\newcommand {\range}  {\mathop{\rm range}\nolimits}
\newcommand {\corange}  {\mathop{\rm corange}\nolimits}
\newcommand {\kernel}   {\mathop{\rm kernel}\nolimits}
\newcommand {\cokernel} {\mathop{\rm cokernel}\nolimits}
\newcommand {\basis}    {\mathop{\rm basis}\nolimits}
\newcommand {\sigmin}   {\mathop{\sigma_{\rm min}}\nolimits}
\newcommand {\ind}      {\mathop{\rm ind}\nolimits}
\newcommand {\opt}      {\mathop{\rm opt}\nolimits}
\newcommand{\diag}{\mbox{\rm diag}}
\newcommand{\upt}{\mbox{\rm up}}
\newcommand{\re}{\mbox{\rm Re}}
\newcommand{\im}{\mbox{\rm Im}}
\newcommand{\dps}{\displaystyle}
\newcommand{\sct} {{\,\stackrel{c}{\sim}\,}}
\newcommand{\sue} {\,{\stackrel{u}{\sim}\,}}
\newcommand{\suc} {\,{\stackrel{uc}{\sim}\,}}
%\renewcommand{\comment}[1]{}
%\renewcommand{\proof}{\par\noindent{\bf Proof}. \ignorespaces}
%\newcommand{\eproof}{\space
%	{\ \vbox{\hrule\hbox{\vrule height1.3ex\hskip0.8ex\vrule}\hrule}} \ignorespaces} %\\[0.2cm]}
%
%=======================================================================================
% new def-s and commands
%\include{HaMe12_Feb18_command}

\def\bbI{\mathbb{I}}
\def\bbL{\mathbb{L}}
\def\bbM{\mathbb{M}}

\def\om{\omega}
\def\leq{\leqslant}
\def\rar{\rightarrow}
\def\Rar{\Rightarrow}
\def\td{\Leftrightarrow}
\def\r{\hro{R}}
\def\C{\hro{C}}
\def\hro{\mathbb}
\def\N{\hro{N}}

\def\a{\alpha}
\def\b{\beta}
\def\B{\mathcal B}
\def\lb{\lambda}
\def\Lb{\Lambda}
\def\vphi{\varphi}
\def\de{\delta}
\def\De{\Delta}
\def\ga{\gamma}
\def\Si{\Sigma}
\def\si{\sigma}
\def\ka{\kappa}
\def\tka{\tilde{\kappa}}
\def\CE{\mathcal{E}}
\def\tE{\tilde{E}}
\def\hE{\hat{E}}
\def\tA{\tilde{A}}
\def\hA{\hat{A}}
\def\bA{\breve{A}}
\def\tB{\tilde{B}}
\def\tD{\tilde{D}}
\def\hB{\hat{B}}
\def\hr{\hat{r}}
\def\hv{\hat{v}}
\def\tr{\tilde{r}}
\def\trho{\tilde{\rho}}

\def\nab{\nabla}

\def\cB{\mathcal B}
\def\cM{\mathcal M}
\def\tcM{\tilde{\mathcal M}}
\def\cN{\mathcal N}
\def\cX{\mathcal X}
\def\cS{\mathcal S}
\def\tU{\tilde{U}}
\def\cU{\mathcal U}
\def\cL{\mathcal L}

\def\tN{\tilde{N}}
\def\hN{\hat{N}}
\def\tk{\tilde{k}}
\def\hk{\hat{k}}
\def\tx{\tilde{x}}
\def\tX{\tilde{X}}
\def\hX{\hat{X}}
\def\tY{\tilde{Y}}
\def\hY{\hat{Y}}
\def\ty{\tilde{y}}
\def\tv{\tilde{v}}
\def\tw{\tilde{w}}

\def\tM{\tilde{M}}
\def\tm{\tilde{m}}
\def\bM{\breve{M}}
\def\hM{\hat{M}}
\def\bm{\breve{m}}

\def\tC{\tilde{C}}
\def\hC{\hat{C}}
\def\hD{\hat{D}}

\def\tH{\tilde{H}}
\def\tF{\tilde{F}}
\def\tG{\tilde{G}}
\def\hG{\hat{G}}
\def\cG{{\cal G}}
\def\baf{\bar{f}}
\def\tg{\tilde{g}}
\def\hg{\hat{g}}
\def\tK{\tilde{K}}

\def\cW{{\cal W}}

\def\tS{\tilde{S}}
\def\tZ{\tilde{Z}}

\def\tx{\tilde{x}}
\def\tf{\tilde{f}}
\def\hf{\hat{f}}
\def\brf{\breve{f}}
\def\bX{\breve{X}}

\def\chA{\widecheck{A}}
\def\chB{\widecheck{B}}
\def\chC{\widecheck{C}}
\def\chD{\widecheck{D}}
\def\chG{\widecheck{G}}
\def\chU{\widecheck{U}}

\def\lsim{\overset{\ell}{\sim}}
% The inverse shift operator
\def\ide{\Delta_{-1}}


\def\be{\begin{equation}}
\def\ee{\end{equation}}         

\newcommand{\ben}{\begin{eqnarray}}
\newcommand{\een}{\end{eqnarray}}

\newcommand{\bsen}{\begin{subeqnarray}}
\newcommand{\esen}{\end{subeqnarray}}

\newcommand{\bens}{\begin{eqnarray*}}
\newcommand{\eens}{\end{eqnarray*}}

\def\bc{\begin{cases}}
\def\ec{\end{cases}}

\newcommand{\bsq}{\begin{subequations}}
\newcommand{\esq}{\end{subequations}}

\newcommand{\m}[1]{
	\begin{bmatrix}
		#1 
	\end{bmatrix}
}

\renewcommand{\pm}[1]{
	\begin{matrix}
		#1 
	\end{matrix}
}

\newcommand{\n}[1]{
	\|	#1 \|
}

\def\Px{P_{\mathrm{x}}}
\def\Py{P_{\mathrm{y}}}

\def\BEA{\cB^{EA}(\r_+,\r^n)}
%%Useful abbreviations

\def\be{\begin{equation}}
\def\ee{\end{equation}}         
\newcommand{\ben}{\begin{eqnarray}}
\newcommand{\een}{\end{eqnarray}}
\newcommand{\bsen}{\begin{subeqnarray}}
\newcommand{\esen}{\end{subeqnarray}}
\newcommand{\bens}{\begin{eqnarray*}}
\newcommand{\eens}{\end{eqnarray*}}
\def\bc{\begin{cases}}
\def\ec{\end{cases}}
\newcommand{\bsq}{\begin{subequations}}
\newcommand{\esq}{\end{subequations}}

\newcommand{\m}[1]{
\begin{bmatrix}
 #1
\end{bmatrix}
}

\newcommand {\rank}     {\mathop{\rm rank}\nolimits}
\newcommand {\corank}   {\mathop{\rm corank}\nolimits}
\newcommand {\range}  {\mathop{\rm range}\nolimits}
\newcommand {\corange}  {\mathop{\rm corange}\nolimits}
\newcommand {\kernel}   {\mathop{\rm kernel}\nolimits}
\newcommand {\cokernel} {\mathop{\rm cokernel}\nolimits}
\newcommand {\diag}     {\mathop{\rm diag}\nolimits}

\def\cal{\mathcal}
\def \ud{\underline }
\def\id{{\indent }}
\def\cf{\cfrac}
\def\f{\frac}
\def\non{{\noindent}}
\def\leq{\leqslant} 
\def\rar{\rightarrow}
\def\Rar{\Rightarrow}
\def\ti{\times}
\def\r{\hro{R}}
\def\e{\cal{E}}
\def\de{\delta}
\def\ep{\varepsilon}
\def\Ep{\epsilon}
\def\De{\Delta}
\def\ift{\infty}
\def\hro{\mathbb}
\def\ho{\mathcal}
\def\E{\mathcal{E}}
\def\A{\mathcal{A}}
\def\N{\mathbb{N}}
\def\vk{\vskip 0.2cm}
\def\con{\subset}
\def\Con{\subseteq}
\def\td{\Leftrightarrow}
\def\df{\frac}
\def\to{\mapsto}
\def\om{\omega}
\def\a{\alpha}
\def\hA{\hat{A}}
\def\tA{\tilde{A}}
\def\dA{\delta A}
\def\lb{\lambda}
\def\to{\mapsto}
\def\a{\alpha}
\def\b{\beta}
\def\ga{\gamma}
\def\Ga{\Gamma}
\def\Si{\Sigma}
\def\dA{\delta A}
\def\lb{\lambda}
\def\Lb{\Lambda}
\def\tE{\tilde{E}} 
\def\tA{\tilde{A}} 
\def\tB{\tilde{B}}
\def\tM{\tilde{M}}  
\def\tb{\bar{t}}
\def\ub{\bar{u}} 
\def\taub{\bar{\tau}} 
\def\hE{\hat{E}}
\def\hB{\hat{B}}
\def\hP{\hat{P}}
\def\hQ{\hat{Q}}
\def\DD{\mathbb{D}}
\def\tf{\tilde{f}}
\def\tx{\tilde{x}}
\def\A{\cal{A}}
\def\B{\cal{B}}
\def\C{\cal{C}}
\def\D{\cal{D}}
\def\F{\cal{F}}
\def\R{\cal{R}}
\def\K{\cal{K}}
\def\M{\cal{M}}
\def\P{\cal{P}}
\def\Q{\cal{Q}}
\def\NN{\cal{N}}
\def\vco{\vartheta}
\def\tPhi{\tilde{\Phi}}
%%%%%%%%%%%%%%%%%%%%%%%%%%%%%%%%%%%%%%%%%%%%%%%%%%%%%%%%%%%%%%%%%%%
%%%%%%%%%%%%%%%%%%%%%%%%%%%%%%%%%%%%%%%%%%%%%%%%%%%%%%%%%%%%%%%%%%%
%%%%%%%%%%%%%%%%%%%%%%%%%%%%%%%%%%%%%%%%%%%%%%%%%%%%%%%%%%%%%%%%%%%
\def\coo{\mathcal{CO_{ODE}}}
\def\cod{\mathcal{CO_{DAE}}}
\def\umu{\hat{u}_{\hat{\mu}}}
\def\amu{\hat{a}_{\hat{\mu}}}
\def\pmu{\hat{\phi}_{\hat{\mu}}}
\def\dmu{\hat{d}_{\hat{\mu}}}
\def\omu{\hat{\omega}_{\hat{\mu}}}
%=========================================================================================
\def\tdB{ \col{-\hA_{12}(\tau)\hA_{22}^{-1}(\tau) f_2(\tau) + f_1(\tau)}{0} }
\def\td2B{ \mat{0}{0}{I_d}{-\hA_{12}(\tau)\hA_{22}^{-1}(\tau)} \col{f_1(\tau)}{f_2(\tau)}}
\def\tdC{ \mat{I_d}{0}{-\hA_{22}^{-1}\hA_{21}}{0} }
\def\tdD{ \col{0}{-\hA_{22}^{-1}f_2} }
\def\td2D{ \mat{0}{0}{0}{-\hA_{22}^{-1}(t)} \col{f_1(t)}{f_2(t)} }
%=========================================================================================
\newcommand{\h}[1]{\hat{#1}}
\newcommand{\til}[1]{\tilde{#1}}
\newcommand{\mat}[4]{
\begin{bmatrix}
#1   & #2  \\
#3   & #4
\end{bmatrix}
}

\newcommand{\matt}[6]{
\begin{bmatrix}
#1   & #2 & #3 \\
#4   & #5 & #6 \\
0    & 0  &  0
\end{bmatrix}
}

\newcommand{\col}[2]{
\begin{bmatrix}
#1\\
#2
\end{bmatrix}
}

\newcommand{\coll}[3]{
\begin{bmatrix}
#1\\
#2\\
#3
\end{bmatrix}
}

\newcommand{\colll}[4]{
\begin{bmatrix}
#1\\
#2\\
#3\\
#4
\end{bmatrix}
}

\newcommand{\ccoll}{
\begin{matrix}
s\\
d\\
a\\
s\\
\phi\\
u^l
\end{matrix}
}

\renewcommand{\ddot}[1]{\overset{..}{#1}}

\def\derxx{\col{\dot{x}}{\ddot{x}}}

\newcommand{\mattB}{
\begin{bmatrix}
B_{11}   & B_{12} & 0 & 0 \\
\dot{B}_{21} & \dot{B}_{22} &  B_{21} & B_{22}
\end{bmatrix}
}

\newcommand{\matKM}[7]{
\begin{bmatrix}
#1   & #2 &  0    & #3    \\
0   &  #4  &  0    & #5     \\
0   & 0   &  #6   & 0      \\
#7  & 0   &  0    & 0      \\
0   & 0   &  0    & 0
\end{bmatrix}
}

\newcommand{\EAF}[8]{
\begin{bmatrix}
#1  & 0    &  #2   & #3  &\vline       & 0  & #4    \\
\hline
0   & #5   &  0    & 0   &\vline       & 0  & #6       \\
#7  & 0    &  0    & 0   &\vline       & #8 & 0        \\
0   & 0    &  0    & 0   &\vline      & 0  & 0        \\
\end{bmatrix}
}

\def\EO{\EAF{I_{\dmu}}{0}{0}{0}{0}{0}{0}{0}}
\def\AO{\EAF{0}{A_{13}}{A_{14}}{B_{12}}{I_{\amu}}{B_{22}}{A_{31}}{I_{\pmu}}}

\def\ET{
\begin{bmatrix}
I_{\phi_1}   & 0                 &\vline & 0    &  0   & 0   &\vline       & 0  & 0    \\
0            & I_{\dmu - \phi_1} &\vline & 0    &  0   & 0   &\vline       & 0  & 0      \\
\hline
0   & 0      &\vline &  0                 & 0  & 0   &\vline                & 0  & 0     \\
\hline
0   & 0      &\vline &  0                 & 0  & 0   &\vline                & 0  & 0     \\
0   & 0      &\vline &  0                 & 0  & 0   &\vline                & 0  & 0     \\
\hline
0   & 0      &\vline &  0                 & 0  & 0   &\vline                & 0  & 0     \\
\end{bmatrix}
}

\def\AT{
\begin{bmatrix}
0            & 0       &\vline          & 0          &  \hA_{14}   & \hA_{15}   &\vline       & 0  & \hB_{12}    \\
0            & 0       &\vline          & 0          &  \hA_{24}   & \hA_{25}   &\vline       & 0  & \hB_{22}      \\
\hline
0            & 0       &\vline          & I_{\amu}   & 0         & 0            &\vline       & 0  & \hB_{32}     \\
\hline
\Si          & 0       &\vline          &  0         & 0         & 0         &\vline      & \hB_{41}  & 0     \\
0            & 0       &\vline          &  0         & 0         & 0         &\vline      & \hB_{51}  & 0     \\
\hline
0            & 0       &\vline          &  0         & 0         & 0         &\vline      & 0  & 0     \\
\end{bmatrix}
}

\def\ETT{
\begin{bmatrix}
I_{\pmu}   & 0                 &\vline & 0    &  0   & 0   &\vline       & 0  & 0    \\
\hline
0            & I_{\dmu - \pmu} &\vline & 0    &  0   & 0   &\vline       & 0  & 0      \\
0   & 0      &\vline &  0                 & 0  & 0   &\vline                & 0  & 0     \\
0   & 0      &\vline &  0                 & 0  & 0   &\vline                & 0  & 0     \\
\end{bmatrix}
}

\def\ATT{
\begin{bmatrix}
0            & 0       &\vline          & 0          &  \hA_{14}   & \hA_{15}   &\vline       & 0  & \hB_{12}    \\
\hline
0            & 0       &\vline          & 0          &  \hA_{24}   & \hA_{25}   &\vline       & 0  & \hB_{22}      \\
0            & 0       &\vline          & I_{\amu}   & 0         & 0            &\vline       & 0  & \hB_{32}     \\
\Si          & 0       &\vline          &  0         & 0         & 0            &\vline       & \hB_{41}  & 0     \\
\end{bmatrix}
}

\def\ETH{
\begin{bmatrix}
0   & 0                  &\vline &  0    & 0    & 0   &\vline       & 0  & 0     \\
\hline
0   & I_{\dmu - \pmu}    &\vline &  0    &  0   & 0   &\vline       & 0  & 0      \\
0   & 0                  &\vline &  0    &  0   & 0   &\vline       & 0  & 0      \\
0   & 0                  &\vline &  0    & 0    & 0   &\vline       & 0  & 0     \\
\end{bmatrix}
}

\def\ATH{
\begin{bmatrix}
0            & 0       &\vline          & 0          &  \hA_{14}   & \hA_{15}     &\vline       & 0  & \hB_{12}    \\
\hline
0            & 0       &\vline          & 0          &  \hA_{24}   & \hA_{25}     &\vline       & 0  & \hB_{22}      \\
0            & 0       &\vline          & I_{\amu}   & 0           & 0            &\vline       & 0  & \hB_{32}     \\
\Si          & 0       &\vline          &  0         & 0           & 0            &\vline       & \hB_{41}  & 0     \\
\end{bmatrix}
}

\def\EF{
\begin{bmatrix}
I_{\dmu - \pmu}    &\vline & 0    & 0    & 0   &\vline       & 0  & 0      \\
\hline
0                  &\vline &  0   & 0    & 0   &\vline       & 0  & 0     \\
0                  &\vline &  0   & 0    & 0   &\vline       & 0  & 0      \\
0                  &\vline &  0   & 0    & 0   &\vline       & 0  & 0     \\
\end{bmatrix}
}

\def\AF{
\begin{bmatrix}
0       &\vline          & 0         & 0          & \hat{\hA}_{13}     &\vline       & 0  & \hB_{22}      \\
\hline
0       &\vline          & 0         & 0          & I_{\pmu}           &\vline       & 0  & \hat{\hB}_{12}    \\
0       &\vline          & 0         & I_{\amu}   & 0                  &\vline       & 0  & \hB_{32}     \\
0       &\vline          & I_{\pmu}  &  0         & 0                  &\vline       & \hat{\hB}_{41}  & 0     \\
\end{bmatrix}
}

\def\matE{\mat{\hE_{11}}{\hE_{12}}{0}{\hE_{22}}}
\def\matA{\mat{\hA_{11}}{\hA_{12}}{\hA_{21}}{\hA_{22}}}

\def\matB{ 
\begin{bmatrix}
B_1 & 0\\
B_2 & 0\\
\dot{B}_1 & B_1\\
\dot{B}_2 & B_2
\end{bmatrix}
}

\def\derE{
\begin{bmatrix}
E_1 & 0\\
0 & 0\\
\dot{E}_1-A_1 & E_1 \\
-A_2 & 0
\end{bmatrix}
}

\def\derA{
\begin{bmatrix}
A_1 & 0\\
A_2 & 0\\
\dot{A}_1 & 0 \\
\dot{A}_2 & 0
\end{bmatrix}
}

\def\heE{\matt{\hE_{11}(t)}{\hE_{12}(t)}{\hE_{13}(t)}{0}{0}{0}}
\def\hee{\matt{I_{d_{\mu}}}{0}{0}{0}{0}{0}}
\def\heA{\matt{\hA_{11}(t)}{\hA_{12}(t)}{\hA_{13}(t)}{\hA_{21}(t)}{\hA_{22}(t)}{\hA_{23}(t)}}
\def\hea{\matt{0}{0}{\hA_{13}(t)}{0}{I_{a_{\mu}}}{0}}

\def\EKM{\matKM{I_s}{0}{0}{I_d}{0}{0}{0}}
\def\EKMT{\matKM{0}{0}{0}{I_d}{0}{0}{0}}
\def\AKM{\matKM{0}{A_{12}}{A_{14}}{0}{A_{24}}{I_a}{I_s}}


\def\TET{
\begin{bmatrix}
I_s & 0      &  0                & 0    & 0                         \\
0   &I_d     &  0                & 0    & 0                             \\

0   & 0      &  0                & 0    & 0                             \\
0   & 0      &  0                & 0    & 0                             \\

0   & 0      &  0                & 0    & 0                             \\
0   & 0      &  0                & 0    & 0                           
\end{bmatrix}
}

\def\TAT{
\begin{bmatrix}
0            & A_{12}            & 0          &  A_{14}   & A_{15}            \\
0            & 0                 & 0          &  A_{24}   & A_{25}         \\

0            & 0                 & I_a        & 0         & 0               \\
I_s          & 0                 &  0         & 0         & 0               \\

0            & A_{52}            &  0         & 0         & 0              \\
0            & 0                 &  0         & 0         & 0              
\end{bmatrix}
}

\def \TBT{
\begin{bmatrix}
0  & B_{12}       \\  
0  & B_{22}       \\
0  & B_{32}     \\
0  & 0          \\
I_{\phi}  & 0     \\
0  & 0     
\end{bmatrix}
}

\def \TCT
{\begin{bmatrix}
0 & 0 & 0 & I_{\om} & 0 \\
C_{21}   &  C_{22} & C_{23} & 0 & 0    
\end{bmatrix}}

\def\TETT{
\begin{bmatrix}
0   & 0      &  0                & 0    & 0                         \\
0   &I_d     &  0                & 0    & 0                             \\

0   & 0      &  0                & 0    & 0                             \\
0   & 0      &  0                & 0    & 0                             \\

0   & 0      &  0                & 0    & 0                             \\
0   & 0      &  0                & 0    & 0                           
\end{bmatrix}
}

\def \TCTT
{\begin{bmatrix}
0 & 0 & 0 & I_{\om} & 0 \\
0 &  C_{22} & C_{23} & 0 & 0    
\end{bmatrix}}

\def\exu{
\begin{bmatrix}
1   &  0  \\
0   &  1  \\
0    & 0  
\end{bmatrix}
}

\def\axu{
\begin{bmatrix}
a_{11}   &  a_{12}  \\
a_{21}   &  a_{22}  \\
0    & 0  
\end{bmatrix}
}

\def\LI{
\begin{bmatrix}
I_{nN}& \vline & 0 & 0  \\ \hline
 0    & \vline & 0 & 0  \\
 0    & \vline & 0 & 0  \\
\end{bmatrix}
}

\newcommand{\RI}{
\begin{bmatrix}
\h{\A}(t) & \vline & \h{\B}(t)    & 0                  \\ \hline 
\h{\C}(t) & \vline & \F(t)    & -I_{mN}   \\
0     & \vline & -I_{pN}+ \NN(t)\R(t)  & \K(t)+  \NN(t)\P(t)   \\
\end{bmatrix}
}

\def\LIC{
\begin{bmatrix}
I_{nN}& \vline & 0 & 0 & 0  \\ \hline 
 0    & \vline & 0 & 0 & 0 \\
 0    & \vline & 0 & 0 & 0 \\
 0    & \vline & 0 & 0 & 0 
\end{bmatrix}
}

\newcommand{\RIC}{
\begin{bmatrix}
\h{\A}(t) & \vline & \h{\B}(t)    & 0       &  0           \\ \hline 
\h{\C}(t) & \vline & \F(t)    & -I_{mN} &  0      \\
0     & \vline & -I_{pN}  & \K(t)   & \NN(t)    \\
0     & \vline & \R(t)    & \P(t)   & -I_{\h{p}N} 
\end{bmatrix}
}

\newcommand{\RICC}{
\begin{bmatrix}
\h{\A}(t) & \vline & \h{\B}(t)    & 0       &  0           \\ \hline 
\h{\C}(t) & \vline & \F(t)    & -I_{mN} &  0      \\
0     & \vline & -I_{pN}+\R(t)\NN(t)  & \K(t)+ \P(t) \NN(t)   & 0    \\
0     & \vline & \R(t)    & \P(t)   & -I_{pN} 
\end{bmatrix}
}

% Notation for chapter 3b
%=========================================================================================
\newcommand{\EB}{
\begin{bmatrix}
E_1 & 0 \\
0   & 0 \\
0   & 0  
\end{bmatrix}
}

\newcommand{\AB}{
\begin{bmatrix}
A_1 & 0 \\
A_2 & 0 \\
C   & -I_p
\end{bmatrix}
}

\newcommand{\BB}{
\begin{bmatrix}
B_1 \\
B_2 \\
0
\end{bmatrix}
}

\def\EAB{
\begin{bmatrix}
E_1  & 0 \\
-A_2 & 0 \\
-C   & I_p
\end{bmatrix}
}

\newcommand{\EBT}{
\begin{bmatrix}
E_1 & 0 \\
-A_2   & 0 \\
-C  & I_p  
\end{bmatrix}
}

\newcommand{\ABT}{
\begin{bmatrix}
A_1 & 0 \\
\dot{A}_2 & 0 \\
\dot{C}   & 0
\end{bmatrix}
}

\def\BBT{
\begin{bmatrix}
B_1 & 0\\
\dot{B}_2 & B_2 \\
0 & 0
\end{bmatrix}
}

\def\ttA{\hat{\hA}}
\def\ttB{\hat{\hB}}

\def\IEAB{
\begin{bmatrix}
\col{E_1}{-A_2}^{-1} & \vline & 0 \\ \hline
C \col{E_1}{-A_2}^{-1} & \vline & I_p
\end{bmatrix}
}

\def\Elast{
\begin{bmatrix}
\col{E_1}{-A_2}^{-1}\col{A_1}{\dot{A}_2} & \vline & 0 \\ \hline
C\col{E_1}{-A_2}^{-1}\col{A_1}{\dot{A}_2} + \dot{C} & \vline & 0
\end{bmatrix}
}

\def\Blast{
\begin{bmatrix}
\col{E_1}{-A_2}^{-1}\col{B_1}{0}\\ \hline
C\col{E_1}{-A_2}^{-1}\col{B_1}{0}
\end{bmatrix}
}

\begin{document}
	
\title{Stability analysis of arbitrarily high-index positive delay-descriptor systems % \thanks{Grants or other notes
%about the article that should go on the front page should be
%placed here. General acknowledgments should be placed at the end of the article.}
}

%\subtitle{Characterizations of positive descriptor systems}

\titlerunning{Characterizations of positive descriptor systems}        % if too long for running head

\author{Phan Thanh Nam \and Ha Phi 
}

%\authorrunning{Short form of author list} % if too long for running head

\institute{Phan Thanh Nam \at
           Technische Universit\"at Berlin, Strasse de 17. Juni 136, Berlin, Germany \\
          \email{mehrmann@math.tu-berlin.de}
          \and
          Phi Ha \at
          Hanoi University of Science, VNU \\ 	Nguyen Trai Street 334, Thanh Xuan, Hanoi, Vietnam \\ 
          \email{haphi.hus@vnu.edu.vn}          
}

\date{Received: \today / Accepted: date}
% The correct dates will be entered by the editor


\maketitle

\begin{abstract}
This paper deals with the stability analysis of positive delay-descrip\-tor systems with arbitrarily high index. 
First we discuss the solvability problem (i.e., about the existence and uniqueness of a solution), which is followed by the study on characterizations of the (internal) positivity. Finally, we discuss the stability analysis. 
Numerically verifiable conditions in terms of matrix inequality for the system's coefficients are proposed, and are examined in several examples. 

\keywords{Positivity \and Delay \and Descriptor systems \and Strangeness-index . } 
% \PACS{PACS code1 \and PACS code2 \and more}
% \subclass{MSC code1 \and MSC code2 \and more}
\end{abstract}


%=========================================================================================
\noindent \textbf{Nomenclature} \\[.1cm]
%
\begin{tabular}{|c|c|}
	\hline
$\N$ ($\N_0$)	&  the set of natural numbers (including $0$)  \\
$\R$ ($\C$)   	&  the set of real (complex) numbers \\
$\C_{-}$		&  the set $\{\lb \in \C \ | \ Re \lb <0\}$ \\
$I$ ($I_n$)		&  the identity matrix (of size $n \times n$) \\
$x^{(j)}$       &  the $j$-th derivative of a function $x$ \\
$C^p([-\tau,0],\r^n)$ & the space of $p$-times continuously differentiable functions \\
& from $[-\tau,0]$ to $\r^n$ (for $ 0 \leq p \leq \infty$) \\
$\|\cdot\|_{\infty}$ & the norm of the Banach space $C^0([-\tau,0],\r^n)$. \\ 
$\im_+ W$ & the space $\{W w_1 \mbox{ for all } w_1 \in \r_+^n \}$. \\ 
\hline
\end{tabular}

%=========================================================================================
%Section 1

\section{Introduction} \label{sec1}
Our focus in the present paper is on the positivity and stability analysis of linear, constant coefficients {\it delay-descriptor systems} of the form
%
\begin{align}\label{delay-descriptor}
  E\dot{x}(t) &= A x(t) + A_d x(t-\tau) + B u(t), \ \mbox{ for all } t\in [t_0,t_f), \\
  y(t) &= C x(t), \notag 
\end{align}
%
where $E$, $A \in \r^{n,n}$, $B\in \r^{n,p}$, $C\in \r^{q,n}$, $x:[t_0-\tau,t_f) \rar \r^n$, $f:[t_0,t_f)\to \r^{n}$, and $\tau>0$ is a constant delay. \
Together with \eqref{delay-descriptor}, we are also concern with the associated \emph{zero-input/free system}
%
\be\label{free system}
E\dot{x}(t) = A x(t) + A_d x(t-\tau), \ \mbox{ for all } t\in [t_0,t_f).
\ee
%
Systems of the form \eqref{delay-descriptor} can be considered as a general combination of two important classes of dynamical systems, namely \emph{differential-algebraic equations (descriptor systems)} (DAEs)
%
\begin{equation}\label{eq1.2}
 E \dot{x}(t) = A x(t) + B u(t),
\end{equation}
%
where the matrix $E$ is allowed to be singular ($\det E=0$), and \emph{delay-differential equations} (DDEs) %
\begin{equation}\label{eq1.3}
 \dot{x}(t) = A x(t) + A_d x(t-\tau) + B u(t). 
\end{equation}
%
delay-descriptor systems of the form \eqref{delay-descriptor} have been arisen in various applications, see \citet{AscP95,Cam80,HalL93,ShaG06,ZhuP97} and the references there in. 
From the theoretical viewpoint, the study for such systems is much more complicated than that for standard DDEs or DAEs. The dynamics of DDAEs has been strongly enriched, and many interesting properties, which occur neither for DAEs nor for DDEs, have been observed for DDAEs \citet{Cam95c,DuLMT13,HaM12,HaM16}. Due to these reasons, recently more and more attention has been devoted to DDAEs, \citet{CamL09,Fri02,HaM12,HaM16,Mic11,ShaG06,TiaYK11,LinT15}. \\

[....] \\

The short outline of this work is as follows. Firstly, in Section \ref{sec2}, we briefly recall the solvability analysis to system \eqref{delay-descriptor}, which is followed by an imporant result about solution comparison for the free system \eqref{free system} (Theorem \ref{solution comparison 1}). 
Based on the explicit solution representation in Section \ref{sec2}, we characterize the posivity of system \eqref{delay-descriptor} in Section \ref{sec3}. We establish there algebraic, numerically verifiable conditions in terms of the system matrix coefficients. To follow, in Section \ref{sec4} we discuss further about the free system \eqref{free system} under biconditional requirements: stability and positivity. Finally, we conclude this research with some discussion and open questions. 

%%%%%%%%%%%%%%%%%%%%%%%%%%%%%%%  Section 2  %%%%%%%%%%%%%%%%%%%%%%%%%%%%%%%%%%
\section{Preliminaries} \label{sec2}

In this section we discuss the solvability analysis, including the solution representation and the comparison principal for the corresponding IVP to system \eqref{delay-descriptor}, which consists of \eqref{delay-descriptor} together with 
an initial condition
%
\begin{equation}\label{initial condition}
	x|_{[t_0-\tau,t_0]} = \vphi: [t_0-\tau,t_0] \rightarrow \r^{n}. 
\end{equation}
%
Here, $\vphi$ is a prescribed initial trajectory (preshape function), which is necessary to achieve uniqueness of solutions.
Without loss of generality, we assume that $t_0 = 0$ and $t_f = n_f \tau$, where $n_f \in \N$. 

\subsection{Existence, uniqueness and explicit solution formula}

It is well-known (e.g. \citet{DuLMT13}) that we may consider different solution concepts for system \eqref{delay-descriptor}.
The reason is, that $E(0)\dot{x}(0^+)$ which arises from the right hand side in \eqref{delay-descriptor} at $0$ may not be equal to $E(0)\dot{\vphi}(0^-)$.
%
Moreover, it has been observed in \citet{BakPT02,Cam80,GugH07} that a discontinuity of $\dot{x}$ at $t=0$ may propagate with time, and typically $\dot{x}$ is discontinuous at every point $j\tau, \ j\in\N_0$ or it may not even exist. To deal with this property of DDAEs, we use the following solution concept. % for \eqref{delay-descriptor}.
%
\begin{definition}\label{solution} Let us consider a fixed input function $u(t)$.\\
i) A function $x:[-\tau,\infty)\rar\r^n$ is called a \emph{piecewise differentiable solution} of \eqref{delay-descriptor}, if $Ex$ is piecewise continuously differentiable, $x$ is continuous and satisfies \eqref{delay-descriptor} at every  $t\in [t_0,t_f) \setminus \underset{j\in \N_0}{\cup} \{j\tau\}$. \\
ii) A function $x:[-\tau,\infty)\rar\r^n$ is called a \emph{classical solution} of \eqref{delay-descriptor}
if it is at least continuous and satisfies \eqref{delay-descriptor} at every  $t\in [t_0,t_f)$. 
\end{definition}

Throughout this paper whenever we speak of a solution, we mean a piecewise differentiable solution. Notice that, like DAEs, DDAEs are not solvable for arbitrary initial conditions, but they have to obey certain consistency conditions.
%
\begin{definition}\label{consistency} An initial function $\vphi$ is called \emph{consistent} with \eqref{delay-descriptor} if the associated initial value problem (IVP) \eqref{delay-descriptor}, \eqref{initial condition} has at least one solution.
	System \eqref{delay-descriptor} is called \emph{solvable} (resp. \emph{regular}) if for every consistent initial function $\vphi$,
	the IVP \eqref{delay-descriptor}, \eqref{initial condition} has a solution (resp. has a unique solution).
\end{definition}
%

Introducing sequences of matrix-valued and vector-valued functions $f_j$, $u_j$, $x_j$ for each $j\in \N$, on the time interval $[0,\tau]$ via
\begin{align*}
	f_j(t) &= f(t+(j-1)\tau), \   u_j(t) = u(t+(j-1)\tau), \\
	x_j(t) &= x(t+(j-1)\tau), \ x_0(t) := \vphi(t-\tau),
\end{align*} 
we can rewrite the IVP \eqref{delay-descriptor}-\eqref{initial condition} as a sequence of non-delayed descriptor systems
%
\begin{equation}\label{j-th DAE}
	E \dot{x}_j(t) = A x_j(t) + A_d x_{j-1}(t) + B u_j(t), 
\end{equation}
%
for all $t\in (0,\tau)$ and for all $j=1,2,...,n_f$. We notice, that for each $j$, the initial condition $x_j(0)$ is given due to the continuity of the solution $x(t)$ at the point $(j-1)\tau$, i.e.,
\begin{equation}\label{continuity condition}
	x_j(0) = x_{j-1}(\tau) \ .
\end{equation}
In particular, $x_1(0) = \phi(0)$ and the function $x_0$ is given. \\

It is well-known (see e.g. \citet{BelC63,HalL93}) that in general, time-delayed systems has been classified into three different types (retarded, neutral, advanced). For example, the 
time-delayed equation
%
\[
a_0 \dot{x}(t) + a_1\dot{x}(t-\tau) + b_0 x(t) + b_1 x(t-\tau) = f(t)
\]
%
is retarded if $a_0\not= 0$ and $a_1=0$; is neutral if $a_0\not= 0$, $a_1\not= 0$; is advanced if $a_0=0$, $a_1\not= 0$, $b_0 \not=0$. \ Obviously, this classification is based on the smoothness comparison between $x(t)$ and $x(t-\tau)$. In literature, not only the theoretical but also numerical solution has been studied mainly for non-advanced systems (i.e., retarded or neutral), due to their apperance in various applications. For this reason, in \cite{Ha15,HaM16,Ung18} the authors poposed a concept of \emph{non-advancedness} for \eqref{delay-descriptor} (see Definition \ref{def2} below).
We also notice, that even though not clearly proposed, due to the author's knowledge, so far results for delay-descriptor are only obtained for certain classes of non-advanced systems, e.g. \citet{AscP95,ShaG06,ZhuP97,ZhuP98,Mic11}.
%
\begin{definition}\label{def2}
A regular delay-descriptor system \eqref{delay-descriptor} is called \emph{non-advanced} if for any consistent and continuous initial function $\vphi$, there exists a piecewise differentiable solution $x(t)$ to the IVP \eqref{delay-descriptor}, \eqref{initial condition}.
\end{definition} 
%
%
\begin{definition}\label{regularity} Consider the DDAE \eqref{delay-descriptor}. The matrix triple $(E,A,B)$ is called \emph{regular} if the (two variable) \emph{characteristic polynomial} $\det(\lb E - A - \om B)$ is not identically zero. 
If, in addition, $B=0$ we say that the matrix pair $(E,A)$ (or the pencil $\lb E-A$) is regular.  
The sets $\si(E,A,B):= \{\lb \in \C \ | \ \det(\lb E -A-e^{-\lb \tau}B) = 0\}$ and $\rho(E,A,B)=\C\setminus \si(E,A,B)$ are called the \emph{spectrum} and the \emph{resolvent set} of \eqref{delay-descriptor}, respectively. 
\end{definition}
%
Provided that the pair $(E,A)$ is regular, we can transform them to the Kronecker-Weierstra\ss canonical form (see e.g. \citet{Dai89,KunM06}). That is, there exist regular matrices $W$, $T\in \R^{n,n}$ such that
%
\begin{equation}\label{KW form}
(E,A) = \left( W \m{I & 0 \\ 0 & N }T , \  W \m{ J & 0 \\ 0 & I}T \right) \ ,
\end{equation}
%
where $N$ is a nilpotent matrix of nilpotency index $\nu$. We also say that the pair $(E,A)$ has a \emph{differentiation index} $\nu$, i.e., $\ind(E,A) = \nu$.

\begin{remark}
Two concepts non-advancedness and differentiation index are independent. In details, a non-advanced system can have arbitrarily high index, as can be seen in the following example. 
\end{remark}

\begin{example}\label{example 1}
Consider the following systems with the parameters $\ep_1$, $\ep_2$. 
%
\begin{equation}\label{eq11}
\underbrace{
	\m{1 & 0 \\ 0 & 0} 
	% \m{0 & 1 &  & \\  & \ddots & \ddots & \\  &  & \ddots & 1 \\  &  &  & 0}
}_{E}  \dot{x}(t) 
= 
\underbrace{
	\m{0 & 1 \\ 1 & 0}
	% \m{ 1 & & & & \\ & \ddots & & & \\ & & & 1 & \\ & & & & 1}
}_{A}  x(t) 
+ 
    % \m{ 0 & & & \ep_1 \\ & \ddots & &  \\ & & 0 &   \\  & & & \ep_2} x(t-h) \ . 
 \underbrace{   \m{0 & \ep_1 \\ 0 & \ep_2} }_{A_d} x(t-\tau)  \ .
\end{equation}
%
It is well-known that in this example $\ind(E,A) = 2$. Furthermore, depending on the value of $\ep_2$, the system will be advanced (if $\ep_2\not=0$) and be non-advanced (if $\ep_2\not=0$). Analogously, one can construct a non-advanced system which has an arbitrarily high index. 
\end{example}

Let $E$ have index $\tnu$, i.e., $\ind(E,I_n)=\tnu$, the Drazin inverse $E^D$ of E is uniquely defined by the properties
%
\begin{equation}
E^DE = E E^D, \ E^D E E^D = E^D ,\ E^D E^{\tnu+1} = E^{\tnu} .
\end{equation}
%
\vskip -1pt
%
\begin{lemma}\label{lem1}\citet{KunM06} Let (E,A) be a regular matrix pair. Then for any $\lb \in \rho(E,A)$, two following matrices commute.
%
\begin{equation}\label{eq20}
\hE:=(\lb E-A)^{-1}E, \ \hA:=(\lb E-A)^{-1}A.
\end{equation}
%
Furthermore, the following commutative identities hold true.
%
\[ \hE \hA^D = \hA^D \hE, \ \hE^D \hA = \hA \hE^D, \ \hE^D \hA^D = \hA^D \hE^D \ . 
\]
%
\end{lemma}
We notice that the matrix products $\hE^D \hE$, $\hE^D \hA$, $\hE\hA^D$, $\hE^D \hB$, $\hA^D \hB$ do not depend on the choice of $\lb$ (see e.g. \citet{Dai89}). Furthermore, they can be numerically computed by transforming the pair $(E,A)$ to their Weierstrass canonical form \eqref{KW form} (see e.g. \citet{Varga19,Vir08}).

For any $\lb \in \rho(E,A)$, we denote 
%
\begin{equation}\label{eq21}
\hA_d :=(\lb E-A)^{-1} A_d, \  \hB:=(\lb E-A)^{-1}B \ .
\end{equation}
%
Making use of the Drazin inverse, in the following theorem we present the explicit solution representation of system \eqref{delay-descriptor}.

\begin{theorem}\label{sol. rep. DAE}
Consider the delay-descriptor system \eqref{delay-descriptor}. Assume that 
$(E, A)$ is a regular matrix pair with a differentiation index $ind(E, A) = \nu$. Let $\hE$, $\hA$, $\hA_d$, $\hB$ be defined as in \eqref{eq20}, \eqref{eq21}. Furthermore, assume that $u$ is sufficiently smooth. Then, every solution $x_j$ of the DAE \eqref{j-th DAE} has the form
%
\begin{align}\label{j-th solution}
x_j(t) =& \  e^{\hE^D\hA t} \hE^D\hE v_j + 
\int_{0}^{t} e^{\hE^D\hA (t-s)}  \hE^D \ \left( \hA_d x_{j-1}(s) + \hB u_j(s) \right) ds \notag \\
&+ (\hE^D\hE - I) \ \sum_{i=0}^{\nu-1} (\hE^D\hA)^i \hA^D \left( \hA_d x^{(i)}_{j-1}(t) + \hB u^{(i)}_{j}(t) \right) ,
\end{align}
%
for some vector $v_j \in \R^n$. 
\end{theorem}
\begin{proof}
The proof is straightly followed from the explicit solution of DAEs, see \cite[Chap. 2]{KunM06}.
\end{proof}
%
Making use of \eqref{continuity condition}, we directly obtain the following corollary.
%
\begin{corollary}
The solution $x(t)$ of system \eqref{delay-descriptor} is continuous at the point $(j-1)\tau$ if and only if the following condition holds.
%
\[ (\hE^D\hE - I) \ x_{j-1}(\tau) = (\hE^D\hE - I) \ \sum_{i=0}^{\nu-1} (\hE^D\hA)^i \hA^D \left( \hA_d x^{(i)}_{j-1}(0) + \hB u^{(i)}_{j}(0) \right) \ .
\]
%
In particular, for the preshape function $\varphi(t)$, we must require
%
\[ (\hE^D\hE - I) \ \left( \vphi(0) + \sum_{i=0}^{\nu-1} (\hE^D\hA)^i \hA^D \left( \hA_d \vphi^{(i)}(-\tau) + \hB u^{(i)}(0) \right) \right) = 0 \ .
\]
% 
\end{corollary}

Following from \eqref{j-th solution}, we directly obtain a simpler form in case of non-advanced system as follows.

\begin{corollary} Consider system \eqref{delay-descriptor} and assume that it is regular and non-advanced. Then, we have
%
\begin{align}\label{sol. formula non-advanced}
x_j(t) =& \  e^{\hE^D\hA t} \hE^D\hE v_j + 
\int_{0}^{t} e^{\hE^D\hA (t-s)}  \hE^D \left( \hA_d x_{j-1}(s) + \hB u_j(s) \right) ds \notag \\
&+ (\hE^D\hE - I) \ \left( \hA^D \hA_d x_{j-1}(t) + \sum_{i=0}^{\nu-1}  (\hE^D\hA)^i  \hA^D \hB u^{(i)}_{j}(t) \right) ,
\end{align}
%
Furthermore, the consistency condition at $t=0$ reads
%
\begin{equation}\label{consistency}
(\hE^D\hE - I) \left( \vphi(0) + \hA^D \hA_d \vphi(-\tau) + \sum_{i=0}^{\nu-1} (\hE^D\hA)^i \hA^D \hB u^{(i)}(0) \right) = 0 \ .
\end{equation}%
\end{corollary}

\subsection{A simple check for the non-advancedness}
Assume that the pair $(E,A)$ is regular with index $\ind(E,A)=\nu$. 
We want to give a simple check whether the free system \eqref{free system} is non-advanced or not. In analoguous to the case of DAEs \cite{BreCP96,KunM06}, we aim to extract the so-called \emph{underlying delay equation} of the form 
%
\begin{equation}\label{underlying DDEs}
\dot{x}(t) = \bA x(t) + \bA_{d0} x(t-h) + \bA_{d1} \dot{x}(t-h),
\end{equation}
%
from system \eqref{free system} and its derivatives, which read in details
%
\[
E x^{(i)}(t) = \bA x^{(i-1)}(t) + \bA_{d} x^{(i-1)}(t-h), \mbox{ for all } i=1,\dots,\nu.
\]
%
We rewrite these equations into the so-called \emph{inflated system}
%
\begin{align}\label{inflated}
& 
\underbrace{
\m{E & & & & \\ -A & E & & & \\ & & \ddots & \ddots & \\ & & & -A & E}
}_{\cE}	
\m{\dot{x} \\ \ddot{x} \\ \vdots \\ x^{(\nu+1)}}
= 
\underbrace{
\m{A & 0 & \hdots & & 0 \\ 0 & 0 & \hdots & & 0 \\ \vdots & \vdots & & & \vdots \\ 0 & 0 & \hdots &  & 0}
}_{\cA}	
\m{x \\ \dot{x} \\ \vdots \\ x^{(\nu)}} \notag \\
& \ + 
\underbrace{
\m{A_d & & & \\ & A_d & & \\  & & \ddots & \\ & & & A_d}
}_{\cA_d}	
\m{x(t-h) \\ \dot{x}(t-h) \\ \vdots \\ x^{(\nu)}(t-h)} \ .
\end{align}
%
Here the matrix coefficients are $\cE,\cA,\cA_d \in \R^{(\nu+1) n,(\nu+1) n}$. For the reader's convenience, below we will use MATLAB notations. System of the form \eqref{underlying DDEs} can be extracted from \eqref{inflated} if and only if there exists a matrix $P=\m{P_0 & P_1 & \dots & P_{\nu}}^T$ in $\R^{(\nu+1) n,n}$ such that
%
\begin{align*}
P^T \cE &= \m{ I_n \ & 0_{n,\nu n} } \\
P^T \cA_d &= \m{* \ & * \ & 0_{n,(\nu-1)n}}, 
\end{align*}
%
where $*$ stands for an arbitrary matrix. Consequently, $P$ is the solution to the following linear systems
%
\begin{equation}\label{adv. check eq.}
\m{\cE & \cA(:,2n+1 : end)}^T P = \m{I_n \ & 0_{n,\nu n} \ & 0_{n,(\nu-1)n} }^T \ .
\end{equation}
%
Therefore, making use of Crammer's rule we directly obtain the simple check for the non-advancedness of system \eqref{free system} in the following theorem.
%
\begin{theorem}\label{thm check advancedness}
Consider the zero-input descriptor system \eqref{free system} 
and assume that the pair $(E,A)$ is regular with index $\ind(E,A)=\nu$. 
Then, this system is non-advanced if and only if the following rank condition is satisfied 
%
\[
\rank \m{\cE^T \\ \cA(:,2n+1 : end)^T}
= \rank \left[ \pm{\cE^T \\ \cA(:,2n+1 : end)^T} \ \vline \ \pm{I_n \\ 0_{(2\nu-1)n,n} } \right]
\]
%
\end{theorem}
%
Theorem \ref{thm check advancedness} applied to the index two case straightly gives us the following corollary.
%
\begin{corollary}
Consider the zero-input descriptor system \eqref{free system} and assume that the pair $(E,A)$ is regular with index $\ind(E,A)=2$.  	
Then, system \eqref{free system} is non-advanced if and only if the following identity hold true.
%
\begin{equation}\label{check advanced}
\rank \m{E^T & -A^T & 0 \\ 0 & E^T & -A^T \\ 0 & 0 & A_d^T } =
n + \rank \m{E^T & -A^T \\ 0 & E^T \\ 0 & A_d^T} \ .
\end{equation}
%
\end{corollary} 

\begin{example}
Let us reconsider system \eqref{eq11} in Example \ref{example 1}. Numerical verification of non-advancedness via condition \eqref{check advanced} completely agrees with theoretical observation.
\end{example}

%%%%%%%%%%%%%%%%%%%%%%%%%%%%%%%  Section 2 %%%%%%%%%%%%%%%%%%%%%%%%%%%%%%%%%%
\subsection{Comparison principal} \label{sec2b}

In this part of Section \ref{sec2}, we will show how to generalize our result to delay-descriptor systems with time-varying delay of the following form
%
\begin{equation}\label{ltv delay-descriptor}
E\dot{x}(t) = A x(t) + A_d x(t-\tau(t)) + B u(t), \ \mbox{ for all } t\in [t_0,t_f),
\end{equation}
%
where the delay function $\tau(t)$ is preassumed continuous and bounded, i.e. $0<\underline \tau \leq \tau(t) \leq \bar \tau$ for all $t\geq 0$. Here $\utau$, $\btau$ are two positive constants.
%
Following \cite{HaM16}, it can be shown that the solution to system \eqref{ltv delay-descriptor} exists, unique and totally determined by any consistent initial function $\vphi$ such that $x(t) = \vphi(t)$ for all $-\btau \leq t \leq 0$.
Indeed, also making use of the method of steps, the solution $x$ is constructively built on consecutive interval $[t_{i-1},t_i]$, $i\in \N$ such that $0=t_0 < t_1<t_2<.... $ and
%
\[
t_i - \tau(t_i) = t_{i-1} \ .
\]
%
As shown in Theorems \ref{solution comparison 1}, \ref{solution comparison 2} below, we 
can directly generalize our result to systems with bounded, time varying delay.

\begin{theorem}\label{solution comparison 1}
Consider system \eqref{delay-descriptor} and assume that the corresponding constant delay system is positive and non-advanced. 
For a fixed input $u$, let $x(t)$ (resp. $\tx(t)$) be a state function corresponds to a preshape function $\vphi(t)$ (resp. $\tvphi(t)$). Furthermore, assume that 
$\vphi(t) \leq \tvphi(t)$ for all $t \in [-\btau,0]$. Then, we have $x(t) \leq \tx(t)$ for all $t\geq 0$.
\end{theorem}
\begin{proof}
Since the input is fixed and the system is non-advanced, the proof can be directly obtain as in the impulse-free case.
\end{proof}

%\begin{lemma}\label{lem2.1} If the system is non-advanced ...  
%It suffices to prove that if $u_j(t) \leq \tu_j(t)$ and $x_{j-1}(t) \leq \tx_{j-1}(t)$ for all $t\in[0,\tau]$ then it follows that $x_{j}(t) \leq \tx_j(t)$ for all $t\in[0,\tau]$.
%\end{lemma}
%
%By simple induction, making use of Lemma \ref{lem2.1}, we obtain the second solution comparison result.

\begin{theorem}\label{solution comparison 2}
Consider system \eqref{delay-descriptor} and assume that the corresponding constant delay system positive. 
Furthermore, assume that $(\hE^D\hE - I) \ (\hE^D\hA)^i  \hA^D \hB \geq 0$ for all $i=0,\dots,\nu-1$. \ Let $x(t)$ (resp. $\tx(t)$) be a state function corresponds to a reference input $u(t)$ (resp. $\tu(t)$)
and a preshape function $\vphi(t)$ (resp. $\tvphi(t)$). Furthermore, assume that the following conditions hold. \\
i)  $\vphi(t) \leq \tvphi(t)$ for all $t \in [-\tau,0]$, \\
ii) $u^{(i)}(t) \leq \tu^{(i)}(t)$ for all $t\geq 0$ and all $i\leq (\nu-1) \ \floor{t/\tau}$. 
Then we have $x(t) \leq \tx(t)$ for all $t\geq 0$.
\end{theorem}
\begin{proof}
The proof is also very simple.
\end{proof}

%Time-dependent delay will affect neither the positivity nor the stability of system \eqref{delay-descriptor}.

%%%%%%%%%%%%%%%%%%%%%%%%%%%%%%%  Section 3 %%%%%%%%%%%%%%%%%%%%%%%%%%%%%%%%%%
\section{Characterizations of positive delay-descriptor system} \label{sec3}

Since most systems occur in application are non-advanced, in this section we focus on the chracterization for positivity of non-advanced delay descriptor systems. We, furthermore, notice that the non-advancedness is a necessary condition for the stability (in the Lyapunov sense) of any time-delayed system, see e.g. \cite{HalL93,DuLMT13}. 

\begin{definition}
Consider the delay-descriptor system \eqref{delay-descriptor} and assume that it is non-advanced, and that the pair $(E,A)$ is regular with $\ind(E,A)=\nu$. We call \eqref{delay-descriptor}  positive if for all $t\geq 0$ we have $x(t) \geq 0$ and $y(t)\geq 0$ for any input function $u$ and any consistent initial function $\vphi(t)$ that satisfy two following conditions. \\
i)  $\vphi(t) \geq 0$ for all $t \in [-\tau,0]$, \\
ii) $u^{(i)}(t) \geq 0$ for all $t\geq 0$ and all $i\leq (\nu-1) \ \floor{t/\tau}$.
\end{definition}
%
Let us denote 
\[
\Ka_{\nu}(\hE\hA^D,\hA^D\hB):=\m{\hA^D\hB,\hE\hA^D \ \hA^D\hB,\dots,(\hE\hA^D)^{\nu-1} \hA^D\hB} \ .
\]
%
Since our systems is linear, time invariant coefficients, it would be sufficient to study the positivity on the first time interval $[0,\tau]$. Making use of \eqref{sol. formula non-advanced}, and let $j=1$, we can split the solution $x_1 = x|_{[0,\tau]}$ as follows
%
\begin{align}\label{eq16}
x_1(t) =& 
\underbrace{
e^{\hE^D\hA t} \hE^D\hE v_1 + 
\int_{0}^{t} e^{\hE^D\hA (t-s)} \hE^D \hA_d x_0(s) + 
(\hE^D\hE - I) \hA^D \hA_d x_0(t)
}_{x_{zi}(t)}  \notag  \\
 &+ \underbrace{
 \int_{0}^{t} e^{\hE^D\hA (t-s)}  \hE^D \hB u_j(s) \!+\!
 (\hE^D\hE - I) \sum_{i=0}^{\nu-1}  (\hE^D\hA)^i  \hA^D \hB u^{(i)}_{j}(t)}_{x_{zs}(t)} \ , 
\end{align}
%
where $x_{zi}(t)$ (resp. $x_{zs}(t)$) is often called (in the theory of linear systems) the \emph{zero input} (resp. \emph{zero state}) solution.

\begin{lemma}\label{Castelan'93}
	Let $F\in\r^{p,n}$ and $M\in\r^{n,n}$ and consider the linear
	system $\dot{z}(t)=Mz(t)$. Then, the following implication holds true:
	\[
	[Fz(0) \geq 0] \Rightarrow [Fz(t) \geq 0 \mbox{ for all } t\geq 0]
	\]
	if and only if there exists a Metzler matrix $H$ such that $FM=HF$.	
\end{lemma}

\begin{proposition}\citet{RamN12}\label{Rami12}
	The following statements are equivalent. \\
	i) The differential-algebraic equation $E\dot{x}(t) = A x(t)$ is positive. \\
	ii) There exists a Metzler matrix $H$ such that $\hE^D \hA = H \ \hE^D \hE$. \\
	iii) There exists a matrix $D$ such that $H:=\bar{A} + D \ (I-P)$ is Metzler.
\end{proposition}


\begin{lemma}\label{zero input lemma}
Consider the delay-descriptor system \eqref{delay-descriptor} and assume that it is non-advanced, and the pair $(E,A)$ is regular with index $\ind(E,A) = \nu$. 
Let the input $u = 0$. Then, system \eqref{delay-descriptor} has a solution $x(t) \geq 0$ for all $t\geq 0$ and all consistent initial function $\vphi(t) \geq 0$ if and only if 
the following conditions are satisfied. \\
i) There exists a Metzler matrix $H$ s.t. $\hE^D \hA = H \ \hE^D \hE$. \\
ii) $\hE^D\hA_d \geq 0$, $(\hE^D\hE - I) \hA^D\hA_d \geq 0$. 
\end{lemma}

\begin{theorem}\label{Thm positivity}
Consider the delay-descriptor system \eqref{delay-descriptor} and assume that it is non-advanced, and the pair $(E,A)$ is regular with index $\ind(E,A) = \nu$. Furthermore, assume that 
 $(\hE^D\hE - I) \ (\hE^D\hA)^i  \hA^D \hB \geq 0$ for all $i=0,\dots,\nu-1$. \ Then, system \eqref{delay-descriptor} is positive if and only if the following conditions hold. \\
i) $\hE^D\hA = H \ \hE^D\hE$ for some Metzler matrix $H$. \\
ii) $\hE^D\hA_d \geq 0$, \ $(\hE^D\hE - I) \hA^D\hA_d \geq 0$, \ $\hE^D\hB \geq 0$, \\
iii) $C$ is non-negative on the subspace 
%
\begin{equation}\label{reachable subspace}
	 {\cal X} : = \im_+ \left[ \hE^D\hE,\ (\hE^D\hE - I)\hA^D \hA_d, \ (\hE^D\hE - I) \ \Ka_{\nu}(\hE\hA^D,\hA^D\hB) \right] \ .
\end{equation} 
%
\end{theorem}
\begin{proof}
	$\Rightarrow$ Due to Lemma \ref{zero input lemma}, we only need to prove part 3. \\
	$\Leftarrow$ Quite simple. 
\end{proof}



If we restrict ourself to the non-delayed case (i.e. $A_d=0$), the direct corollary of Theorem \ref{Thm positivity} is straightforward. We, moreover, notice that this corollary has slightly improved the result \cite[Thm. 3.4]{Vir08}.

\begin{corollary}\label{Thm positivity - DAE version}
	Consider the descriptor system \eqref{eq1.2} and assume that the pair $(E,A)$ is regular with index $\ind(E,A) = \nu$. Furthermore, assume that the inequalities $(\hE^D\hE - I) \ (\hE^D\hA)^i  \hA^D \hB \geq 0$ hold true for $i=0,\dots,\nu-1$. \\
	Then, system \eqref{eq1.2} is positive if and only if the following conditions hold. \\
	i) $\hE^D\hA = H \ \hE^D\hE$ for some Metzler matrix $H$. \\
	ii) $\hE^D\hB \geq 0$, \\
	iii) $C$ is non-negative on the subspace ${\cal X}$.
\end{corollary}



%%%%%%%%%%%%%%%%%%%%%%%%%%%%%%%  Section 4 %%%%%%%%%%%%%%%%%%%%%%%%%%%%%%%%%%
\section{Stability of positive delay-descriptor system} \label{sec4}


%%%%%%%%%%%%%%%%%%%%%%%%%%%%%%%  Section 6 %%%%%%%%%%%%%%%%%%%%%%%%%%%%%%%%%%
\section{Conclusion}\label{conclusion}
In this paper, we have discussed the positivity of strangeness-free descriptor systems in continuous time. Beside that, the characterization of positive delay-descriptor systems has been treated as well. The theoretical results are obtained mainly via an algebraic approach and a projection approach. The projection approach investigates the positivity of a given descriptor system by the positivity of an inherent ODE obtained by projecting the given system onto a subspace. On the other hand, the algebraic approach derives an underlying ODE without changing the state, input and output. Then, studying these hidden ODEs is the key point. The main difficulty here is that the derivative of the input $u$ may occur in the new system. Despite their disadvantages, these methods can provide both necessary conditions and sufficient conditions. Beside these theoretical methods, the behaviour approach, which leads to some feasible conditions, is also implemented.

%========================================================================================
\vskip 0.2cm
\textbf{Acknowledgment} The author would like to thank the anonymous referee for his suggestions to improve this paper.
%=========================================================================================

%\bibliographystyle{abbrv}
\bibliographystyle{unsrtnat}
\bibliography{Phi_July_2020}

%============================================================================
%\appendix
% \section*{Appendix}
% \begin{lemma}(\cite{Vir08})  
	Consider the regular matrix pair $(E,A)$ and let $\hE$, $\hA$ be defined as in \eqref{eq20}. If for all $v \geq 0$ we have $e^{\hE^D\hA t} \hE^D\hE v\geq 0$ for all $t\geq 0$, then there exists $\a\geq 0$ such that $\hE^D\hA + \a \hE^D\hE \geq 0$.
\end{lemma}

\begin{lemma}\label{zero input lemma}
	Consider the delay-descriptor system \eqref{delay-descriptor} and assume that it is non-advanced, and the pair $(E,A)$ is regular with index $\ind(E,A) = \nu$. 
	Let the input $u = 0$. Then, system \eqref{delay-descriptor} has a solution $x(t) \geq 0$ for all $t\geq 0$ and all consistent initial function $\vphi(t) \geq 0$ if and only if 
	the following conditions are satisfied. \\
	1) $\hE^D\hA + \a \hE^D\hE \geq 0$ for some $\a \geq 0$. \\
	2) $\hE^D\hA_d \geq 0$, $(\hE^D\hE - I) \hA^D\hA_d \geq 0$. 
\end{lemma}


\begin{theorem}\label{Thm positivity}
	Consider the delay-descriptor system \eqref{delay-descriptor} and assume that it is non-advanced, and the pair $(E,A)$ is regular with index $\ind(E,A) = \nu$. Furthermore, assume that \\
	i) $(\hE^D\hE - I) \ (\hE^D\hA)^i  \hA^D \hB \geq 0$ for $i=0,\dots,\nu-1$, \\
	ii) $\hE^D\hE \geq 0$. \\ 
	Then system \eqref{delay-descriptor} is positive if and only if the following conditions hold. \\
	1) $\hE^D\hA + \a \hE^D\hE \geq 0$ for some $\a \geq 0$. \\
	2) $\hE^D\hA_d \geq 0$, $(\hE^D\hE - I) \hA^D\hA_d \geq 0$, $\hE^D\hB \geq 0$, \\
	3) $C$ is non-negative on the subspace 
	%
	\begin{equation}\label{reachable subspace}
		{\cal X} : = \im_+ \left[ \hE^D\hE,\ (\hE^D\hE - I)\hA^D \hA_d, \ (\hE^D\hE - I) \ \Ka_{\nu}(\hE\hA^D,\hA^D\hB) \right] \ .
	\end{equation} 
	%
\end{theorem}

\end{document}