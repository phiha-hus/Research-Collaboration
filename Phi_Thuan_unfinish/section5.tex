%%%%%%%%%%%%%%%%%%%%%%%%%%%%%%%%%%%%%%%%%%%%%%%%%%%%%%%%%%%%%%%%%%%%%%%%%%%%%%%%%%%%%%%%%%%%%%%%%%%%%%%%%%%%%%%%%%%%%%%%%%%%%%%%%%%%%%%%%%%%%%%%%%%%
%\section{Solvability of BVPs for second order SiDEs}\label{Sec5}
%
%In this section we consider the solvability analysis of boundary value problems (BVPs) for second order SiDEs. 
%Without loss of generality, we may assume that the considered system is already in the strangeness-free form
%%
%\begin{equation}\label{bvp}
%\m{A_{n,1} \\ 0 \\ 0} x(n+2) +  \m{B_{n,1} \\ B_{n,2} \\ 0} x(n+1) + \m{C_{n,1} \\ C_{n,2} \\ C_{n,3}} x(n) = \m{f_{1}(n) \\ f_{2}(n) \\ f_{n,3} } \ 
%\mbox{ for all } n\in \N(n_0,n_f-2).
%\end{equation}
%%
%The boundary condition that we are concerned with is of the following form 
%%
%\begin{equation}\label{boundary}
%C x(n_0) + D x(n_f) = b.
%\end{equation}
%%
%The classical method to analyze the solvability of BVPs is making use of Green function. This, however, has only been considered for first order ODEs and DAEs, but  neither for high order DAEs nor for SiDEs. In this section, we aim to complete this research gap.
%